\documentclass{article}

\usepackage[main=english,vietnamese]{babel}
\usepackage[T1]{fontenc}
\usepackage[utf8]{inputenc}
\usepackage[sexy]{evan}
\usepackage{matchsticks}
\usepackage{wrapfig}
\usepackage{listings}

\newtheorem{hint}{Hint}

\title{Digits of a number}
\author{Nghia Doan}
\date{\today}

\begin{document}

\maketitle

In this article, we look at some problems with digits of number and the approaches to solve them using different methods.

\section{Divisibility rules}

In this section, we use the divisibility rules,
i.e. what digits should a number have in order to be divisible by $3, 4,$ or $8,$ and so on.

\begin{example*}[One]
    What is the greatest multiple of $8$ whose digits are all different?
\end{example*}

\begin{soln}
    The \textit{divisibility rule for $8$} states that
    the last three digits of a multiple of $8$ must be divisible by $8$.
    To create the largest $8-$digit number, the last three digits must be $0, 1,$ and $2.$
    Thus, the largest $3-$digit multiple of $8$ with those digits is $120.$
    Thus the desired number is $\boxed{9876543120.}$
\end{soln}
    
\begin{example*}[Two]
    What is the least multiple of $36$ that contains only digits $4$ and $5.$
\end{example*}

\begin{soln}
    Divisibility rule for $9$ state that the sum of digits of the number must be $9, 18, \ldots.$
    Let examine the sum of the digits from the least possible value $9$ and then going up.
    If the sum is $9,$ then $45$ or $54$ are not divisible by $4,$ so $9=4+5$ is not a possible sum.
    If the sum is $18,$ the $2-$digit multiple of $4$ can be made from two pairs of $4$ and $5$ is $44.$
    Thus the number is $\boxed{5544}.$
\end{soln}

\bigbreak

\noindent\rule{16.5cm}{0.4pt}

\begin{exercise*}[Three]
    \label{exercise:three}
    Find a $7-$digit number containing only digits $2$ or digits $3$ such that 
    there are more of digits $2$ than of digits $3$ and the number is divisible by both $3$ and $4.$
\end{exercise*}

\newpage

\section{Remainders of a perfect powers}

In this section, we look at the remainders of a perfect power - a perfect square, a perfect cube, or a higher power of integer
- when divided by an integer such as $3, 4, 8,$ or $9,$ and so on.

\begin{example*}[Four]
    Is there a $5-$digit perfect square whose sum of digits is $29$?
\end{example*}

\begin{soln}
    A perfect square is divisible by $3$ or has a remainder of $1$ when divided by $3$ (why?).
    Since the remainder of a number when divided by $3$ is the same as the remainder of its sum of digits when divided by $3,$
    and $29$ has a remainder of $2$ when divided by $3$ so there is no such number.
\end{soln}

\begin{example*}[Five]
    Find the perfect cube \textbf{n} such that all digits of $n$ are $9$ except the unit digit, which is $5.$
\end{example*}

\begin{soln}
    There is no such perfect cube since a perfect cube has a remainder $0, 1,$ or $8$ when divided by $9.$
\end{soln}

\bigbreak

\noindent\rule{16.5cm}{0.4pt}

\begin{exercise*}[Six]
    \label{exercise:six}
    Find $\textbf{n} > 3$ such that the $(n+1)$-digit binary number $\overline{10\ldots01_2}$ is a perfect power of $3.$
\end{exercise*}

\section{Digits as variables}

In this section, we use some algebra tools to establish equations for the digits of a number,
then solving those equations to obtain the value for them.

\begin{example*}[Seven]
    Digits \textbf{a}, \textbf{b}, and \textbf{c} are used to form $3-$digit numbers $\overline{abc}, \overline{bca},$ and $\overline{cab}.$
    The sum of these numbers is $1332,$ find $a+b+c.$
\end{example*}

\begin{soln}
    $\overline{abc} = 100a + 10b + c,$ similarly with others. Their sum is $111(a+b+c)=1332,$ $a+b+c=12.$
\end{soln}

\begin{example*}[Eight]
    Find all $4-$digit number \textbf{n} whose sum of digits is $2010-n.$
\end{example*}

\begin{soln}
    Let $n=\overline{abcd}.$ Then $1001a+101b+11c+2d=2010.$ If $a=1,$ then $b=9,$ so $11c+2d=100,$ so $c=8,d=2.$
    If $a=2,$ then $b=c=0,d=4.$ The solutions are $\boxed{1982,\ 2004}.$
\end{soln}

\bigbreak

\noindent\rule{16.5cm}{0.4pt}

\begin{exercise*}[Nine]
    \label{exercise:nine}
    Find a potitive integer \textbf{a} such that $(1+2+\ldots + a)- 1000a$ is a $3-$digit number. 
\end{exercise*}

\newpage

\section{The last digits of a number}

In this section, we show the use of so-called modular arithmetic in the easiest possible way to find the last digits of some numbers.

\begin{example*}[Ten]
    What is the last digit of $\left(... \left((7)^7\right)^7... \right)^{7}$?
    There are $1001$ digits $7.$
\end{example*}

\begin{soln}
    By testing $7 \equiv 7 \Mod{10}, 7^7 = (7)(7^2)^3 \equiv -7 \Mod{10}, (7^7)^7 \equiv (-7)^7 \equiv 7 \Mod{10}, \ldots$
    By Induction Principle, it can be proved that the last digit of the generic expression is $7$ if it has an odd amount of $7,$
    otherwise it is $3.$ The given one has an odd number of $7,$ so its last digit is $\boxed{7.}$  
\end{soln}

\begin{example*}[Eleven]
    In how many zeros can the number $1^n+2^n+3^n+4^n$ end for \textbf{n} positive integer?
\end{example*}

\begin{soln}
    For $n=1,$ and $2,$ the sum ends in one and two zeros.
    Now, for all $n \ge 3,$ $2^n, 4^n$ are divisible by $8,$
    and $1^n + 3^n$ congruent to $2$ or $4$ modulo $8$.
    Thus, the sum cannot end in three or more zeros.
\end{soln}

\bigbreak

\noindent\rule{16.5cm}{0.4pt}

\begin{exercise*}[Twelf]
    \label{exercise:twelf}
    Find the last five digit of $5^{1981}.$
\end{exercise*}

\section{Hints to the exercises}

\begin{hint}[\nameref{exercise:three}]
    First find the last two digits based on divisibility rule for $4.$
    Then find the number of digits $2$ in the first five digits.
\end{hint}

\begin{hint}[\nameref{exercise:six}]
    Let $\overline{10\ldots01_2} = 2^n + 1 = 3^m.$ Then casework based on the parity of $m.$
\end{hint}

\begin{hint}[\nameref{exercise:nine}]
    Investigate two cases, $a < 1999$ and $a \ge 2000.$
\end{hint}

\begin{hint}[\nameref{exercise:twelf}]
    Find the last $5$ digits of $5^{1981} - 5^5 = 5^5(5^{1976}-1).$
\end{hint}

\end{document}