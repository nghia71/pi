\documentclass{article}

\usepackage[main=english,vietnamese]{babel}
\usepackage[T1]{fontenc}
\usepackage[utf8]{inputenc}
\usepackage[sexy]{evan}
\usepackage{matchsticks}
\usepackage{wrapfig}
\usepackage{listings}

\newtheorem{hint}{Hint}

\title{Making the right statement}
\author{Nghia Doan}
\date{\today}

\begin{document}

\maketitle

\begin{example*}[Angels and Demons]

    On the Island of Knights and Liars, there are two types of people:
    the Knights, who always tell the truth, the Liars, who always lie.
    On the day of the Festival of Life, some angels and demons visited the Island of Knights and Liars.
    It is known that the angels always tell the truth and the demons always lie.

    It was a coincidence that Lan visited the island on that same day.
    She met someone who made a statement so that she knew for sure that he was an angel
    (perhaps because he is really handsome?)

    What was the statement?
\end{example*}

\begin{soln}
    The statement is \textit{I am not a Knight.}
    Neither a Liar, a Knight, nor a demon could make such statement.
    Thus the one who made the statement should be an angel.
\end{soln}

\begin{example*}[Three-word question]

    Four brothers named An, Binh, Chi, and Danh are quadruplets indistinguishable in apperance.
    \begin{itemize}[topsep=0pt, partopsep=0pt, itemsep=0pt]
        \ii An is an \textit{acurrate truth-teller.}
        \ii Binh is an \textit{inacurrate truth-teller}, meaning
        he is totally deluded in all his beliefs but always states honestly what he does believe.
        \ii Chi is an \textit{acurrate lier,} meaning
        all of his beliefs are correct, but he lies about every one of them.
        \ii Danh is an \textit{inacurrate lier}, he is both deluded and dishonest, meaning
        he will try to give you false information but is unable to.
    \end{itemize} 

    An and Binh are both married; the other two brothers are not.
    An and Chi are both rich; the other two brothers are not. 
    One day you meet one of the brothers. 

    Your task to find out whether he is married by asking a three-word question.
    \textit{For example, "are you married?" would be such a question.}
\end{example*}

\begin{soln}
    One of the solutions would be \boxed{\textit{Are you rich?}}
    Note that An and Binh both answer it with \textit{Yes},
    while Chi and Danh would answer with \textit{No}.
    So if the answer is \textit{Yes}, then that brother is married, otherwise he is unmarried.
\end{soln}

Now, lets change a bit the situation to make it more challenging.
Lets assume that on the Island of Knights and Liars,
there are exactly three types of people:
the Knights, who always tell the truth, the Liars, who always lie,
and normals, who sometimes lie and sometimes tell the truth.

\begin{example*}[Marrying the right prince]
    
    Mai visited the island and met three handsome princes.
    It is known that \textit{one of them is a Knight}, \textit{one of them is a normal},
    and \textit{one of them is a Liar.}
    It is known that the normal one is a \textit{werewolf},
    who used to devour people at full moon.
    It is also known that \textit{the Liar is younger than the normal},
    and \textit{the normal is younger than the Knight.}
    
    All three princes ask for Mai's hand.
    She would agree for the Knight one or even the Liar one, but want to avoid the normal one at any cost.
    She can \textbf{only ask a single question} from \textit{only one of the princes}
    who she can freely choose among the three,
    and the question can only be answered with \textit{Yes} or \textit{No.}

    What would be her question?
\end{example*}

\begin{soln}
    First, let sort the princes by age, 
    \[ 
        \text{Knight\ } > \text{\ normal\ } > \text{\ Liar}.
    \]
    
    Now, Mai can ask any of the princes, let say $A,$ by pointing to the two princes,
    indicating them $B$ and $C,$ 
    \[
        \boxed{\text{Is B older then C?}}
    \]

    \textit{Case 1:} if $A$ is a Knight.
    The answer is \textit{Yes} if $B$ is normal, $C$ is a Liar;
    \textit{No} if $B$ is a Liar, $C$ is normal.

    \textit{Case 2:} if $A$ is a Liar.
    The answer is \textit{Yes} if $B$ is normal, $C$ is a Knight;
    \textit{No} if $B$ is a Knight, $C$ is normal.

    From the analysis of the answers in the first two cases,
    Mai can safely marry $C$ if the answer is \textit{Yes},
    and marry $B$ if the answer is \textit{No.} 

    \textit{Case 3:} if $A$ is a normal. It does not matter what she said.
    Mai can marry any of the other two.
\end{soln}

\begin{example*}[Vampires in Transylvania]
    In Transylvania, about half of the inhabitants are humans and another half are vampires.
    An ongoing infection made some of inhabitants insane. The rest are still sane.
    All inhabitants look pretty much alike.
    The only difference is the distinct behavior in belief and truth-telling:
    \begin{enumerate}[topsep=0pt, partopsep=0pt, itemsep=0pt]
        \ii sane humans make only true statements,
        \ii insane humans uncontrollably lie,
        \ii sane vampires always lie, and
        \ii insane vampires always tell the truth.
    \end{enumerate}
    
    \textit{For example, if you ask the inhabitants whether the earth is round,
    a sane human knows the earth is round and truthfully say so,
    an insane human believes the earth is not round and says it is not round,
    a sane vampire knows the earth is round, but will then lie and say it isn't,
    and an insane vampire believes the earth is not round and then lies and say it is round.}

    Detective Benny goes to Transylvania.
    What question should he ask a Transylvanian to be answered with a \textit{Yes}
    regardless of the type the inhabitant is? 
\end{example*}

\begin{soln}
    The question \textit{Do you believe you are a human?} will have \textit{Yes} as an answer.

    \begin{enumerate}[topsep=0pt, partopsep=0pt, itemsep=0pt]
        \ii sane humans make only true statements, so the answer must be \textit{Yes.}
        \ii insane humans uncontrollably lie, he thought that he is not a human, so he lies,
        thus the answer must be \textit{Yes.}
        \ii sane vampires always lie, he knew that he is not a human, so he lies,
        thus the answer must be \textit{Yes.}
        \ii insane vampires always tell the truth, he thought that he is not a vampire,
        he tells the truth, hence the answer must be \textit{Yes.}
    \end{enumerate}

    Another cool solution is to ask \textit{Are you truthful?}.
    It is interesting, but all of them will answer this question with a \textit{Yes.}
\end{soln}

\end{document}