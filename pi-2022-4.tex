\documentclass{article}

\usepackage[main=english,vietnamese]{babel}
\usepackage[T1]{fontenc}
\usepackage[utf8]{inputenc}
\usepackage[sexy]{evan}
\usepackage{matchsticks}
\usepackage{wrapfig}
\usepackage{listings}

\title{The Power of Mathematical Reasoning}
\author{Nghia Doan \& Catherine Doan}
\date{\today}

\begin{document}

\maketitle

\section*{What is a mathematical statement?}

In this article, we discuss mathematical reasoning by a number of examples.
A \textbf{mathematical statement} is a declarative sentence that is \textit{either objectively true or false but not both.}
The key is that there must be \textit{no ambiguity.}
So, a sentence such as “\textit{The weather is beautiful}” is not a mathematical statement, since
whether the sentence is true or not is a matter of opinion.
A question such as “\textit{Is it raining?}” is not a mathematical statement
because it is a question, which is not declaring or asserting that something is true or false.
A mathematical statement is also sometimes called a \textbf{proposition.}

\begin{remark*}
    There are many examples of ambiguous statements. A famous example is \textit{the liar paradox.}
    
    \textit{A person said "I am a liar."
    If this was true then he was a liar, thus what he said was not true, therefore he was a truth-teller.
    This is a contradiction because he could not be both liar and truth-teller at the same time.
    If this was untrue, then he told the truth, which meant that he was a liar. The same contradiction applies.}
\end{remark*}

\begin{exercise*}[Unstoppable cannonball hits an immovable post]
    \label{exercise:pi-2022-4-p0}
    By unstoppable cannonball we shall mean a cannonball which knocks over everything in its way.
    By an immovable post we shall mean a post which cannot be knocked over by anything.
    
    What happens if a unstoppable cannonball hits an immovable post?
\end{exercise*}

\section*{It is so obvious, $\ldots$ or is it?}

Let's start with a few classic examples.
These problems are simple, but don't let them fool you.

\begin{example*}[Who lied?]
    \label{example:pi-2022-4-p1}
    Two American Indians were sitting on a log - a big Indian and a little Indian.
    The big Indian was smoking and the little Indian played with a toy.
    The little Indian was the son of the big Indian,
    but the big Indian was not the father of the little Indian.

    How is it possible?
\end{example*}

\begin{soln}[\nameref{example:pi-2022-4-p1}]
    The fact that the big Indian was smoking should not lead to the assumption that the big Indian was a male.
    Since the little Indian was the son of the big Indian,
    therefore the big Indian was \framebox{the mother} of the little Indian.
\end{soln}

\begin{example*}[Whose picture I am looking at?]
    \label{example:pi-2022-4-p2}
    A man was looking at a portrait. Someone asked him "Whose picture are you looking at?"
    He replied: "Brothers and sisters have I none, but this man's father is my father's son."
    (\textit{This man's father means, of course, the father of the man in the picture.})

    Whose picture was the man looking at?
\end{example*}

\begin{soln}[\nameref{example:pi-2022-4-p2}]
    The man made a statement that contains two parts, each a seperate statement.
    In the first one, he stated that he had neither brothers nor sisters.
    In the second one, the phrase "\textit{my father's son}" implies that the person in the phrase
    is one of his brother or himself.
    However, the first statement says he has no brother, therefore that person is himself.
    Thus, he is the man's father and the person in the painting is \framebox{his son.}
\end{soln}

\begin{example*}[How could she set the clock?]
    \label{example:pi-2022-4-p3}
    The clock in Lilian's house has stopped. Lilian wanted to set the correct time,
    but she was unable to do that, since there is the only one clock in the house.
    That evening, she went to visit her friend, Bianca,
    who has an excellent wall clock that always shows correct time.
    When Lilian came home after the visit, she was able to set the correct time on her clock.

    How did she do it?
\end{example*}

\begin{soln} \nameref{example:pi-2022-4-p3}
    Lilian checked the time displayed on her clock twice: when she left the house and when she came back.
    The difference of these two values will be equal to the total time when she was away.
    Assume that she walked to Bianca house, took a look at the clock and promptly left,
    the difference above is \textit{twice the length of the trip} between the two houses.
    By marking the time shown on Bianca's clock,
    she should add that to the length of the trip and set the time on her clock.
\end{soln}

\begin{example*}[Defeat King Haggard]
    \label{example:pi-2022-4-p4}
    There are ten death-water wells in the far, far away Dark Forest.
    The wells are marked by signs with number from 1 to 10.
    The water from these wells look and taste like normal water;
    however, even a single gulp of it is lethal - unless one has an antidote.
    The only known antidote is water from a higher-numbered well.
    (\textit{For example, if you had a sip of water from the well number 6,
    you could save yourself by drinking from the well 7, 8, 9, or 10.})    
    Unfortunately, there is no antidote for the water from the tenth well.
    While the first nine wells are easily accessible,
    the tenth one is located in the castle of King Haggard, the ruler of the Dark Forest.
    
    Prince Martin challenged the King to the death-water duel.
    By the rules of the duel, each participant should drink a cup of water offered by the opponent.
    King Haggard eagerly agreed. His plan was to use his tenth death well;
    he reasoned that water from this well will neutralize any drink offer by Martin,
    and will definitely kill the prince.
    The duel took place as planned. Each participant drank the liquid offered by the opponent.
    To everybody surprise and delight. Prince Martin lived and King Haggard died.
    
    How did this happen?
\end{example*}

\begin{soln} \nameref{example:pi-2022-4-p4}
    Prince Martin \textit{gave King Haggard a drink of clear water.}
    Thus instead of curing him, the water from the tenth well poisons him.
    Prince Martin, before the duel, drank some water from a death well with a number smaller than 10.
    Therefore \textit{Haggard's drink worked as antidote for Prince Martin.}
\end{soln}

\bigbreak

\begin{exercise*}[Which one is correct?]
    \label{exercise:pi-2022-4-p5}
    Which phrase is correct: "the yolk \textit{is} white" or "the yolk \textit{are} white."?
\end{exercise*}

\begin{exercise*}[How can they see each other?]
    \label{exercise:pi-2022-4-p6}
    Two camels were facing in the opposite direction.
    One was facing due east and the other one was facing due west.

    How can they manage to see each other, without walking, turning around, or even moving their heads?
\end{exercise*}

\newpage 

\section*{If this then that}

Life is not always simple. Sometimes, when the problem at hand is more complicated, 
looking at all possible cases, or caseworking, can help to find the correct answer.

\begin{example*}[Who is younger?]
    \label{example:pi-2022-4-p7}
    A brother and a sister were once asked who was younger. "I am older," said the brother.
    "I am younger," said the sister.

    Now, if you knew that \textit{at least one of them lied,}
    then who was younger?
\end{example*}

\begin{soln}[\nameref{example:pi-2022-4-p7}]
    There are two different approaches to this problem.

    The first one is to \textit{evaluate all possible truths} and see which one matches the given conditions.
    There are two cases: the brother was younger, or the sister was younger.  
    \begin{itemize}[topsep=0pt, partopsep=0pt, itemsep=0pt]
        \ii \textit{Case 1:} If the brother was younger, then the brother lied, and so did the sister.
        \ii \textit{Case 2:} If the sister was younger, then both of them told the truth, which would not be possible.
    \end{itemize}    
    Thus, the first case is correct, the brother was younger, and they both lied.
    
    Now, the second approach is to \textit{evaluate all possible assumptions},
    in this case whether the brother (and then the sister) told the truth.
    \begin{itemize}[topsep=0pt, partopsep=0pt, itemsep=0pt]
        \ii \textit{Case 1:} If the brother told the truth, then the sister did too.
        This contradicts the condition that at least one of them lied.
        \ii \textit{Case 2:} If the brother lied, then the sister lied too.
        This satisfies the condition, and therefore it is the correct one.
    \end{itemize}    

    Thus, \framebox{the brother was younger.}
\end{soln}

\begin{example*}[Who stole what?]
    \label{example:pi-2022-4-p8}
    Father was not happy when he heard that Mother could not make his favourite cake because butter, eggs and milk all were stolen.
    Mother recalled seeing Chipmunk, Groundhog, and Sparrow sneaking out of the kitchen when she came into it.
    Everyone was carrying something, but she couldn't tell who was carrying what.
    After a brief investigation all the ingredients were found at the homes of Chipmunk, Groundhog, and Sparrow.
    Here were what they said,
    \begin{itemize}[topsep=0pt, partopsep=0pt, itemsep=0pt]
        \ii Chipmunk: Groundhog stole the butter.
        \ii Groundhog: Sparrow stole the eggs.
        \ii Sparrow: I stole the milk.
    \end{itemize}
    As it happened, \textit{the one who stole the butter told the truth} and \textit{the one who stole the eggs lied.}

    Who stole what?
\end{example*}

\begin{soln} \nameref{example:pi-2022-4-p8}
    Let's try casework on \textit{who stole the butter.}
    \begin{itemize}[topsep=0pt, partopsep=0pt, itemsep=0pt]
        \ii \textit{Case 1:} Assume that Sparrow have stolen the butter. 
        Then she would be the truth-teller, but she said she stole the milk.
        \ii \textit{Case 2:} If Chipmunk stole the butter, then what she said about Groundhog is true,
        thus Groundhog stole the butter, which is a contradiction.
        \ii \textit{Case 3:} Groundhog must have stolen the butter. 
        Thus what Groundhog said was true, so Sparrow stole the eggs, therefore Chipmunk stole the milk.
    \end{itemize}    
    The answer is \framebox{Groundhog stole the butter, Sparrow stole the eggs, and Chipmunk stole the milk.}
\end{soln}

\bigbreak

\begin{exercise*}[Which room has the tiger?]
    \label{exercise:pi-2022-4-p9}
    Karl was captured while sneaking into the Kingdom of the Hungry Tigers.
    He was lead in front of three rooms, each with a separate door marked with a sign,
    as shown below in \Cref{fig:pi-2022-4-p9}.
    \begin{figure}[h]
        \centering
        \begin{tabular}{|c|c|c|}
        \hline
        I & II & III \\
        Room III is empty & The tiger is in Room I & This room is empty \\ \hline
        \end{tabular}
        \caption{\nameref{exercise:pi-2022-4-p9}}
        \label{fig:pi-2022-4-p9}
    \end{figure}
    A large treasure chest was placed in one of the rooms and a hungry tiger in another.
    A beautiful girl told him that,
    \begin{enumerate}[topsep=0pt, partopsep=0pt, itemsep=0pt]
        \ii The sign on the door of the room containing the treasure was true,
        \ii The sign on the door of the room with the tiger was false, and
        \ii The sign on the door of the empty room could be either true or false.
        \ii If he opens the room with the tiger, he will be eaten.
        \ii If he opens the room with the chest, they will set him free and give him the chest.
    \end{enumerate}

    Which room has the tiger?
\end{exercise*}

\newpage 

\section*{Knights and Liars, Patients and Doctors, and other stories}

Classic stories have some twists. Inductive reasoning leads us to a general conclusion.
Deductive reasoning applies a general assumption to a specific case.

\begin{example*}[Who killed the dragon?]
    \label{example:pi-2022-4-p10}
    On the Island of Knights and Liars, there are two types of people:
    the knights who always tell the truth and the liars who always lie.
    Three mighty warriors live on the island.
    Their names are Albert, David, and Victor.
    Two of them are Liars, and one is a Knight.
    The friends keep their affiliations secret (so nobody know who's what).

    One of their great deeds was a battle with a terrible dragon
    that was terrorizing the students of the Math, Chess, and Codding Club.
    Not much is known about this battle except that the dragon was slayed by a Knight.
    In a recently discovered letter, Albert stated that Victor had slayed the dragon
    and David were merely watching.

    Who actually killed the dragon?
\end{example*}

\begin{soln} \nameref{example:pi-2022-4-p10}
    If Albert is a Knight, then Victor slayed the dragon.
    Because the dragon slayer was a Knight, this would make Victor a Knight,
    which is impossible because only one of them is a Knight.
    Therefore Albert is a Liar, which means that Victor did not kill the dragon.
    
    Thus, \framebox{David was the dragon-slayer.}
\end{soln}

\begin{example*}[Who were removed?]
    \label{example:pi-2022-4-p11}
    In the Mental Hospital of Geniuses there are only doctors and patients.
    Each of these inhabitants is sane or insane, but cannot be both.
    The sane people were a hundred percent accurate in all of their beliefs, 
    and the insane people were a hundred percent inaccurate in all of their beliefs.
    After being in the hospital for a while, some patients got cured and became sane.
    Unfortunately some doctors lost their minds and became insane.
    The sane patients and insane doctors, if found, were removed from the hospital.
    
    One day, Inspector Melanie visited the hospital in order to determine who should be removed.
    She interviewed three people $A$, $B$, and $C$:
    \begin{itemize}[topsep=0pt, partopsep=0pt, itemsep=0pt]
        \ii $A$ said $B$ was insane.
        \ii $B$ said $A$ is a doctor.
        \ii $C$ said $B$ is a patient and $A$ is insane.
    \end{itemize}

    Did Inspector Melanie remove $A$, $B$, or both of them?
\end{example*}

\begin{soln} \nameref{example:pi-2022-4-p11}
    Suppose that $A$ was sane. Then, what $A$ said was true, so $B$ was insane, and therefore $B$'s belief that $A$ was a doctor
    was false, thus, $A$ is a sane patient and should be removed.
    $C$ said $A$ was insane, so $C$ was insane, thus $B$ was a doctor, thus $B$ should be removed as well.
    
    Now, if $A$ was insane, then $B$ was sane, and therefore $A$ was a insane doctor and should be removed.
    $C$ said $A$ was insane, so $C$ was sane, thus $B$ was patient, so $B$ should also be removed.
    
    Thus, \framebox{both $A$ and $B$ should be removed.}
\end{soln}

\bigbreak

\begin{exercise*}[Was Sam awake?]
    \label{exercise:pi-2022-4-p12}
    Whenever Rob the bandit is asleep, everything he believes is wrong.
    In other words, everything Rob believes in his sleep is false.
    On the other hand, everything he believes while he is awake is true.
    Last night at $10$ o'clock sharp,
    while guarding the newly captured treasure chest,
    Rob believed that he and Sam the thug were asleep at that time.
    
    Was Sam asleep or awake at that time?
\end{exercise*}

\newpage 

\section*{Solutions to the Exercises}

\begin{soln} \nameref{exercise:pi-2022-4-p0}
    This is an example of a statement that cannot be determined as true or false. 
\end{soln}

\begin{soln} \nameref{exercise:pi-2022-4-p5}
    Actually, the yolk is \textit{yellow.}
\end{soln}

\begin{soln} \nameref{exercise:pi-2022-4-p6}
    The two camels were \textit{facing each other} the whole time, hence facing in opposite directions.
\end{soln}

\begin{soln} \nameref{exercise:pi-2022-4-p9}
    Instead of looking for the tiger, let's \textit{look for the treasure, because the sign on its room is true.}
    
    It cannot be in room II, because then according to the sign, room I has the tiger, room III is empty.
    Thus, the sign on room I is true, which contradicts that the sign on the room with the tiger is false.
    
    It cannot be in room III, because the according to the sign, room III is empty.
    Therefore the treasure is in room I, room III is empty, and the tiger is in room II.
\end{soln}

\begin{soln} \nameref{exercise:pi-2022-4-p12}
    If Rob the bandit was awake at the time, he could not have had \textit{the false belief
    that both he and Sam the thug were asleep.}
    Therefore he was asleep. This means that his belief was false,
    so it is not true that both were asleep.
    Therefore Sam was awake.
\end{soln}

\end{document}