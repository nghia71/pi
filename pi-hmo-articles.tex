\documentclass{article}

\usepackage[main=english,vietnamese]{babel}
\usepackage[T1]{fontenc}
\usepackage[utf8]{inputenc}
\usepackage[sexy]{evan}
\usepackage{matchsticks}
\usepackage{wrapfig}
\usepackage{listings}

\begin{otherlanguage*}{vietnamese}
\title{Các Kỳ Thi Toán Học Tại Hungary}
\end{otherlanguage*}

\author{Nghia Doan}
\date{\today}

\begin{document}

\begin{otherlanguage*}{vietnamese}

\maketitle

\section{Giới thiệu về các kỳ thi toán Hungary}

Toán học luôn đóng vai trò quan trọng trong hệ thống giáo dục Hungary. Quốc gia này có một truyền thống toán học lâu đời và đã đào tạo nên nhiều nhà toán học xuất sắc.
Các kỳ thi toán học tại Hungary không chỉ giúp phát hiện và nuôi dưỡng tài năng mà còn đóng vai trò là cánh cửa để học sinh tham gia các đấu trường quốc tế như IMO, EGMO và MEMO.

Hệ thống các kỳ thi toán học tại Hungary vô cùng phong phú và đa dạng, đến mức học sinh gần như có thể tham gia một cuộc thi mỗi tuần.
Từ các kỳ thi cấp địa phương như Quỹ Vì Trẻ Em Tài Năng Toán Học - Matematikában Tehetséges Gyermekekért (MATEGYE), KöMaL và Arany Dániel đến các kỳ thi cấp quốc gia như OKTV và Kürschák József, mỗi cuộc thi đều có lịch trình chặt chẽ và độ khó khác nhau,
giúp học sinh có cơ hội rèn luyện liên tục. Ngoài ra, các kỳ thi quốc tế như MEMO, NMMV và IMO cũng tạo ra những sân chơi mang tầm vóc lớn hơn,
giúp học sinh Hungary có thể thử sức và phát triển kỹ năng toán học trong suốt cả năm.
Với sự dày đặc của các cuộc thi, học sinh không chỉ trau dồi kiến thức mà còn rèn luyện tư duy sáng tạo, khả năng giải quyết vấn đề và bản lĩnh thi đấu,
để có cơ hội đạt vị thế cao trên đấu trường quốc tế.

Bài viết này sẽ giới thiệu hệ thống các kỳ thi toán học tại Hungary theo ba cấp độ: các kỳ thi cấp thành phố/tỉnh, các kỳ thi toán cấp quốc gia quan trọng
và những kỳ thi liên quan trực tiếp đến quá trình tuyển chọn đội tuyển tham dự IMO, EGMO, MEMO.

\newpage

\section{Các kỳ thi toán học cấp thành phố/tỉnh}

\subsection{Kỳ thi Toán cấp tỉnh MATEGYE}

Kỳ thi Toán cấp tỉnh do Quỹ MATEGYE tổ chức là một kỳ thi hai vòng dành cho học sinh từ lớp 3 đến lớp 12 tại nhiều tỉnh của Hungary.
Mục tiêu của kỳ thi là tạo điều kiện cho học sinh thử sức, phát triển tư duy toán học và so tài cùng bạn bè đồng trang lứa.
Kỳ thi diễn ra tại 14 tỉnh, với mỗi tỉnh có một đơn vị đồng tổ chức để hỗ trợ việc tổ chức thi.

Kỳ thi mở rộng cho tất cả học sinh thuộc các trường học trong tỉnh đăng ký, không giới hạn số lượng thí sinh từ mỗi trường.
Kỳ thi được chia thành các hạng mục theo cấp lớp: học sinh lớp 3 và 4 thi chung một hạng mục,
học sinh lớp 5 đến lớp 8 được chia thành hai hạng mục dành riêng cho trường tiểu học và trung học (gimnázium),
trong khi học sinh lớp 9 đến lớp 12 thi theo hai hạng mục riêng biệt dành cho trung học phổ thông (gimnázium) và trung học kỹ thuật (technikum).
Đối với học sinh theo học chương trình trung học kéo dài năm năm, các em sẽ thi theo cấp lớp tương ứng với nội dung toán học mình đang học.

\textbf{Cấu trúc và quy trình thi}

Kỳ thi bao gồm hai vòng thi với độ khó tăng dần.
\begin{enumerate}[topsep=0pt, partopsep=0pt, itemsep=0pt]
    \ii Vòng đầu tiên diễn ra thường vào đầu tháng 12 tại chính trường học của thí sinh. Đề thi và đáp án do MATEGYE cung cấp,
    và bài thi sẽ được giáo viên của trường chấm theo hướng dẫn chính thức. Kết quả vòng này sẽ quyết định học sinh nào được mời tham gia vòng hai.
    \ii Vòng hai diễn ra thường vào đầu tháng 2 năm sau tại các trường do đơn vị đồng tổ chức của từng tỉnh lựa chọn.
    Ở vòng này, bài làm sẽ được chấm bởi hội đồng giám khảo chuyên môn do MATEGYE chỉ định.
\end{enumerate}

Mỗi vòng thi bao gồm 5 bài toán tự luận, yêu cầu học sinh trình bày lời giải chi tiết và lập luận chặt chẽ. Nội dung đề thi được xây dựng phù hợp với từng cấp lớp.
\begin{itemize}[topsep=0pt, partopsep=0pt, itemsep=0pt]
    \ii Đối với lớp 3 và 4, các bài toán tập trung vào số học cơ bản, bài toán đố, quy luật số học và hình học trực quan.
    \ii Học sinh lớp 5 đến lớp 8 làm các bài toán về số học, đại số sơ cấp, hình học phẳng, tổ hợp và tư duy logic.
    \ii Ở cấp trung học từ lớp 9 đến lớp 12, đề thi có độ khó cao hơn, gồm số học, đại số nâng cao (phương trình, bất đẳng thức), hình học không gian, tổ hợp và xác suất.
\end{itemize}

Thời gian làm bài của thí sinh được quy định theo cấp lớp: học sinh lớp 3 và 4 có 60 phút,
học sinh lớp 5 đến lớp 8 có 90 phút, và học sinh lớp 9 đến lớp 12 có 120 phút để hoàn thành bài thi.
Trong quá trình làm bài, học sinh lớp 3 đến lớp 8 chỉ được sử dụng bút, thước kẻ, compa và thước đo góc,
trong khi học sinh lớp 9 đến lớp 12 được phép dùng bảng công thức và máy tính bỏ túi.

Học sinh đăng ký trực tuyến thường vào đầu tháng 10 tại www.mategye.hu. Lệ phí tham gia là 1500 Ft (100.000 VND)/học sinh,
được thanh toán dựa trên hóa đơn do MATEGYE cung cấp sau khi hoàn tất đăng ký.

\textbf{Chấm điểm và giải thưởng}

Kết quả kỳ thi sẽ được công bố thường vào đầu tháng 3 trong một buổi lễ trao giải do đơn vị tổ chức tại từng tỉnh quyết định.
Học sinh được xếp hạng dựa trên điểm số đạt được trong vòng hai.

Giải thưởng được phân bổ dựa trên số lượng thí sinh tham gia. Tối đa ba thí sinh xuất sắc nhất trong mỗi hạng mục sẽ nhận cúp,
trong khi các thí sinh đạt thành tích cao sẽ nhận giấy chứng nhận và phần thưởng.
Ở cấp lớp 3 và 4, cũng như lớp 5 và 6, tối đa ba thí sinh xuất sắc nhất sẽ được vinh danh.
Đối với học sinh lớp 7 và 8 (gimnázium) và cấp kỹ thuật, số lượng thí sinh đạt giải có thể lên đến sáu người,
trong khi với học sinh lớp 9 đến lớp 12 (gimnázium), số lượng thí sinh đạt giải có thể lên đến mười người.

Những giáo viên có học sinh đạt thành tích cao nhất cũng sẽ được vinh danh.
Đặc biệt, các học sinh xuất sắc trong hạng mục tiểu học và trung học
có thể được lựa chọn tham gia Kỳ thi Toán học Quốc tế dành cho học sinh nói tiếng Hungary (NMMV) nếu kỳ thi này được tổ chức.

\textbf{Ý nghĩa kỳ thi}

Kỳ thi Toán cấp tỉnh không chỉ là một sân chơi toán học cho tất cả các học sinh ở mọi nơi mà còn là cơ hội để học sinh thử thách bản thân, phát triển tư duy và chuẩn bị cho các kỳ thi toán chuyên sâu hơn.
Với hệ thống tổ chức chặt chẽ và quy trình đánh giá minh bạch, kỳ thi này đóng vai trò quan trọng trong việc thúc đẩy giáo dục toán học tại Hungary.

\bigbreak

\noindent\rule{16.5cm}{0.4pt}

\textbf{Một số bài thi tiêu biểu}

\bigbreak

\begin{problem*}[2016-2017, vòng 2, lớp 8, bài 5]
    Một chiếc đồng hồ có kim chỉ giờ và kim phút vuông góc với nhau hai lần trong khoảng thời gian từ 5 giờ đến 6 giờ.
    Hỏi có bao nhiêu phút trôi qua từ lần đầu tiên kim đồng hồ ở vị trí vuông góc đến lần thứ hai?
\end{problem*}

\begin{problem*}[2016-2017, vòng 2, lớp 9, bài 6]
    Hai số nguyên dương có tổng, khi cộng thêm hiệu, tích và thương của chúng, ta được kết quả là 32. Hỏi hai số đó là bao nhiêu?
\end{problem*}

\begin{problem*}[2016-2017, vòng 2, lớp 10, bài 6]
    Một tam giác vuông có hai cạnh góc vuông dài 6 và 8 đơn vị.
    \begin{enumerate}[topsep=0pt, partopsep=0pt, itemsep=0pt]
        \ii Bán kính của đường tròn nội tiếp là bao nhiêu đơn vị?
        \ii Bán kính của đường tròn tiếp xúc với cạnh huyền và hai đường thẳng chứa hai cạnh góc vuông là bao nhiêu đơn vị?
    \end{enumerate}
\end{problem*}

\begin{problem*}[2013-2014, vòng 2, lớp 11, bài 6]
    Giải phương trình: 
    \[
        \sqrt{\left(17+12\sqrt{2}\right)^x} + \sqrt{\left(17-12\sqrt{2}\right)^x} = \frac{10}{3}.
    \]
\end{problem*}

\begin{problem*}[2013-2014, vòng 2, lớp 12, bài 6]
    Tìm giá trị của biểu thức sau: 
    \[
        \sqrt{1+\frac{1}{1^2}+\frac{1}{2^2}} + \sqrt{1+\frac{1}{2^2}+\frac{1}{3^2}} + \cdots + \sqrt{1+\frac{1}{2013^2}+\frac{1}{2014^2}}.
    \]
\end{problem*}

\newpage

\subsection{Cúp Con Dê - Kecske Kupa}

Kecske Kupa là một kỳ thi toán đồng đội dành cho học sinh từ lớp 5 đến lớp 8, do Quỹ MATEGYE tổ chức.
Kỳ thi diễn ra qua ba vòng với thể thức độc đáo, khuyến khích tư duy sáng tạo và phản xạ nhanh.
Các đội thi gồm bốn học sinh cùng trường và cùng khối lớp, phối hợp để giải quyết các bài toán theo một hình thức thi đặc biệt.

\textbf{Hình thức và nội dung kỳ thi}

Mỗi đội sẽ giải các bài toán thuộc bốn chủ đề toán học, gồm Kecskeszámtan (Số học Kecske), Kecskegebra (Đại số Kecske),
Kecskemetria (Hình học Kecske) và Kecskegyetem (Tư duy tổng hợp Kecske). Mỗi chủ đề có tám bài toán, được in trên giấy màu khác nhau để phân biệt.

Các bài toán được thiết kế để kiểm tra khả năng tư duy logic, lập luận chặt chẽ và chiến lược giải quyết vấn đề. Nội dung bài thi bao gồm nhiều dạng toán khác nhau:
\begin{itemize}[topsep=0pt, partopsep=0pt, itemsep=0pt]
    \ii Số học và lý thuyết số: Các phép toán cơ bản, số nguyên tố, tính chất chia hết, dãy số và hệ thập phân.
    \ii Đại số và phương trình: Biểu thức đại số, phương trình, bất đẳng thức, hệ phương trình, dãy số và quy luật số học.
    \ii Hình học và đo lường: Bài toán hình học phẳng và không gian, diện tích, thể tích, định lý Pythagoras, định lý Thales và các bài toán tối ưu hóa.
    \ii Tổ hợp và xác suất: Nguyên lý đếm, hoán vị, tổ hợp, xác suất cơ bản, lập bảng và phân tích chiến thuật tối ưu.
    \ii Tư duy logic và chiến lược: Câu đố toán học, trò chơi chiến lược, phân tích mô hình, suy luận logic phức tạp.
\end{itemize}

Tại mỗi vòng thi, đội sẽ nhận bài đầu tiên của từng chủ đề và có thể tự chọn thứ tự giải quyết. Sau khi hoàn thành một bài, đội sẽ nộp bài cho ban giám khảo để chấm điểm ngay.
Nếu đáp án đúng, đội nhận điểm và tiếp tục với bài tiếp theo cùng chủ đề. Nếu sai, đội bị trừ 1 điểm và có thể thử lại hoặc chuyển sang bài mới.

Một bài toán có thể thử tối đa sáu lần. Nếu sai cả sáu lần, đội sẽ bị trừ 6 điểm và buộc phải chuyển sang bài tiếp theo.
Một số bài toán không có đáp án đúng trong hệ thống, yêu cầu đội phải phát hiện điều này sớm để tránh mất điểm do thử lại quá nhiều lần.

Điểm số của các bài toán tăng dần theo mức độ khó. Các bài toán đơn giản có điểm số từ 5-8 điểm, bài trung bình từ 9-14 điểm,
trong khi các bài toán khó nhất có thể đạt từ 16-20 điểm. Điều này đòi hỏi đội thi phải có chiến lược hợp lý trong việc lựa chọn bài toán để tối đa hóa điểm số.

\textbf{Lịch trình kỳ thi}

Kỳ thi diễn ra qua ba vòng:
\begin{enumerate}[topsep=0pt, partopsep=0pt, itemsep=0pt]
    \ii Vòng 1 diễn ra thường vào đầu tháng 12 tại các điểm thi địa phương.
    \ii Vòng 2 tổ chức thường vào đầu tháng 3 năm sau,
    gồm phần thi chính và phiên đấu giá Kecske Tallérs (đồng bạc con dê) – đơn vị điểm thưởng đặc biệt mà các đội có thể dùng để đổi lấy phần thưởng.
    \ii Vòng chung kết được tổ chức thường vào đầu tháng 5 tại Kecskemét, nơi các đội xuất sắc nhất từ vòng 2 tranh tài để giành chiến thắng chung cuộc.
\end{enumerate}

Tất cả đội đăng ký đều có thể tham gia hai vòng đầu. Kết quả từ hai vòng này sẽ quyết định đội nào vào chung kết.
Danh sách đội vào chung kết được xác định dựa trên tổng điểm số và thứ hạng trong từng khu vực.

Học sinh đăng ký trực tuyến thường vào cuối tháng 10 tại www.mategye.hu. Lệ phí tham gia là 2000 Ft (140.000 VND)/học sinh, tức 8000 Ft (550.000 VND)/đội.

Mỗi trường có thể đăng ký nhiều đội. Trong suốt giải đấu, mỗi đội có thể thay đổi thành viên một lần,
nhưng người thay thế không được từng tham gia thi với đội khác trong cùng năm học.

\textbf{Đánh giá và giải thưởng}

Điểm số của các đội được cập nhật trực tiếp trên màn hình trong suốt kỳ thi, giúp đội thi theo dõi vị trí của mình và điều chỉnh chiến thuật kịp thời.

Kết quả được xét riêng theo từng cấp lớp. Hai đội đứng đầu mỗi cấp lớp tại từng địa điểm thi sẽ tự động vào chung kết nếu có đủ số lượng đội tham gia.
Ngoài ra, những đội có tổng điểm cao nhất toàn quốc cũng có thể được chọn vào vòng chung kết.

Các đội tham gia vòng 1 và vòng 2 sẽ nhận Kecske Tallérs, dùng để đấu giá phần thưởng sau vòng 2.
Những đội xuất sắc nhất trong vòng này sẽ nhận chứng nhận và Kecske Tallérs thưởng.

Ở vòng chung kết, đội đứng đầu mỗi cấp lớp nhận quà lưu niệm và chứng nhận, trong khi các đội còn lại cũng được trao chứng nhận thành tích.

\textbf{Ý nghĩa kỳ thi}

Kỳ thi không chỉ kiểm tra khả năng toán học mà còn khuyến khích tinh thần đồng đội, chiến lược và tư duy phản xạ.
Hình thức thi linh hoạt giúp học sinh phát triển kỹ năng hợp tác, ra quyết định nhanh và làm quen với cách giải toán sáng tạo trong áp lực thời gian.

\begin{remark*}
    Các đề thi không công khai.
\end{remark*}

\newpage

\subsection{Kỳ thi Toán - Khoa học Đồng đội Dürer}

Kỳ thi Dürer là một trong những kỳ thi học thuật đồng đội uy tín nhất tại Hungary, tập trung vào ba môn Toán học, Vật lý và Hóa học.
Được tổ chức lần đầu vào năm 2007, kỳ thi khuyến khích tư duy sáng tạo, khả năng giải quyết vấn đề và làm việc nhóm hiệu quả.
Đây không phải là một kỳ thi cá nhân mà là sân chơi dành cho các đội thi từ hai đến ba học sinh,
nơi các thành viên phải phối hợp chặt chẽ để tìm ra lời giải chính xác và tối ưu nhất.
Gần đây kỳ thi được tổ chức với sự hỗ trợ của Chương trình Tài năng Quốc gia (Nemzeti Tehetség Program).

\textbf{Đối tượng tham gia và phân loại}

Kỳ thi dành cho các đội học sinh từ lớp 5 đến lớp 12, với hai nhóm chính:
\begin{itemize}[topsep=0pt, partopsep=0pt, itemsep=0pt]
    \ii KisDürer: Dành cho học sinh tiểu học (lớp 5-8).
    \ii Dürer: Dành cho học sinh trung học (lớp 9-12).
\end{itemize}

Mỗi đội có thể lựa chọn thi theo một trong ba môn: Toán học, Vật lý hoặc Hóa học. Các đội thi thuộc nhóm Dürer (trung học) sẽ đối mặt với những bài toán có độ khó cao hơn,
đòi hỏi khả năng tư duy trừu tượng và vận dụng nhiều kỹ thuật nâng cao hơn so với nhóm KisDürer (tiểu học).

\textbf{Cấu trúc và nội dung kỳ thi}

Kỳ thi diễn ra qua hai vòng: vòng loại và vòng chung kết.
\begin{enumerate}[topsep=0pt, partopsep=0pt, itemsep=0pt]
    \ii Vòng loại: Các đội sẽ giải một bộ đề gồm các bài toán tự luận thuộc lĩnh vực Toán, Vật lý hoặc Hóa học.
    Đề thi yêu cầu không chỉ nắm vững kiến thức mà còn phải có tư duy logic, lập luận chặt chẽ và kỹ năng giải quyết vấn đề theo nhóm.
    \ii Vòng chung kết: Những đội có kết quả tốt nhất ở vòng loại sẽ được mời tham gia vòng chung kết.
    Tại đây, các đội sẽ đối mặt với những thử thách phức tạp hơn, bao gồm cả bài thi lý thuyết và bài tập thực hành.
    Đề thi không chỉ kiểm tra khả năng tư duy toán học - khoa học mà còn yêu cầu sự phối hợp nhịp nhàng giữa các thành viên để đưa ra chiến lược giải quyết bài toán hiệu quả nhất.
\end{enumerate}
	
\textbf{Các loại bài toán}

Các bài toán trong kỳ thi Dürer không chỉ kiểm tra kiến thức mà còn yêu cầu thí sinh phối hợp tư duy sáng tạo và vận dụng kiến thức vào thực tiễn.
Một số dạng bài toán thường gặp bao gồm:
\begin{itemize}[topsep=0pt, partopsep=0pt, itemsep=0pt]
    \ii Toán học: Chứng minh bất đẳng thức, bài toán tổ hợp, số học. Hình học phẳng và không gian, yêu cầu tính diện tích, thể tích, chứng minh quan hệ hình học.
    Phương trình hàm, đa thức và lý thuyết đồ thị.
    \ii Vật lý: Cơ học, điện từ học, nhiệt động lực học. Ứng dụng các định luật vật lý để giải quyết bài toán thực tế.
    Một số bài toán yêu cầu thiết kế mô hình hoặc tính toán dựa trên số liệu thực nghiệm.
    \ii Hóa học: Phản ứng hóa học, cân bằng phương trình, hóa lý. Phân tích hợp chất, tính toán hóa học dựa trên số liệu thực nghiệm.
    Một số bài toán yêu cầu hiểu biết về hóa học ứng dụng.
\end{itemize}

Ví dụ, một bài toán hình học có thể yêu cầu các đội thiết kế một logo 2D sử dụng các hình tam giác đều và hình vuông, sao cho chu vi của logo đạt 13 cm.
Một bài toán khác có thể yêu cầu tính diện tích của một viên gạch có các cạnh là cung tròn, với chiều cao của viên gạch là 12 cm.
Những bài toán này không chỉ kiểm tra kiến thức mà còn khuyến khích học sinh phối hợp suy luận, chia sẻ chiến lược và hỗ trợ lẫn nhau trong quá trình tìm lời giải.

\textbf{Mục tiêu của kỳ thi}

Với cấu trúc thi đặc biệt và các bài toán có độ thử thách cao, kỳ thi Dürer đã trở thành một sự kiện quan trọng cho những học sinh yêu thích toán học và khoa học.
Kỳ thi Dürer không chỉ là một bài kiểm tra kiến thức mà còn tạo ra một môi trường học tập thú vị, thúc đẩy tư duy sáng tạo và tinh thần đồng đội trong học sinh.
Thông qua các thử thách đa dạng, kỳ thi giúp học sinh phát triển kỹ năng tư duy phản biện, làm việc nhóm và khả năng giải quyết vấn đề theo hướng hợp tác hiệu quả.

Ngoài ra, kỳ thi còn giúp các em chuẩn bị tốt hơn cho các kỳ thi học thuật quốc gia và quốc tế, cũng như phát triển niềm đam mê nghiên cứu khoa học.
Những học sinh có thành tích xuất sắc trong kỳ thi Dürer thường tiếp tục tham gia các kỳ thi lớn như Olympic Toán Quốc tế (IMO),
Olympic Vật lý Quốc tế (IPhO) hoặc Olympic Hóa học Quốc tế (IChO).

\bigbreak

\noindent\rule{16.5cm}{0.4pt}

\textbf{Một số bài thi tiêu biểu}

\bigbreak

\begin{problem*}[2024-2025, vòng chung kết, mức C, bài 6]
    Trên một bàn cờ $4 \times 4$, có một quân mã đứng trên một ô (không có quân cờ nào khác). Hai người chơi lần lượt thay nhau di chuyển quân mã.
    Không được di chuyển đến ô mà quân mã đã từng đi qua trước đó, bao gồm cả ô xuất phát. Người thua cuộc là người không thể thực hiện nước đi hợp lệ.
    Bạn có chiến lược nào để luôn thắng nếu được chọn đi trước hay đi sau?
\end{problem*}

\begin{problem*}[2024-2025, vòng chung kết, mức D, bài 4]
    Một vị vua tàn nhẫn đã xây dựng một nhà tù với 36 phòng giam được sắp xếp theo dạng lưới $6 \times 6$, trong đó mỗi cặp phòng giam liền kề đều bị ngăn cách bởi một bức tường.
    Một số phòng đã được chỉ định cho tù nhân, nhưng mỗi phòng chỉ thuộc về tối đa một tù nhân.

    Về sau, nhà vua cảm thấy mình quá tàn nhẫn, nên quyết định phá bỏ một số bức tường sao cho bất kỳ phòng nào cũng có thể đi đến bất kỳ phòng nào khác.
    Tuy nhiên, ông không muốn tù nhân quá thoải mái, nên muốn giữ lại ít nhất một bức tường
    giữa bất kỳ hai tù nhân nào ban đầu nằm trên cùng một hàng hoặc cột để họ không thể nhìn thấy nhau từ vị trí của mình.

    Tối đa có bao nhiêu tù nhân có thể bị giam giữ nếu nhà vua phá tường theo cách thỏa mãn các điều kiện trên?
\end{problem*}

\begin{problem*}[2016-2017, vòng chung kết, mức E+, bài 5]
    Tồn tại hay không vô số các đường thằng sao cho: không có hai đường nào song song với nhau, không có ba đường nào đồng quy,
    và các giao điểm của các đường thẳng này đều cách mỗi đường thằng một khoảng cách có độ dài nguyên dương?
\end{problem*}

\newpage

\section{Các kỳ thi toán học cấp quốc gia}

\subsection{Kỳ thi Toán học Zrínyi Ilona}

Kỳ thi Toán học Zrínyi Ilona là một trong những cuộc thi toán quan trọng nhất tại Hungary, do Quỹ MATEGYE tổ chức hàng năm.
Mục tiêu chính của kỳ thi là phổ biến toán học, phát triển tư duy logic và kỹ năng giải quyết vấn đề cho học sinh từ lớp 2 đến lớp 12.
Cuộc thi không chỉ tạo điều kiện cho học sinh thử sức mà còn giúp chuẩn bị cho các kỳ thi toán học chuyên sâu hơn.

\textbf{Cấu trúc và thể lệ}

Kỳ thi gồm ba cấp độ: cấp tiểu học (lớp 2-8) thi chung một hạng mục, còn cấp trung học phổ thông (lớp 9-12) chia thành hai hạng mục là gimnázium (trung học phổ thông)
và technikum (trung học kỹ thuật).

Thời gian làm bài được quy định theo cấp lớp. Học sinh lớp 2-4 có 60 phút để hoàn thành 25 câu hỏi, lớp 5-6 có 75 phút để làm 25 câu, và lớp 7-12 có 90 phút để giải 30 câu hỏi.
Các bài thi đều ở dạng trắc nghiệm với 5 phương án trả lời (A, B, C, D, E), trong đó chỉ có một đáp án đúng.

Nội dung câu hỏi bao gồm nhiều lĩnh vực khác nhau trong toán học. Đối với học sinh nhỏ tuổi, bài thi tập trung vào số học, hình học cơ bản và tư duy logic.
Ở cấp độ cao hơn, đề thi có sự kết hợp giữa số học, đại số, hình học, tổ hợp và xác suất, yêu cầu học sinh vận dụng kiến thức linh hoạt và suy luận chặt chẽ.
Một số dạng bài phổ biến trong kỳ thi gồm:
\begin{itemize}[topsep=0pt, partopsep=0pt, itemsep=0pt]
    \ii Số học và lý thuyết số, bao gồm phép toán, ước số, bội số, số nguyên tố và các bài toán chia hết.
    \ii Đại số, với biểu thức, phương trình, bất đẳng thức và quy luật số học.
    \ii Hình học, bao gồm tính diện tích, chu vi, thể tích, quan hệ góc và các bài toán hình học phẳng.
    \ii Tổ hợp và xác suất, liên quan đến đếm, hoán vị, tổ hợp và các bài toán tư duy logic.
    \ii Ứng dụng toán học, với các bài toán về tốc độ, thời gian, tài chính và bài toán thực tế.
\end{itemize}

Hệ thống điểm số được tính theo công thức $4H - R + F$ hoặc $5H + U$, trong đó $H$ (helyes) là số câu trả lời đúng, $R$ (rossz) là số câu sai, $F$ (feladatok) là số câu hỏi,
và $U$ (üres) là số câu không trả lời. Nếu một câu hỏi không được trả lời, thí sinh không bị trừ điểm, nhưng nếu tô hai hoặc nhiều đáp án hoặc có dấu hiệu chỉnh sửa sai quy định,
câu trả lời đó sẽ bị tính là sai.

Thí sinh không được sử dụng bất kỳ công cụ hỗ trợ nào ngoài bút viết và giấy nháp do ban tổ chức cung cấp. Chỉ phiếu trả lời trắc nghiệm được thu lại để chấm điểm.

\textbf{Quy trình và vòng thi}

Kỳ thi diễn ra qua ba vòng với độ khó tăng dần:
\begin{enumerate}[topsep=0pt, partopsep=0pt, itemsep=0pt]
    \ii Vòng 1 tổ chức tại trường học của thí sinh thường vào cuối tháng 11. Các bài thi của học sinh sẽ được chấm tự động, và kết quả công bố vào ngày 09/12/2024.
    \ii Vòng 2 diễn ra thường vào cuối tháng 2 năm sau, với bài thi gồm số lượng câu hỏi tương tự vòng 1 nhưng độ khó cao hơn.
    Chỉ những học sinh đạt tối thiểu 50\% số điểm tối đa hoặc nằm trong nhóm có thành tích cao nhất tại trường mới đủ điều kiện vào vòng 2.
    Danh sách thí sinh đủ điều kiện vào vòng này được công bố vào cuối tháng 12.
    \ii Vòng chung kết toàn quốc được tổ chức thường vào cuối tháng 4, nơi những thí sinh xuất sắc nhất của từng khu vực tranh tài.
    Số suất tham dự vòng chung kết phụ thuộc vào tổng số thí sinh tham gia từ mỗi khu vực.
    Nếu một khu vực có từ 50 đến 149 thí sinh, một thí sinh được chọn vào chung kết; với 150-249 thí sinh sẽ có hai suất, 250-349 thí sinh có ba suất,
    và nếu có từ 350 thí sinh trở lên sẽ có bốn suất.
    Vòng chung kết có độ khó cao hơn hẳn hai vòng trước, đòi hỏi tư duy logic và kỹ năng phân tích bài toán phức tạp.
    Điểm đặc biệt của vòng này là có sự kết hợp giữa các câu hỏi trắc nghiệm và phần thi tự luận nhằm đánh giá tư duy chiến lược của thí sinh.
\end{enumerate}

\textbf{Đánh giá và trao giải}

Kết quả được chấm riêng theo từng cấp lớp và hạng mục thi. Nếu có hai thí sinh có cùng số điểm, thứ tự ưu tiên sẽ dựa trên tiêu chí số câu trả lời sai ít hơn.
Nếu vẫn bằng nhau, hệ thống sẽ xét đến mức độ khó của các bài mà thí sinh giải đúng; bài nào được nhiều thí sinh làm đúng hơn sẽ có trọng số thấp hơn,
còn bài hiếm người giải đúng hơn sẽ có trọng số cao hơn. Nếu vẫn bằng nhau, thí sinh sẽ được xếp đồng hạng.

Những thí sinh đạt thành tích cao nhất trong vòng 2 sẽ nhận chứng nhận và phần thưởng hiện vật.
Ở vòng chung kết, mười thí sinh xuất sắc nhất mỗi cấp lớp sẽ nhận huy chương và giải thưởng đặc biệt, trong khi các thí sinh khác sẽ nhận chứng nhận thành tích.
Lễ trao giải và kết quả cuối cùng sẽ được công bố ngay sau buổi thi.

\textbf{Ý nghĩa và tầm quan trọng}

Kỳ thi Toán học Zrínyi Ilona không chỉ là một cuộc thi mà còn là cơ hội để học sinh thử thách bản thân,
rèn luyện tư duy logic và chuẩn bị cho các kỳ thi toán học tầm cỡ.
Hàng năm, cuộc thi thu hút hàng chục nghìn học sinh tham gia và là một trong những sự kiện toán học quan trọng nhất tại Hungary.
Không chỉ vinh danh những thí sinh xuất sắc, cuộc thi còn góp phần xây dựng nền tảng toán học vững chắc cho thế hệ trẻ,
khuyến khích học sinh Hungary phát triển niềm đam mê với môn toán học.

\bigbreak

\noindent\rule{16.5cm}{0.4pt}

\textbf{Một số bài thi tiêu biểu}

\bigbreak

\begin{problem*}[Ví dụ mẫu, lớp 8, bài 30]
    Anna đã tạo một hình 67-giác lồi từ giấy. Bea cắt hình này thành hai phần bằng một đường thẳng, sau đó tiếp tục cắt một trong các phần nhận được bằng một đường thẳng khác.
    Quá trình này được lặp lại cho đến khi thu được 8 $n-$giác. Giá trị của $n$ là bao nhiêu?
    \[
        (A) \quad 11 \qquad
        (B) \quad 12 \qquad
        (C) \quad 13 \qquad
        (D) \quad 14 \qquad
        (E) \quad 15
    \]
\end{problem*}

\begin{problem*}[Ví dụ mẫu, lớp 10, bài 29]
    András và Balázs khởi hành cùng lúc từ thành phố A đến thành phố B bằng cách đi bộ. András đi nhanh hơn Balázs, mỗi km mất ít hơn Balázs 5 phút.
    Sau khi đi được một phần năm quãng đường, András quay lại thành phố A, nghỉ 10 phút, rồi tiếp tục hành trình đến thành phố B, nơi anh đến cùng lúc với Balázs.
    Khoảng cách giữa hai thành phố A và B là bao nhiêu km, nếu Balázs mất 2,5 giờ để đi hết quãng đường đó?
    \[
        (A) \quad 8 \qquad
        (B) \quad 10 \qquad
        (C) \quad 15 \qquad
        (D) \quad 16 \qquad
        (E) \quad 20
    \]
\end{problem*}

\begin{problem*}[Ví dụ mẫu, lớp 12, bài 28]
    Có bao nhiêu cách để đọc được từ GORDIUSZ từ sơ đồ nếu ta chỉ có thể di chuyển sang phải hoặc xuống dưới, và không được đi cùng một hướng quá hai lần liên tiếp?
    \begin{center}
        \begin{tabular}{|c|ccccccc}
        \hline
        G & \multicolumn{1}{c|}{O} & \multicolumn{1}{c|}{R} & \multicolumn{1}{c|}{D} & \multicolumn{1}{c|}{I} & \multicolumn{1}{c|}{U} & \multicolumn{1}{c|}{S} & \multicolumn{1}{c|}{Z} \\ \hline
        O & \multicolumn{1}{c|}{R} & \multicolumn{1}{c|}{D} & \multicolumn{1}{c|}{I} & \multicolumn{1}{c|}{U} & \multicolumn{1}{c|}{S} & \multicolumn{1}{c|}{Z} &                        \\ \cline{1-7}
        R & \multicolumn{1}{c|}{D} & \multicolumn{1}{c|}{I} & \multicolumn{1}{c|}{U} & \multicolumn{1}{c|}{S} & \multicolumn{1}{c|}{Z} &                        &                        \\ \cline{1-6}
        D & \multicolumn{1}{c|}{I} & \multicolumn{1}{c|}{U} & \multicolumn{1}{c|}{S} & \multicolumn{1}{c|}{Z} &                        &                        &                        \\ \cline{1-5}
        I & \multicolumn{1}{c|}{U} & \multicolumn{1}{c|}{S} & \multicolumn{1}{c|}{Z} &                        &                        &                        &                        \\ \cline{1-4}
        U & \multicolumn{1}{c|}{S} & \multicolumn{1}{c|}{Z} &                        &                        &                        &                        &                        \\ \cline{1-3}
        S & \multicolumn{1}{c|}{Z} &                        &                        &                        &                        &                        &                        \\ \cline{1-2}
        Z &                        &                        &                        &                        &                        &                        &                        \\ \cline{1-1}
        \end{tabular}
    \end{center}    
    \[
        (A) \quad 8 \qquad
        (B) \quad 32 \qquad
        (C) \quad 42 \qquad
        (D) \quad 100 \qquad
        (E) \quad 128
    \]
\end{problem*}

\newpage

\subsection{Kỳ thi Toán học Quốc gia Varga Tamás}

Kỳ thi Toán học Varga Tamás là một trong những cuộc thi toán quan trọng nhất dành cho học sinh trung học cơ sở tại Hungary,
được tổ chức hàng năm nhằm phát hiện và nuôi dưỡng tài năng toán học. Cuộc thi mang tên Varga Tamás (1919-1987), một nhà sư phạm xuất sắc có nhiều đóng góp cho giáo dục toán học.
Được khởi xướng vào năm 1987 bởi các giáo viên từ ELTE Trefort Utcai Gimnázium và Fazekas Mihály Gimnázium,
cuộc thi ban đầu mang tính thử nghiệm và chính thức trở thành một phần quan trọng của hệ thống giáo dục Hungary từ năm học 1990-1991.

Mô hình ba vòng thi được định hình từ những năm đầu tiên, với sự điều chỉnh để phù hợp với hệ thống giáo dục thay đổi.
Từ năm 1993, khi các trường trung học 6 năm và 8 năm xuất hiện, cuộc thi bắt đầu phân chia thành hai hạng mục:
một dành cho học sinh có số giờ học toán tiêu chuẩn và một dành cho học sinh có chương trình toán nâng cao.
Năm 2001, cuộc thi được chuyển giao cho Fejér Megyei Pedagógiai Szakszolgálati és Szakmai Intézet để tổ chức và quản lý.

\textbf{Cấu trúc và thể lệ}

Cuộc thi dành riêng cho học sinh lớp 7 và 8, với hai hạng mục: nhóm tiêu chuẩn (học tối đa 4 giờ toán mỗi tuần) và nhóm nâng cao (học hơn 4 giờ toán mỗi tuần).
Cả hai nhóm đều thi chung ở vòng đầu, nhưng từ vòng hai sẽ được xét tách biệt.

Kỳ thi gồm ba vòng với độ khó tăng dần:
\begin{enumerate}[topsep=0pt, partopsep=0pt, itemsep=0pt]
    \ii Vòng 1 (trường học): Tổ chức vào tháng 10 hoặc 11 tại các trường đăng ký. Thí sinh làm bài thi trong 90 phút, gồm 5-6 bài toán tự luận, với mức độ từ cơ bản đến nâng cao.
    Bài thi không chỉ kiểm tra kiến thức chương trình phổ thông mà còn đánh giá tư duy logic, kỹ năng lập luận và khả năng sáng tạo của học sinh.
    \ii Vòng 2 (khu vực/megyei): Tổ chức vào tháng 1, dành cho những thí sinh đạt tối thiểu 50\% số điểm tối đa ở vòng 1.
    Ở vòng này, học sinh được phân vào hai hạng mục dựa trên số giờ học toán hàng tuần. Bài thi kéo dài 120 phút, với 6-8 bài toán tự luận, đòi hỏi khả năng suy luận cao hơn,
    bao gồm các chủ đề như số học, đại số, hình học và tổ hợp.
    \ii Vòng chung kết (toàn quốc): Diễn ra vào tháng 3 hoặc 4, dành cho khoảng 50 thí sinh xuất sắc nhất trong hạng mục nâng cao và 100 thí sinh từ hạng mục tiêu chuẩn.
    Bài thi vòng này kéo dài 150 phút, gồm 8-10 bài toán, với các câu hỏi thử thách hơn, yêu cầu thí sinh đưa ra lời giải chi tiết và chặt chẽ.
    Các bài toán thường có sự kết hợp của nhiều lĩnh vực toán học, từ số học, hình học, tổ hợp đến bất đẳng thức và phương trình hàm.
\end{enumerate}

Điểm số của từng bài toán thường dao động từ 4 đến 10 điểm, tùy vào độ khó và mức độ chi tiết của lời giải. Nếu lời giải có sai sót nhỏ nhưng vẫn thể hiện tư duy đúng, thí sinh có thể nhận điểm một phần.

\textbf{Đánh giá và trao giải}

Bài thi vòng 1 được chấm bởi giáo viên của trường, vòng 2 do hội đồng khu vực chấm, và vòng chung kết do hội đồng giám khảo quốc gia gồm các chuyên gia từ Bolyai János Matematikai Társulat đảm nhiệm.

Điểm số được tính dựa trên số lượng bài giải đúng và mức độ hoàn chỉnh của lời giải. Nếu có hai thí sinh đạt cùng số điểm, thứ hạng sẽ được quyết định dựa trên:
\begin{enumerate}[topsep=0pt, partopsep=0pt, itemsep=0pt]
    \ii Số bài giải đúng hoàn chỉnh nhiều hơn.
    \ii Điểm số đạt được trong các bài khó nhất.
    \ii Tổng điểm ưu tiên, dựa trên số lượng thí sinh giải được mỗi bài (bài càng khó, điểm ưu tiên càng cao).
    Nếu vẫn bằng nhau, thí sinh sẽ đồng hạng.
\end{enumerate}

Những thí sinh xuất sắc tại vòng 2 sẽ nhận chứng nhận và phần thưởng hiện vật. Ở vòng chung kết, ba thí sinh hàng đầu mỗi hạng mục nhận huy chương và giải thưởng đặc biệt.
Các thí sinh còn lại được trao chứng nhận thành tích.

\textbf{Ý nghĩa và ảnh hưởng}

Từ khi ra đời, kỳ thi đã trở thành bệ phóng cho nhiều tài năng toán học Hungary. Mỗi năm, hàng nghìn học sinh tham gia vòng đầu tiên,
và nhiều em sau này đã gặt hái thành công tại các cuộc thi toán quốc tế, bao gồm Olympic Toán Quốc tế (IMO).
Cuộc thi không chỉ giúp phát hiện tài năng mà còn góp phần nâng cao chất lượng dạy và học toán trên toàn quốc.

Với truyền thống lâu đời và uy tín, kỳ thi Toán học Varga Tamás tiếp tục đóng vai trò quan trọng trong hệ thống giáo dục Hungary,
khuyến khích học sinh rèn luyện tư duy logic, khám phá vẻ đẹp của toán học và chuẩn bị cho những thử thách lớn hơn trong tương lai.

\bigbreak

\noindent\rule{16.5cm}{0.4pt}

\textbf{Một số bài thi tiêu biểu}

\bigbreak

\begin{problem*}[2024-2025, vòng 1, lớp 7, nhóm 2, bài 3]
    Bốn tam giác đều bằng nhau có các đỉnh được gán các số 1, 2, 3 sao cho mỗi số xuất hiện đúng một lần trong mỗi tam giác.
    Sau đó, bốn tam giác này được ghép lại để tạo thành một tam giác lớn theo sơ đồ.
    Các số tại các đỉnh trùng nhau được cộng lại, và kết quả được ghi tại trung điểm của ba cạnh của tam giác lớn. Đáng chú ý là ba trung điểm này nhận cùng một tổng.
    \begin{center}
        \includegraphics[width=3cm]{../Learning-Problem-Solving-2nd-Edition/svg/pdf/varga-tamas-2425-2f-7o-2k-3.pdf}
    \end{center}
    Tổng đó có thể là bao nhiêu?
\end{problem*}

\begin{problem*}[2024-2025, vòng 1, lớp 8, nhóm 1, bài 3]
    Cho hình chữ nhật \( ABCD \) có cạnh \( AB \) dài gấp ba lần cạnh \( BC \). Gọi \( E \) là điểm chia ba cạnh \( AB \), gần \( A \) hơn, và \( F \) là trung điểm của cạnh \( AD \).
    Biết diện tích tam giác \( \triangle EFC \) là \( 16 \) cm\(^2\), hãy xác định độ dài của cạnh \( AB \).
\end{problem*}

\begin{problem*}[2024-2025, vòng 2, lớp 8, nhóm 2, bài 5]
    Xét một đường tròn và chọn \( 2n \) điểm trên đó (\( n \geq 3 \)) sao cho độ dài các cung giữa hai điểm kề nhau chỉ có ba giá trị khác nhau
    và không có hai cung liên tiếp nào có cùng độ dài. Các điểm này được tô màu luân phiên đỏ và xanh, sao cho có \( n \) điểm đỏ và \( n \) điểm xanh.  
    
    Chứng minh rằng đa giác \( n \) cạnh tạo bởi các điểm xanh và đa giác \( n \) cạnh tạo bởi các điểm đỏ có diện tích bằng nhau và chu vi bằng nhau!
\end{problem*}

\newpage

\subsection{Kỳ thi Toán học Arany Dániel}

Kỳ thi mang tên Arany Dániel (1863-1945), một giáo viên toán học người Hungary, người đã sáng lập tạp chí Középiskolai Matematikai Lapok (KöMaL) - xem phần sau - vào năm 1893.
Cuộc thi khởi nguồn từ Kỳ thi Toán Trung học Toàn quốc, tổ chức lần đầu vào năm 1947 và chính thức mang tên Arany Dániel từ năm 1951.
Từ năm 1977, kỳ thi này kế thừa Kỳ thi Toán học Kalmár László, dành cho học sinh lớp 9 và 10 tại Hungary.

\textbf{Hệ thống thi và thể lệ}

Kỳ thi Arany Dániel được chia thành hai cấp độ: KEZDŐK (Người mới bắt đầu) và HALADÓK (Người nâng cao).
Cấp độ KEZDŐK dành cho học sinh lớp 9 và 10, được chia thành ba nhóm theo số giờ học toán bắt buộc mỗi tuần.
Trong khi đó, cấp độ HALADÓK chỉ dành cho học sinh lớp 10 và cũng được phân thành ba nhóm tùy thuộc vào chương trình học toán chuyên sâu hay không.

\textbf{Cấu trúc và quy trình thi}

Kỳ thi diễn ra qua ba vòng thi, với độ khó tăng dần.
\begin{enumerate}[topsep=0pt, partopsep=0pt, itemsep=0pt]
    \ii Vòng đầu tiên, còn gọi là vòng trường (iskolai forduló), diễn ra ngay tại trường của thí sinh. Đề thi gồm 4-5 bài toán tự luận, tùy theo cấp độ của thí sinh.
    Mỗi bài có điểm số khác nhau, thường dao động từ 5 đến 10 điểm, tùy vào độ khó. Các bài toán chủ yếu thuộc các lĩnh vực số học, đại số, hình học và tổ hợp,
    với yêu cầu giải thích chi tiết lời giải. Giáo viên toán tại trường sẽ chấm bài dựa trên hướng dẫn chính thức, và những thí sinh đạt đủ số điểm yêu cầu sẽ được ghi danh vào vòng tiếp theo.
    \ii Vòng hai, còn gọi là vòng khu vực/quốc gia (második forduló), thử thách thí sinh với 5 bài toán có độ khó cao hơn. Đề thi có cấu trúc tương tự vòng đầu,
    nhưng các bài toán yêu cầu phương pháp giải sáng tạo và lập luận chặt chẽ hơn. Điểm tối đa cho mỗi bài có thể lên đến 15 điểm.
    Sau khi hoàn thành bài thi, tất cả bài làm sẽ được gửi về Bolyai János Matematikai Társulat,
    nơi hội đồng giám khảo cấp quốc gia chấm điểm và lựa chọn những thí sinh xuất sắc nhất vào vòng chung kết.
    \ii Vòng chung kết (Döntő) diễn ra tại một địa điểm tập trung, nơi các thí sinh có thành tích cao nhất trong vòng hai tranh tài.
    Ở vòng này, thí sinh phải giải 3 bài toán có độ phức tạp cao, mỗi bài thường có thang điểm từ 10 đến 20 điểm, tùy vào mức độ thử thách.
    Các bài toán không chỉ yêu cầu tư duy toán học sâu sắc mà còn đòi hỏi khả năng trình bày lập luận chặt chẽ. Hội đồng giám khảo cấp quốc gia sẽ chấm bài và công bố kết quả cuối cùng.
\end{enumerate}

\textbf{Quy định và công cụ được phép sử dụng}

Thời gian làm bài ở mỗi vòng thi là 240 phút. Trong suốt kỳ thi, thí sinh không được sử dụng máy tính, điện thoại hay bất kỳ thiết bị điện tử nào.
Tuy nhiên, các em được phép mang theo sách giáo khoa, tài liệu toán học in sẵn, bảng công thức và sổ tay toán học để tham khảo.

\textbf{Ý nghĩa của kỳ thi}

Kỳ thi Arany Dániel không chỉ là một cuộc thi toán thông thường mà còn có ý nghĩa giáo dục sâu sắc.

Trước hết, cuộc thi giúp phát hiện và bồi dưỡng tài năng toán học trẻ. Đây là cơ hội để học sinh trung học thử sức với những bài toán mang tính thách thức cao,
phát triển tư duy logic và sáng tạo. Những thí sinh xuất sắc thường tiếp tục tham gia các kỳ thi lớn hơn như OKTV (Országos Középiskolai Tanulmányi Verseny – Kỳ thi Học Sinh Giỏi Toán Quốc gia Hungary).

Bên cạnh đó, kỳ thi góp phần thúc đẩy tinh thần cạnh tranh lành mạnh, giúp học sinh làm quen với áp lực thi cử, rèn luyện kỹ năng quản lý thời gian và tư duy chiến lược.
Nhiều thí sinh đạt thành tích cao trong kỳ thi này sau đó tham gia các kỳ thi quốc tế.

Về mặt định hướng nghề nghiệp, kỳ thi đóng vai trò là một bước đệm quan trọng cho những học sinh có nguyện vọng theo đuổi các ngành khoa học – công nghệ.
Nhiều thí sinh từng đạt giải trong kỳ thi này sau đó đã trở thành nhà toán học, kỹ sư, nhà khoa học và giảng viên đại học.

Ngoài ra, kỳ thi còn tạo ra một sân chơi trí tuệ, khuyến khích học sinh yêu thích toán học.
Ngay cả những học sinh không theo đuổi toán chuyên sâu vẫn có thể tận hưởng niềm vui khi khám phá kiến thức và rèn luyện tư duy logic.
Không chỉ mang tính thi đấu, kỳ thi còn giúp học sinh trải nghiệm “vẻ đẹp” của toán học thông qua những bài toán thú vị và các phương pháp giải sáng tạo.

\bigbreak

\noindent\rule{16.5cm}{0.4pt}

\textbf{Một số bài thi tiêu biểu}

\bigbreak

\begin{problem*}[2023-2024, nhập môn I-II, nhóm 1, vòng 2, bài 1]
    Ádám tính tổng:  
    \[
        10^{2023} + 10^{2018} + 10^{2013} + \dots + 10^{18} + 10^{13} + 10^8 + 10^3
    \]
    và viết ra kết quả. Hỏi Ádám đã viết bao nhiêu chữ số \( 0 \)?
\end{problem*}

\begin{problem*}[2023-2024, nâng cao II, nhóm 1, vòng 1, bài 2]
    Giải hệ phương trình nghiệm nguyên:
    \[
        \left\{
            \begin{array}{rcl}
                x^2 - y^2 - z^2 &=& 1\\
                -x+y+z &=& -3
            \end{array}
        \right.
    \]
\end{problem*}

\begin{problem*}[2023-2024, nâng cao II, nhóm 2, vòng chung kết, bài 3]
    Có thể chọn bốn số thực sao cho thỏa mãn cả bốn điều kiện sau đây không?  
    \begin{enumerate}[topsep=0pt, partopsep=0pt, itemsep=0pt]
        \ii Tồn tại ba số trong đó có tổng và tích đều là số hữu tỷ.  
        \ii Tồn tại ba số trong đó có tổng là số hữu tỷ nhưng tích là số vô tỷ.  
        \ii Tồn tại ba số trong đó có tổng là số vô tỷ nhưng tích là số hữu tỷ.
        \ii Tồn tại ba số trong đó có tổng và tích đều là số vô tỷ.
    \end{enumerate}
\end{problem*}

\newpage

\subsection{Kỳ thi Toán học Quốc tế dành cho học sinh nói tiếng Hungary (NMMV)}

Kỳ thi Toán học Quốc tế dành cho học sinh nói tiếng Hungary (Nemzetközi Magyar Matematikaverseny - NMMV) là một sự kiện học thuật thường niên dành cho học sinh các quốc gia nơi có giảng dạy toán học bằng tiếng Hungary. Kỳ thi không chỉ là một sân chơi trí tuệ mà còn là cơ hội để học sinh giao lưu, trao đổi kinh nghiệm và củng cố tinh thần đoàn kết giữa các cộng đồng người Hungary trên thế giới.

\textbf{Lịch sử và mục tiêu}

Ý tưởng về kỳ thi NMMV xuất phát từ thầy giáo toán Mihály Bencze từ Brașov (Romania) vào năm 1991 tại Hội nghị Rátz László ở Szeged. Với sự nỗ lực của thầy György Oláh, kỳ thi đầu tiên đã được tổ chức và từ đó trở thành sự kiện thường niên. NMMV được tổ chức luân phiên giữa Hungary và các quốc gia khác như Romania, Slovakia, Ukraina và các nước Nam Tư cũ.

Mục tiêu chính của cuộc thi là tạo ra một môi trường chung cho học sinh nói tiếng Hungary có cơ hội gặp gỡ, thử sức với các bài toán đầy thử thách, từ đó phát triển tư duy logic, kỹ năng giải quyết vấn đề và làm giàu kiến thức toán học.

\textbf{Cấu trúc và thể lệ cuộc thi}

Kỳ thi NMMV được tổ chức theo hình thức thi cá nhân và chia thành ba nhóm theo độ tuổi:
\begin{enumerate}[topsep=0pt, partopsep=0pt, itemsep=0pt]
    \ii Nhóm A: Học sinh lớp 11-12
    \ii Nhóm B: Học sinh lớp 9-10
    \ii Nhóm C: Học sinh lớp 7-8
\end{enumerate}

Mỗi nhóm sẽ thi trong 4 giờ, với 4 bài toán tự luận. Đề thi yêu cầu học sinh không chỉ vận dụng kiến thức cơ bản mà còn cần tư duy sáng tạo và khả năng lập luận chặt chẽ.

Các bài toán trong kỳ thi được thiết kế để phản ánh tinh thần toán học Hungary, với sự cân bằng giữa các chủ đề:
\begin{itemize}[topsep=0pt, partopsep=0pt, itemsep=0pt]
    \ii Số học: Chứng minh chia hết, số nguyên tố, dãy số.
    \ii Đại số: Phương trình, bất phương trình, đa thức.
    \ii Hình học: Chứng minh hình học, tính diện tích, thể tích, góc.
    \ii Tổ hợp: Xác suất, hoán vị, đếm, chiến lược tối ưu.
\end{itemize}

\textbf{Quy trình đánh giá và trao giải}

Bài thi được chấm theo thang điểm 0-7 cho mỗi bài toán, với tổng điểm tối đa là 28 điểm.
Ban giám khảo đánh giá dựa trên mức độ chính xác của lời giải, cách lập luận và cách trình bày của thí sinh.

Dựa trên tổng điểm, thí sinh được xếp hạng và trao các giải thưởng.
Ngoài giải cá nhân, giải thưởng Urbán János được trao hàng năm cho một thí sinh xuất sắc từ Hungary và một thí sinh từ nước ngoài,
nhằm vinh danh những học sinh có thành tích đặc biệt xuất sắc trong kỳ thi.

\textbf{Tầm quan trọng và ý nghĩa của kỳ thi}

NMMV không chỉ là một cuộc thi toán học mà còn là một cầu nối văn hóa, giúp học sinh từ các quốc gia khác nhau có cơ hội gặp gỡ, học hỏi lẫn nhau và cùng nhau phát triển trong lĩnh vực toán học.
Cuộc thi góp phần giữ gìn và phát huy truyền thống giáo dục toán học xuất sắc của Hungary, đồng thời giúp học sinh chuẩn bị cho các kỳ thi toán học quốc gia và quốc tế.

Với lịch sử hơn ba thập kỷ phát triển, NMMV tiếp tục là một sân chơi trí tuệ uy tín, tạo cơ hội cho học sinh không chỉ thể hiện năng lực toán học mà còn giao lưu,
kết nối và phát triển tư duy logic trên một nền tảng giáo dục vững chắc.

\bigbreak

\noindent\rule{16.5cm}{0.4pt}

\textbf{Một số bài thi tiêu biểu}

\bigbreak

\begin{problem*}[2023-2024, lớp 9, bài 4]
    Giải phương hệ phương trình sau:
    \[
        \left\{
            \begin{array}{rcl}
                |x| + |y| + z &=& 2021\\
                |x| + y + |z| &=& 2023\\
                x + |y| + |z| &=& 2025\\
            \end{array}
        \right.
    \]
    và viết ra kết quả. Hỏi Ádám đã viết bao nhiêu chữ số \( 0 \)?
\end{problem*}

\begin{problem*}[2023-2024, lớp 10, bài 4]
    Chứng minh rằng, với mọi cách đặt 2023 điểm trong một hình vuông \( 241 \times 241 \),
    luôn có thể tìm được một hình vuông con kích thước \( 200 \times 200 \) sao cho nó chứa ít nhất 675 điểm.
\end{problem*}

\begin{problem*}[2023-2024, lớp 12, bài 3]
    Xét tam giác vuông \( ABC \) có đường tròn nội tiếp và đường tròn ngoại tiếp có bán kính lần lượt là \( r \) và \( R \). Gọi \( CD \) là đường cao ứng với cạnh huyền \( AB \).  
    
    Dựng hình vuông \( CEFG \) có cạnh bằng \( CD \), trong đó các đỉnh \( E \) và \( G \) lần lượt nằm trên các đoạn \( AC \) và \( BC \).
    Gọi \( T \) là diện tích phần giao nhau của tam giác \( ABC \) và hình vuông \( CEFG \), còn \( t \) là phần diện tích của hình vuông \( CEFG \) không bị tam giác \( ABC \) che phủ.  
    \begin{enumerate}[topsep=0pt, partopsep=0pt, itemsep=0pt]
        \ii Chứng minh rằng:
        \[
            \frac{t}{T} = \frac{r}{2R}.
        \]
        \ii Giá trị \( \frac{t}{T} \) có thể thay đổi trong khoảng nào?
    \end{enumerate}
\end{problem*}

\newpage

\subsection{Kỳ thi Học sinh Giỏi Quốc gia (OKTV)}

Kỳ thi Học sinh Giỏi Quốc gia (OKTV) là một trong những cuộc thi học thuật quan trọng nhất dành cho học sinh trung học tại Hungary.
Kỳ thi này không chỉ đánh giá năng lực học tập mà còn giúp học sinh giành được lợi thế trong quá trình tuyển sinh đại học.

\textbf{Điều kiện tham gia}

OKTV dành cho học sinh trung học tại Hungary, bao gồm cả học sinh chính thức và khách. Học sinh phải theo học hệ chính quy và đang học ở năm cuối hoặc gần cuối cấp theo chương trình của trường.
Để đủ điều kiện, học sinh phải học hoặc đã hoàn thành môn thi theo chương trình giảng dạy, hoặc đã thi đỗ kỳ thi phân loại của môn đó.
Đối với các môn ngoại ngữ, những học sinh có ngôn ngữ thi là tiếng mẹ đẻ hoặc đã sống một khoảng thời gian dài tại quốc gia sử dụng ngôn ngữ đó sẽ không được tham gia.

Mỗi học sinh có thể đăng ký nhiều môn nhưng chỉ có thể thi một hạng mục trong mỗi môn. Quy trình đăng ký được thực hiện thông qua trường học và phải hoàn tất thời hạn trong tháng 9.
Sau khi học sinh đăng ký, hiệu trưởng hoặc người được ủy quyền sẽ gửi thông tin lên hệ thống OKTV trước thời hạn trong tháng 9.

\textbf{Cấu trúc kỳ thi}

OKTV bao gồm hai hoặc ba vòng thi, tùy theo từng môn học. Đối với các môn có hai vòng, học sinh sẽ thi vòng đầu tại trường, nơi giáo viên sẽ chấm bài theo hướng dẫn chấm điểm của trung ương.
Những thí sinh xuất sắc nhất sẽ vào vòng chung kết, nơi đề thi khó hơn và kiểm tra toàn diện kiến thức cũng như khả năng tư duy logic.

Những môn như Toán, Vật lý, Hóa học, Tin học, Sinh học, Ngữ văn và Lịch sử có ba vòng thi.
\begin{enumerate}[topsep=0pt, partopsep=0pt, itemsep=0pt]
    \ii Vòng đầu tiên được tổ chức tại trường, nơi học sinh làm bài theo đề thi trung tâm trong khoảng 120 đến 180 phút.
    \ii Những học sinh có kết quả cao nhất sẽ tiếp tục vào vòng hai, diễn ra tại các trung tâm thi cấp khu vực với thời gian làm bài kéo dài từ 180 đến 240 phút.
    Vòng này có mức độ khó cao hơn và có thể bao gồm các bài nghiên cứu hoặc phân tích chuyên sâu.
    \ii Những thí sinh xuất sắc nhất sẽ lọt vào vòng chung kết, được tổ chức tập trung tại Budapest hoặc các trung tâm thi lớn khác.
    Đề thi vòng chung kết thường yêu cầu khả năng tư duy sáng tạo, vận dụng kiến thức một cách linh hoạt và giải quyết các bài toán hoặc tình huống phức tạp.
\end{enumerate}

\textbf{Hình thức thi}

Tùy theo môn học, kỳ thi có thể bao gồm thi viết, thi nói hoặc thi thực hành.

Trong các môn khoa học và xã hội, học sinh sẽ làm bài thi viết với thời gian từ 3 đến 4 giờ.
Môn Toán, Vật lý, Hóa học thường có 5 đến 7 bài toán phức tạp, đòi hỏi tư duy logic và kỹ năng giải quyết vấn đề.
Ngữ văn và Lịch sử có phần viết luận, trong đó học sinh phải phân tích tài liệu hoặc trình bày lập luận một cách thuyết phục.

Các môn ngoại ngữ yêu cầu học sinh thực hiện bài kiểm tra nghe – nói, trong khi các môn nghệ thuật hoặc sân khấu có thể yêu cầu thí sinh trình bày một phần trình diễn hoặc phân tích tác phẩm.
Đối với các môn khoa học ứng dụng như Tin học, Sinh học, Hóa học, học sinh có thể phải thực hiện các bài thực hành, thí nghiệm hoặc lập trình trực tiếp.

Mặc dù thí sinh không được sử dụng máy tính cá nhân, điện thoại hoặc các thiết bị điện tử, họ vẫn có thể mang theo tài liệu tham khảo bằng giấy,
bao gồm sách giáo khoa và bảng công thức nếu được phép.

\textbf{Chấm điểm và xếp hạng}

Hệ thống chấm điểm của kỳ thi OKTV được thiết kế để đảm bảo tính khách quan và minh bạch. Trong vòng đầu tiên, bài thi được chấm bởi giáo viên trường theo hướng dẫn từ Bộ Giáo dục.
Từ vòng hai trở đi, bài thi được chấm bởi hội đồng giám khảo quốc gia theo thang điểm chi tiết. Đối với các bài luận hoặc bài toán tự luận, mỗi bài thi sẽ được ít nhất hai giám khảo chấm độc lập.
Nếu có sự chênh lệch đáng kể, bài thi sẽ được chấm lại bởi giám khảo thứ ba.

Sau vòng chung kết, 30 học sinh có kết quả cao nhất sẽ được xếp hạng theo ba nhóm: Nhóm 1-10 gồm những học sinh xuất sắc nhất, Nhóm 11-20 và Nhóm 21-30.
Nếu có nhiều thí sinh có cùng số điểm, thứ hạng sẽ được quyết định dựa trên điểm từ các vòng trước. Những học sinh trong top 3 sẽ nhận được chứng nhận danh dự từ Bộ Giáo dục.

Ngoài danh hiệu và giải thưởng, kết quả OKTV có thể mang lại lợi thế lớn trong tuyển sinh đại học.
Nhiều trường đại học hàng đầu tại Hungary, như Đại học Eötvös Loránd (ELTE), Đại học BME, Đại học Szeged,
công nhận thành tích OKTV như một phần tiêu chí xét tuyển trực tiếp hoặc cộng điểm ưu tiên.
Đặc biệt, các thí sinh đạt thứ hạng cao trong môn Toán, Lý, Hóa, Sinh, Tin học có cơ hội được tuyển thẳng vào các chương trình danh giá.

\textbf{Ý nghĩa của kỳ thi}

OKTV không chỉ là một kỳ thi đánh giá năng lực học thuật mà còn là một sân chơi thử thách, giúp học sinh phát triển tư duy, trau dồi kỹ năng nghiên cứu và chuẩn bị cho tương lai.
Đây là một cơ hội quan trọng cho những ai mong muốn khẳng định bản thân và giành được lợi thế trong hành trình học thuật và sự nghiệp sau này.

\bigbreak

\noindent\rule{16.5cm}{0.4pt}

\textbf{Một số bài thi tiêu biểu}

\bigbreak

\begin{problem*}[2023-2024, nhóm 1, bài 1]
    Cho một cấp số cộng \( (a_n) \) không phải là hằng số (\( n \in \mathbb{N}^+ \)) và một hàm số được xác định trên tập số thực bởi quy tắc:
    \[
        f(x) = x^3 + a x^2 + a x + a.
    \]
    
    Biết rằng các nghiệm của phương trình \( f(x) = 0 \) là \( a_9, a_{10} \) và \( a_{11} \). Hãy tìm số hạng đầu tiên và công sai của cấp số cộng này.
\end{problem*}

\begin{problem*}[2023-2024, nhóm 2, bài 3]
    Xét một tứ giác nội tiếp \( ABCD \) có độ dài các cạnh là \( AB = a \), \( BC = b \), \( CD = c \), và \( DA = d \).  
    
    Ký hiệu \( r_a \) là bán kính của đường tròn tiếp xúc ngoài với cạnh \( AB \) tại một điểm bên trong tứ giác,
    đồng thời tiếp xúc với hai đường thẳng chứa các cạnh \( BC \) và \( AD \). Tương tự, định nghĩa các bán kính \( r_b \), \( r_c \), và \( r_d \).  
    
    Chứng minh rằng:
    \[
        \frac{1}{r_a} + \frac{1}{r_b} + \frac{1}{r_c} + \frac{1}{r_d} \geq \frac{4}{\sqrt{abcd}}.
    \]
\end{problem*}

\begin{problem*}[2023-2024, nhóm 3, bài 3]
    Cho \( k \) và \( \ell \) là các số nguyên dương, dãy số nguyên dương \( a_1 \leq a_2 \leq \dots \leq a_k \) và \( b_1 \leq b_2 \leq \dots \leq b_\ell \).
    \( q(x) \) là một đa thức có hệ số nguyên và bậc ít nhất là 1.  
    
    Giả sử rằng với mọi số nguyên dương \( n \), số $a_1^n + a_2^n + \dots + a_k^n + q(n)$ luôn là ước của số $b_1^n + b_2^n + \dots + b_\ell^n + q(n).$
    
    Chứng minh rằng \( k = \ell \) và \( a_i = b_i \) với mọi \( 1 \leq i \leq k \).
\end{problem*}

\newpage

\section{Các kỳ thi để lựa chọn các đội tuyển quốc gia}

\subsection{Középiskolai Matematikai Lapok - Tạp chí Toán học Trung học Hungary}

Középiskolai Matematikai Lapok (KöMaL) là một trong những tạp chí toán học dành cho học sinh trung học lâu đời nhất trên thế giới, được thành lập vào năm 1893 bởi Arany Dániel.
Trong hơn 130 năm hoạt động, KöMaL đã góp phần nuôi dưỡng nhiều thế hệ tài năng toán học Hungary,
nhiều người trong số đó đã đạt thành tích cao tại các kỳ thi toán quốc tế và trở thành những nhà toán học hàng đầu thế giới.

\textbf{Hệ thống bài toán và điểm số}

Một trong những nét đặc trưng của KöMaL là hệ thống bài toán dành cho học sinh giải và gửi lời giải về ban biên tập để được chấm điểm.
Hàng năm, tạp chí xuất bản khoảng 300 bài toán, chia thành nhiều cấp độ khó khác nhau,
từ bài toán cơ bản dành cho học sinh mới tiếp cận đến những bài toán nâng cao thách thức cả sinh viên đại học.

Hệ thống bài toán được chia thành các mục sau:
\begin{itemize}[topsep=0pt, partopsep=0pt, itemsep=0pt]
    \ii Bài toán cơ bản (K/B): Dành cho học sinh trung học cơ sở (K) và trung học phổ thông (B), thường liên quan đến số học, phương trình, bất đẳng thức và tổ hợp.
    \ii Bài toán nâng cao (A): Có độ khó tương đương với bài toán trong kỳ thi IMO, dành cho những học sinh xuất sắc.
    \ii Bài toán hình học (C): Tập trung vào các chủ đề hình học cổ điển, như đường tròn, tam giác, đa giác và tọa độ.
    \ii Bài toán về tư duy logic (D): Các bài toán đòi hỏi suy luận sáng tạo và cách tiếp cận mới.
    \ii Bài toán vật lý (P): Dành cho học sinh yêu thích vật lý, với các bài toán liên quan đến động lực học, điện từ học và cơ học lượng tử cơ bản.
\end{itemize}

Học sinh có thể gửi lời giải của mình qua hệ thống trực tuyến của KöMaL. Mỗi lời giải được giám khảo chấm điểm trên thang 10, với những bài toán khó có thể đạt tới 15-20 điểm.
Cuối năm, những học sinh đạt tổng điểm cao nhất được vinh danh trên tạp chí và có cơ hội nhận các giải thưởng danh giá.

\textbf{Ảnh hưởng và thống kê}

Theo thống kê, mỗi năm KöMaL thu hút khoảng 2000-3000 học sinh từ khắp Hungary và cả một số quốc gia khác. Trong số đó, hơn 500 học sinh thường xuyên gửi bài giải,
và khoảng 100 học sinh đạt thứ hạng cao nhất trong hệ thống xếp hạng của tạp chí. Tạp chí cũng có đóng góp đáng kể vào thành công của Hungary trong các kỳ thi toán quốc tế.

Nhiều nhà toán học Hungary nổi tiếng từng tham gia KöMaL khi còn là học sinh, trong đó có:
\begin{itemize}[topsep=0pt, partopsep=0pt, itemsep=0pt]
    \ii László Lovász, người đoạt Giải thưởng Abel 2021, từng giải bài trên KöMaL từ khi còn là học sinh.
    \ii Paul Erdős, một trong những nhà toán học có ảnh hưởng nhất thế kỷ 20, từng tham gia KöMaL trong thời niên thiếu.
    \ii Endre Szemerédi, người đoạt Giải thưởng Abel 2012, từng là độc giả trung thành của KöMaL và được truyền cảm hứng từ các bài toán của tạp chí.
\end{itemize}

\textbf{KöMaL trong thời đại kỹ thuật số}

Ngày nay, KöMaL vẫn giữ được vị trí quan trọng trong nền giáo dục toán học Hungary. Tạp chí hiện có phiên bản trực tuyến với hàng nghìn bài toán có thể truy cập miễn phí.
Ngoài ra, hệ thống nộp bài trực tuyến giúp học sinh từ khắp nơi trên thế giới có thể tham gia giải bài toán mà không bị giới hạn về địa lý.

Mặc dù trải qua hơn một thế kỷ tồn tại, KöMaL vẫn duy trì triết lý giáo dục toán học của mình: khuyến khích học sinh suy nghĩ sáng tạo,
giải quyết vấn đề một cách độc lập và khám phá vẻ đẹp của toán học. Nhờ vào KöMaL, Hungary tiếp tục duy trì vị thế là một trong những quốc gia có nền giáo dục toán học mạnh trên thế giới.

\bigbreak

\noindent\rule{16.5cm}{0.4pt}

\textbf{Một số bài thi tiêu biểu}

\bigbreak

\begin{problem*}[12/2024, nhóm C, bài 1833]
    (5 điểm) Giải hệ phương trình sau:
    \[
        \left\{
            \begin{array}{rcl}
                a + c &=& b\\
                a^3 - c &=& b^2\\
                a + b &=& c^3\\
            \end{array}
        \right.
    \]
    trong đó \( a, b, c \) là các số tự nhiên. 
\end{problem*}

\begin{problem*}[12/2024, nhóm B, bài 5424]
    (4 điểm) Với mọi số nguyên dương \( n \), ký hiệu \( K_n \) là hình nhận được bằng cách lấy một bảng ``cờ vua'' kích thước \( (2n) \times (2n) \),
    rồi cắt bỏ bốn hình vuông kích thước \( (n-1) \times (n-1) \) ở bốn góc, như trong hình minh họa bên dưới.
    \begin{center}
        \includegraphics[width=10cm]{../Learning-Problem-Solving-2nd-Edition/png/komal-12-2024-b-5454.png}
    \end{center}
    
    Ký hiệu \( a_n \) là số cách lát kín \( K_n \) bằng các domino kích thước \( 2 \times 1 \) mà không có khe hở hay chồng lấn. (Ví dụ: \( a_1 = 2 \), \( a_2 = 8 \)).  
    Chứng minh rằng \( 2a_n \) luôn là một số chính phương với mọi \( n \).
\end{problem*}

\begin{problem*}[12/2024, nhóm A, bài 895]
    (7 điểm) Gọi một hàm \( f: \mathbb{R} \to \mathbb{R} \) là \textit{tuần hoàn yếu} nếu nó liên tục và thỏa mãn:
    \[
        f(x+1) = f(f(x)) + 1, \quad \forall x \in \mathbb{R}.
    \]
    
    \begin{enumerate}[topsep=0pt, partopsep=0pt, itemsep=0pt]
        \ii Có tồn tại một hàm tuần hoàn yếu thỏa mãn \( f(x) > x \) với mọi \( x \in \mathbb{R} \) không?  
        \ii Có tồn tại một hàm tuần hoàn yếu thỏa mãn \( f(x) < x \) với mọi \( x \in \mathbb{R} \) không?
    \end{enumerate}
\end{problem*}

\newpage

\subsection{Kỳ thi Toán học Kürschák József}

Kỳ thi Toán học Kürschák József là một trong những cuộc thi toán lâu đời và danh giá nhất tại Hungary,
được tổ chức lần đầu tiên vào năm 1894 bởi Hội Toán học và Vật lý Hungary (tiền thân của Hội Toán học Bolyai János).
Cuộc thi ban đầu được lập ra nhằm vinh danh Bá tước Eötvös Loránd khi ông được bổ nhiệm làm Bộ trưởng Tôn giáo và Giáo dục Công cộng.
Trải qua nhiều giai đoạn phát triển và thay đổi, kỳ thi này trở thành một biểu tượng trong nền toán học Hungary, với nhiều nhà toán học nổi tiếng từng tham gia và đạt thành tích cao.

\textbf{Lịch sử và sự phát triển}

Trong những năm đầu, kỳ thi được gọi là “Kỳ thi Toán học và Vật lý của Hội” và chỉ dành cho học sinh đã tốt nghiệp trung học.
Sau này, nó được đổi tên thành “Kỳ thi Toán học Eötvös Loránd” để vinh danh nhà khoa học này sau khi ông qua đời.
Tuy nhiên, từ năm 1949, cuộc thi chính thức mang tên “Kỳ thi Toán học Kürschák József” nhằm ghi nhận đóng góp to lớn của Kürschák József,
một trong những nhà toán học có ảnh hưởng nhất tại Hungary.

Ban đầu, kỳ thi chỉ diễn ra tại Budapest và Kolozsvár, nhưng do những biến động lịch sử như Thế chiến thứ nhất,
Thế chiến thứ hai và cuộc cách mạng Hungary năm 1956, có một số năm cuộc thi không được tổ chức.
Sau năm 1947, Hội Toán học Bolyai János đã tái tổ chức kỳ thi và mở rộng phạm vi cho cả học sinh trung học.
Từ đó, kỳ thi trở thành một trong những tiêu chuẩn quan trọng để đánh giá và phát hiện tài năng toán học của học sinh Hungary.

\textbf{Cấu trúc và nội dung kỳ thi}

Kỳ thi Toán học Kürschák József diễn ra hàng năm vào khoảng tháng 10 hoặc tháng 11 và chỉ gồm một vòng duy nhất. Mỗi thí sinh có 4 giờ để giải quyết 3 bài toán.
Không giống như nhiều kỳ thi khác, thí sinh không được phép sử dụng bất kỳ tài liệu tham khảo hay công cụ hỗ trợ nào.
Mục tiêu của kỳ thi không chỉ là kiểm tra khả năng giải toán mà còn đánh giá tư duy sáng tạo, cách tiếp cận vấn đề và khả năng lập luận logic của thí sinh.

Các bài toán trong kỳ thi thường được thiết kế sao cho không đòi hỏi kiến thức vượt quá chương trình trung học nhưng lại yêu cầu khả năng tư duy linh hoạt và sáng tạo.
Các lĩnh vực chính thường xuất hiện trong đề thi bao gồm:
\begin{itemize}[topsep=0pt, partopsep=0pt, itemsep=0pt]
    \ii Số học và Lý thuyết số: Các bài toán liên quan đến tính chất số nguyên, chia hết, số chính phương, đồng dư, và các định lý cơ bản như định lý Fermat nhỏ, định lý Wilson.
    Ví dụ, một bài toán có thể yêu cầu chứng minh rằng một dãy số vô hạn có một số vô hạn các số nguyên tố.
    \ii Đại số: Các bài toán về đa thức, phương trình hàm, bất đẳng thức, hệ phương trình.
    Chẳng hạn, một bài toán có thể yêu cầu chứng minh một bất đẳng thức liên quan đến trung bình cộng và trung bình nhân.
    \ii Hình học: Đề thi thường có ít nhất một bài hình học thuần túy, yêu cầu chứng minh một tính chất của đường tròn, tam giác hoặc các phép biến đổi hình học.
    Một ví dụ điển hình là bài toán chứng minh rằng trong một đa giác lồi $n$ cạnh, có thể chọn tối đa $n$ đường chéo sao cho chúng đôi một cắt nhau.
    \ii Tổ hợp: Những bài toán đếm, nguyên lý Dirichlet, lý thuyết đồ thị, sắp xếp và bài toán chu trình Hamilton.
    Một bài toán trong kỳ thi năm 2007 yêu cầu thí sinh áp dụng thuật toán Gale-Shapley để tìm cách ghép đôi ổn định giữa nam và nữ dựa trên sở thích cá nhân.
\end{itemize}

\textbf{Cách chấm điểm và đánh giá}

Các bài làm được chấm dựa trên độ chính xác, sự chặt chẽ của lập luận, mức độ sáng tạo trong cách tiếp cận và trình bày.
Các bài toán không chỉ yêu cầu một đáp án đúng mà còn phải có một lời giải rõ ràng, hợp lý.
Những thí sinh có bài làm xuất sắc nhất sẽ được trao giải thưởng Kürschák, đây là một trong những danh hiệu danh giá nhất trong hệ thống các cuộc thi toán học của Hungary.

Một bài toán có thể được chấm tối đa 10 điểm, tùy thuộc vào mức độ hoàn chỉnh của lời giải.
Nếu một lời giải có hướng tiếp cận đúng nhưng chưa đầy đủ, thí sinh có thể được một phần điểm.
Những lời giải có tính sáng tạo đặc biệt, thể hiện cách nhìn nhận vấn đề mới mẻ thường được đánh giá cao.

Các thí sinh đạt kết quả cao trong kỳ thi thường tiếp tục tham gia các kỳ thi lớn hơn như Olympic Toán Quốc tế (IMO).
Đặc biệt, trong nhiều năm, những học sinh đạt giải cao trong cuộc thi này được miễn kỳ thi đầu vào đại học và được xét tuyển trực tiếp vào các chương trình toán học hàng đầu tại Hungary.

\textbf{Tầm quan trọng và ảnh hưởng}

Kỳ thi Toán học Kürschák József không chỉ giúp phát hiện các tài năng toán học mà còn đóng góp quan trọng vào việc nâng cao chất lượng giáo dục toán học tại Hungary.
Những bài toán được chọn lọc kỹ lưỡng qua từng năm đã trở thành nguồn tài liệu quý giá cho học sinh và giáo viên.

Các bài toán từ kỳ thi đã được tập hợp trong tuyển tập “Matematikai Versenytételek” do Kürschák khởi xướng.
Bộ sách này không chỉ chứa các đề thi mà còn kèm theo nhiều lời giải chi tiết, giúp học sinh hiểu sâu hơn về các phương pháp giải toán.
Tuyển tập này đã được dịch ra nhiều ngôn ngữ, bao gồm tiếng Anh, Nhật, Nga, Romania và Ba Tư, cho thấy tầm ảnh hưởng rộng rãi của cuộc thi.

Những người chiến thắng trong kỳ thi này đã trở thành những nhà toán học hàng đầu, đóng góp đáng kể cho nền toán học thế giới.
Trong số đó có nhiều tên tuổi lớn như Fejér Lipót, Kármán Tódor, Kőnig Dénes, Haar Alfréd và Teller Ede.

\bigbreak

\noindent\rule{16.5cm}{0.4pt}

\textbf{Một số bài thi tiêu biểu}

\bigbreak

\begin{problem*}[2024, bài 1]
    Cho tứ giác \( ABCD \) được chia thành một số tứ giác nội tiếp sao cho:
    \begin{itemize}[topsep=0pt, partopsep=0pt, itemsep=0pt]
        \ii Không có hai tứ giác nào trong phân hoạch này có điểm trong chung (điểm trong là điểm không nằm trên chu vi).
        \ii Các đỉnh của mỗi tứ giác trong phân hoạch không nằm trên bất kỳ cạnh nào của một tứ giác khác trong phân hoạch, cũng như không nằm trên cạnh nào của tứ giác \( ABCD \).
    \end{itemize}
    Chứng minh rằng \( ABCD \) cũng là một tứ giác nội tiếp.
\end{problem*}

\begin{problem*}[2024, bài 2]
    Vương quốc Một Chiều trong thời cổ đại nằm dọc theo một đường thẳng. Ban đầu không có thành phố nào. Các thành phố được thành lập lần lượt tại \( n \) vị trí khác nhau.  
    Từ thành phố thứ hai trở đi, mỗi thành phố mới thành lập sẽ được ghép đôi làm \textit{thành phố anh em} với thành phố gần nhất đã tồn tại (nếu có hai thành phố gần nhất,
    chọn thành phố được thành lập trước đó).  

    Trên bản đồ còn sót lại của vương quốc, chỉ hiển thị vị trí các thành phố và khoảng cách giữa chúng, nhưng không có thông tin về thứ tự thành lập.
    Các nhà sử học đang cố gắng xác định xem có thể suy ra từ bản đồ rằng mỗi thành phố có tối đa 41 thành phố anh em hay không.  

    \begin{enumerate}[topsep=0pt, partopsep=0pt, itemsep=0pt]
        \ii Với \( n = 10^6 \), hãy đưa ra một bản đồ sao cho có thể suy ra điều kiện trên.  
        \ii Chứng minh rằng với \( n = 10^{13} \), không có bản đồ nào có thể dẫn đến kết luận này.  
    \end{enumerate}
\end{problem*}

\begin{problem*}[2024, bài 3]
    Gọi \( p \) là một số nguyên tố và \( H \subseteq \{0, 1, \dots, p-1\} \) là một tập hợp không rỗng.
    Giả sử rằng với mỗi phần tử \( a \in H \), tồn tại hai phần tử khác \( a \), ký hiệu \( b, c \in H \), sao cho:
    \[
        b + c - 2a \equiv 0 \Mod{p}.
    \]
    
    Chứng minh rằng:
    \[
        p < 4k,
    \]
    trong đó \( k \) là số phần tử của tập \( H \).
\end{problem*}

\newpage

\subsection{Olympiad Toán học Trung Âu (MEMO) - Sân chơi quốc tế cho học sinh yêu toán}

Olympiad Toán học Trung Âu (Middle European Mathematical Olympiad - MEMO) là một kỳ thi toán học quốc tế dành cho học sinh trung học đến từ các quốc gia Trung Âu.
Kỳ thi được tổ chức lần đầu tiên vào năm 2007, xuất phát từ kỳ thi Toán học Áo-Ba Lan,
với mục tiêu giúp học sinh rèn luyện và tích lũy kinh nghiệm thi đấu quốc tế trước khi tham gia các kỳ thi lớn hơn như Olympic Toán học Quốc tế (IMO).
Vì vậy, MEMO không chỉ là một cuộc thi mà còn là một bước đệm quan trọng trong sự nghiệp toán học của nhiều học sinh trong khu vực.

Kỳ thi diễn ra hàng năm vào tháng 8 hoặc tháng 9 với sự tham gia của 10 quốc gia: Áo, Ba Lan, Cộng hòa Séc, Croatia, Đức, Hungary, Litva, Slovakia, Slovenia và Thụy Sĩ.
Ngoài ra, nước chủ nhà có thể mời thêm một đội tuyển khách. Cuộc thi không mở đăng ký rộng rãi mà chỉ dành cho các đội tuyển được mời, và việc tham gia có thể đi kèm với phí đăng ký.

MEMO có cấu trúc đặc biệt so với các kỳ thi toán học khác, bao gồm cả phần thi cá nhân và phần thi đồng đội, nhấn mạnh không chỉ vào năng lực cá nhân mà còn vào khả năng làm việc nhóm.

\textbf{Cấu trúc kỳ thi và nội dung bài toán}

MEMO gồm hai phần thi: thi cá nhân và thi đồng đội.

Phần thi cá nhân diễn ra trong 5 tiếng, trong đó mỗi thí sinh phải giải 4 bài toán thuộc 4 lĩnh vực: đại số, tổ hợp, hình học và số học.
Các bài toán này có độ khó tương đương với IMO, đòi hỏi tư duy sáng tạo, kỹ năng lập luận chặt chẽ và khả năng trình bày khoa học.

Phần thi đồng đội cũng kéo dài 5 tiếng, trong đó mỗi đội 6 thành viên sẽ cùng nhau giải quyết 8 bài toán (2 bài thuộc mỗi lĩnh vực).
Phần thi này yêu cầu sự phối hợp hiệu quả giữa các thành viên, phân chia công việc hợp lý và tận dụng điểm mạnh của từng người để đạt được kết quả tối ưu.
Đây là điểm khác biệt đáng chú ý so với IMO, nơi phần thi cá nhân là trọng tâm chính.

\textbf{Lựa chọn bài toán và chấm điểm}

Các bài toán của MEMO được đề xuất bởi các quốc gia tham gia và được lựa chọn bởi ban tổ chức của nước chủ nhà.
Danh sách sơ bộ sẽ được gửi đến trưởng đoàn và phó trưởng đoàn ít nhất 3 tuần trước cuộc thi để đánh giá.
Quá trình chọn lọc dựa trên hai tiêu chí: độ hấp dẫn (thang điểm 1-3) và độ khó (thang điểm 1-5). Sau đó, hội đồng giám khảo sẽ họp để chọn bộ đề chính thức.

Bài thi được chấm theo thang điểm 7, giống như IMO. Điểm số được đánh giá dựa trên mức độ chính xác của lời giải, cách trình bày và phương pháp tiếp cận bài toán.
Những bài toán có cách giải sáng tạo hoặc lập luận chặt chẽ sẽ được đánh giá cao hơn.
Sau khi chấm điểm, kết quả sẽ được công bố và học sinh có thể theo dõi tiến trình của mình so với các thí sinh khác.

\textbf{Lịch sử hình thành và phát triển}

MEMO bắt nguồn từ kỳ thi Toán học Áo-Ba Lan, một cuộc thi song phương giữa hai nước.
Năm 2007, kỳ thi này mở rộng quy mô với sự tham gia của Cộng hòa Séc, Croatia, Thụy Sĩ, Slovakia và Slovenia, tạo tiền đề cho sự ra đời của MEMO.
Sau đó, năm 2008, Đức và Hungary gia nhập, và đến năm 2009, Litva trở thành thành viên thứ 10.

Kể từ đó, kỳ thi được tổ chức luân phiên giữa các quốc gia thành viên, với mỗi nước đăng cai một lần.
Việc tổ chức không chỉ nhằm tìm kiếm những tài năng toán học xuất sắc mà còn thúc đẩy sự hợp tác giáo dục giữa các nước Trung Âu trong lĩnh vực toán học.

\textbf{Ý nghĩa và vai trò của MEMO}

MEMO không chỉ là một cuộc thi mà còn là cơ hội giúp học sinh phát triển tư duy toán học và rèn luyện khả năng giải quyết vấn đề.
Việc tham gia kỳ thi giúp học sinh làm quen với môi trường thi đấu quốc tế, nâng cao kỹ năng tư duy sáng tạo và trình bày toán học.

Ngoài ra, kỳ thi cũng tạo ra một cộng đồng toán học quốc tế, nơi học sinh có thể giao lưu, học hỏi từ bạn bè quốc tế và mở rộng tầm nhìn về toán học.
Không ít học sinh từng tham gia MEMO đã tiếp tục theo đuổi sự nghiệp nghiên cứu toán học hoặc đạt thành tích cao trong các kỳ thi quốc tế lớn hơn.

\bigbreak

\noindent\rule{16.5cm}{0.4pt}

\textbf{Một số bài thi tiêu biểu}

\bigbreak

\begin{problem*}[2024, bài 1]
    Xác định tất cả các số \( k \in \mathbb{N}_0 \) sao cho tồn tại hàm số \( f: \mathbb{N}_0 \to \mathbb{N}_0 \) thoả mãn:
    \[
        f(2024) = k,\ f(f(n)) \leq f(n+1) - f(n), \quad \forall n \in \mathbb{N}_0.
    \]
   
    \textit{Ở đây, \( \mathbb{N}_0 \) là tập hợp tất các các số nguyên không âm.}
\end{problem*}

\begin{problem*}[2024, bài 2]
    Trên một bảng đen vô hạn có một tờ giấy. Marvin bí mật chọn một đa giác lồi \( P \) có 2024 đỉnh nằm hoàn toàn trong tờ giấy. Tigerin muốn xác định các đỉnh của \( P \).  
    
    Trong mỗi bước, Tigerin có thể vẽ một đường thẳng \( g \) trên bảng đen, sao cho đường thẳng này nằm hoàn toàn bên ngoài tờ giấy.
    Marvin sau đó phản hồi bằng cách vẽ đường thẳng \( h \), song song với \( g \) và là đường gần nhất với \( g \) mà đi qua ít nhất một đỉnh của \( P \).  
    
    Chứng minh rằng tồn tại một số nguyên dương \( n \) sao cho Tigerin luôn có thể xác định được các đỉnh của \( P \) trong nhiều nhất \( n \) bước.
\end{problem*}

\begin{problem*}[2024, bài 3]
    Cho tam giác nhọn \( ABC \) không cân. Chọn một đường tròn \( \omega \) đi qua \( B \) và \( C \),
    cắt các đoạn thẳng \( AB \) và \( AC \) lần lượt tại \( D \neq A \) và \( E \neq A \). Gọi \( F \) là giao điểm của \( BE \) và \( CD \).  
    
    Gọi \( G \) là điểm trên đường tròn ngoại tiếp tam giác \( ABF \) sao cho \( GB \) là tiếp tuyến của \( \omega \).
    Tương tự, gọi \( H \) là điểm trên đường tròn ngoại tiếp tam giác \( ACF \) sao cho \( HC \) là tiếp tuyến của \( \omega \).  
    
    Chứng minh rằng tồn tại một điểm \( T \neq A \), không phụ thuộc vào cách chọn \( \omega \), sao cho đường tròn ngoại tiếp tam giác \( AGH \) luôn đi qua \( T \).
\end{problem*}

\newpage

\section{Đội tuyển Hungary qua các kỳ thi IMO}

\subsection{Quy trình tuyển chọn đội tuyển Olympic Toán Quốc tế IMO, MEMO, EGMO của Hungary}

Hungary có một hệ thống tuyển chọn nghiêm ngặt để chọn ra những học sinh xuất sắc nhất đại diện quốc gia tại các kỳ thi toán quốc tế như IMO, EGMO và MEMO.
Hungary là một trong những quốc gia có truyền thống mạnh trong các kỳ thi này, với nhiều huy chương vàng và thành tích xuất sắc trong lịch sử IMO.
Quy trình này được thiết kế nhằm đánh giá toàn diện năng lực thí sinh thông qua nhiều kỳ thi khác nhau, đảm bảo rằng đội tuyển được tuyển chọn từ những học sinh có thành tích tốt nhất.

Đặc biệt thông tin xếp hạng các học sinh được công khai và cập nhật tức thời tại: https://cms.renyi.hu/olimpiak/hu/allas.

\textbf{Các kỳ thi quan trọng trong quá trình tuyển chọn}
Quá trình tuyển chọn dựa trên thành tích của học sinh trong các kỳ thi toán học quan trọng. Để đủ điều kiện tham gia các vòng tuyển chọn,
học sinh phải đang theo học trung học phổ thông tại Hungary và có thành tích nổi bật trong các kỳ thi toán học cấp quốc gia hoặc các cuộc thi quốc tế trước đó.

\begin{itemize}[topsep=0pt, partopsep=0pt, itemsep=0pt]
    \ii \textbf{Kỳ thi Kürschák József}: Một trong những kỳ thi toán lâu đời nhất thế giới, dành cho học sinh trung học và sinh viên năm nhất.
    Kỳ thi gồm 3 bài toán khó, yêu cầu lời giải chi tiết. Thành tích trong kỳ thi này có ảnh hưởng đáng kể đến điểm tuyển chọn.
    \ii \textbf{Kỳ thi KöMaL (A \& B)}: Được tổ chức hàng tháng bởi tạp chí "Középiskolai Matematikai és Fizikai Lapok", gồm hai hạng mục:
    \begin{itemize}
        \ii Cuộc thi "A" với các bài toán nâng cao, yêu cầu chứng minh chi tiết.
        \ii Cuộc thi "B" phù hợp với nhiều học sinh hơn, với bài toán đơn giản hơn.
    \end{itemize}
    \ii \textbf{Các kỳ thi tuyển chọn quốc gia (Válogatóversenyek)}: Bao gồm bốn kỳ thi tuyển chọn chính, được tổ chức từ tháng 11 đến tháng 4,
    đóng vai trò quyết định trong việc chọn đội tuyển IMO, EGMO và MEMO.
\end{itemize}

\textbf{Các vòng tuyển chọn và cách tính điểm}
Việc tính điểm và xếp hạng thí sinh được thực hiện dựa trên kết quả từ các kỳ thi trên. Mỗi thí sinh tích lũy điểm theo công thức:

\begin{center}
    \begin{tabular}{|c|c|c|c|}
        \hline
        \textbf{Vòng thi} & \textbf{Thời gian} & \textbf{Số bài toán} & \textbf{Điểm tối đa} \\
        \hline
        V1 (Vòng tuyển chọn thứ nhất) & Tháng 11 & 4 bài & 28 điểm \\
        \hline
        V2 (Vòng tuyển chọn thứ hai/Surányi János Memorial) & Tháng 3 & 3 bài & 21 điểm \\
        \hline
        V3 (Vòng tuyển chọn thứ ba) & Tháng 4 & 3 bài & 21 điểm \\
        \hline
        V4 (Vòng tuyển chọn thứ tư) & Tháng 4 & 3 bài & 21 điểm \\
        \hline
    \end{tabular}    
\end{center}

Ngoài ra, học sinh còn có thể nhận điểm thưởng từ:
\begin{itemize}[topsep=0pt, partopsep=0pt, itemsep=0pt]
    \ii \textbf{Kỳ thi Kürschák (Kü)}: Điểm số dựa vào thành tích đạt được
    (10 điểm cho giải nhất, 7 điểm cho giải nhì, 4 điểm cho giải ba, 2 điểm cho giải khuyến khích, 0 điểm nếu không đạt giải).
    \ii \textbf{Điểm KöMaL (Kö1, Kö2)}: Được tính dựa trên tỷ lệ phần trăm số bài toán giải đúng từ cuộc thi KöMaL trong hai năm liên tiếp, với tối đa 4 điểm cho mỗi năm.
\end{itemize}

Sau khi tính tổng điểm, 6 học sinh có điểm cao nhất sẽ được chọn vào đội tuyển IMO.
Trong trường hợp có hai học sinh có cùng điểm số, ban tuyển chọn có thể xem xét thêm kết quả từ các kỳ thi trước đó hoặc tổ chức một bài kiểm tra bổ sung để quyết định thứ hạng cuối cùng.
Các học sinh có điểm cao tiếp theo (thường xếp hạng 7-12) có thể được cân nhắc tham gia MEMO.

\textbf{Tuyển chọn đội tuyển EGMO}

Hungary cũng cử đội tuyển tham gia kỳ thi EGMO dành riêng cho nữ sinh.
Hungary đã có thành tích ấn tượng tại EGMO, với nhiều năm giành huy chương vàng, bạc và đồng, khẳng định năng lực của các nữ sinh trong đấu trường toán học quốc tế.
Quá trình tuyển chọn đội tuyển EGMO dựa trên kết quả của vòng tuyển chọn IMO đầu tiên (V1) và thành tích trong Kürschák \& KöMaL.
Bốn nữ sinh có điểm cao nhất sẽ được chọn vào đội tuyển chính thức, với khả năng có 1-2 thí sinh dự bị.

\textbf{Tuyển chọn đội tuyển MEMO}

Đội tuyển MEMO gồm 6 học sinh, thường là những học sinh có điểm cao nhưng không lọt vào top 6 IMO. Quá trình tuyển chọn dựa vào các kỳ thi tuyển chọn quốc gia (V1-V4), Kürschák và KöMaL.
Những thí sinh xếp hạng 7-12 thường có cơ hội được chọn.

\textbf{Yêu cầu và quy định}

\begin{itemize}[topsep=0pt, partopsep=0pt, itemsep=0pt]
    \ii \textbf{Đăng ký tham gia}: Học sinh phải đăng ký trước cho các kỳ thi tuyển chọn và kỳ thi Kürschák.
    \ii \textbf{Quy định trong kỳ thi}: Chỉ được sử dụng compa, thước kẻ và bút viết. Không được dùng máy tính hoặc tài liệu hỗ trợ.
\end{itemize}

\textbf{Tóm tắt quy trình tuyển chọn}
\begin{center}
    \begin{tabular}{|c|c|c|}
        \hline
        \textbf{Kỳ thi} & \textbf{Số thí sinh} & \textbf{Cách tuyển chọn} \\
        \hline
        IMO & 6 học sinh & Tổng điểm cao nhất từ V1 + V2 + V3 + V4 + Kürschák + KöMaL \\
        \hline
        EGMO & 4 nữ sinh & Điểm cao nhất từ V1 + thành tích trong Kürschák \& KöMaL \\
        \hline
        MEMO & 6 học sinh & Điểm cao nhưng không lọt vào top 6 IMO (thường là hạng 7-12) \\
        \hline
    \end{tabular} 
\end{center}

\textbf{Kết luận}

Hệ thống tuyển chọn đội tuyển toán Hungary đảm bảo tính công bằng, minh bạch và toàn diện.
Việc kết hợp nhiều nguồn điểm giúp chọn ra những học sinh có năng lực tốt nhất, không chỉ dựa vào một kỳ thi duy nhất mà đánh giá qua cả quá trình học tập và thi đấu.
Nhờ quy trình này, Hungary luôn có đội tuyển mạnh tại các kỳ thi toán quốc tế, khẳng định vị thế của mình trên đấu trường IMO, EGMO và MEMO.

\newpage

\subsection{Lịch sử và thành tích của Hungary tại Olympic Toán Quốc tế (IMO)}

Hungary có một lịch sử đáng tự hào tại Olympic Toán Quốc tế (IMO), thể hiện truyền thống toán học mạnh mẽ và cam kết nuôi dưỡng tài năng trẻ.

\textbf{Thành tích tổng quát}

\begin{itemize}[topsep=0pt, partopsep=0pt, itemsep=0pt]
    \ii \textbf{Lần tham gia đầu tiên}: 1959
    \ii \textbf{Số lần tham gia}: 64
    \ii \textbf{Bảng thành tích huy chương}:
    \begin{itemize}[topsep=0pt, partopsep=0pt, itemsep=0pt]
        \ii Vàng: 88
        \ii Bạc: 174
        \ii Đồng: 116
        \ii Giải khuyến khích: 10
    \end{itemize}
\end{itemize}

\textbf{Những thành tựu nổi bật}
\begin{itemize}[topsep=0pt, partopsep=0pt, itemsep=0pt]
    \ii Hungary giành vị trí số một toàn đoàn sáu lần vào các năm 1961, 1962, 1969, 1970, 1971 và 1975.
    \ii Đội tuyển Hungary năm 1971 đặc biệt ấn tượng với bốn thí sinh giành huy chương vàng và bốn thí sinh giành huy chương bạc.
    Đáng chú ý, bảy trong số những thành viên này đến từ một trường trung học danh tiếng ở Budapest, và ba trong số những thí sinh giành huy chương vàng sau này đã có sự nghiệp học thuật xuất sắc.
\end{itemize}

\textbf{Những cá nhân xuất sắc}

\begin{itemize}[topsep=0pt, partopsep=0pt, itemsep=0pt]
    \ii \textbf{László Lovász}: Tham gia IMO bốn lần (1963-1965), đạt ba huy chương vàng, hai giải đặc biệt, và một huy chương bạc. 
    Ông sau này trở thành một nhà toán học nổi tiếng, giữ chức Chủ tịch Liên minh Toán học Quốc tế và có nhiều đóng góp quan trọng trong lĩnh vực tổ hợp và khoa học máy tính lý thuyết.
    \ii \textbf{József Pelikán}: Tham gia IMO bốn lần (1963-1966), đạt ba huy chương vàng, hai giải đặc biệt, và một huy chương bạc.  
    Sau đó, ông giữ nhiều vai trò quan trọng trong công tác tổ chức IMO, bao gồm việc làm Chủ tịch Hội đồng Cố vấn của kỳ thi.
\end{itemize}

\textbf{Các giai đoạn thành tích của Hungary tại IMO}
\begin{itemize}[topsep=0pt, partopsep=0pt, itemsep=0pt]
    \ii Giai đoạn đầu cho đến trước năm 1990 (1959–1989): Hungary là một trong những quốc gia sáng lập IMO vào năm 1959.
    Trong giai đoạn này, đội tuyển Hungary luôn nằm trong nhóm dẫn đầu, giành vị trí số một toàn đoàn sáu lần.
    Đặc biệt, năm 1975, Hungary giành giải nhất đồng đội dù không có huy chương vàng cá nhân nào, với thành tích năm huy chương bạc và ba huy chương đồng.
    \ii Giai đoạn chuyển tiếp (1990-2004): Những năm 1990 và đầu những năm 2000 là thời kỳ chuyển đổi của Hungary tại IMO.
    Dù vẫn duy trì thành tích tốt, nhưng sự cạnh tranh gia tăng khi ngày càng nhiều quốc gia tham gia.
    Trong giai đoạn này, Hungary vẫn nằm trong nhóm 20 đội tuyển mạnh nhất, với nhiều cá nhân giành huy chương vàng.
    \ii Thành tích gần đây (2005-nay): Trong những năm gần đây, Hungary vẫn tiếp tục duy trì thành tích đáng kể tại IMO.
    Đội tuyển thường xuyên nằm trong top 30, với một số thành tích nổi bật như vị trí thứ 8 vào năm 2024.
    Nhiều thí sinh cá nhân cũng xuất sắc giành huy chương tại các kỳ thi gần đây.
\end{itemize}

\textbf{Thành tích đội tuyển Hungary tại IMO suy giảm}

Hungary từng là một trong những quốc gia hàng đầu tại IMO, nhưng thứ hạng của đội tuyển đã giảm dần theo thời gian.
Khi so sánh với các quốc gia hiện đang có thành tích tốt hơn, có thể thấy một số yếu tố quan trọng đã góp phần vào sự suy giảm này.

\begin{itemize}[topsep=0pt, partopsep=0pt, itemsep=0pt]
    \ii \textit{Hệ thống giáo dục và mức lương giáo viên}: Mức lương giáo viên và hệ thống giáo dục vững chắc đóng vai trò quan trọng trong việc thu hút và giữ chân những giáo viên chất lượng cao.
    Các quốc gia như Trung Quốc và Hàn Quốc có hệ thống giáo dục chuyên sâu với sự hỗ trợ mạnh mẽ từ chính phủ, đảm bảo mức lương hấp dẫn cho giáo viên.
    Trong khi đó, hệ thống giáo dục của Hungary gặp phải những thách thức, bao gồm mức lương giáo viên chưa đủ cạnh tranh, ảnh hưởng đến chất lượng giảng dạy.
    \ii \textit{Đầu tư kinh tế vào giáo dục}: Các quốc gia dành nhiều ngân sách hơn cho giáo dục thường có thành tích học thuật tốt hơn.
    Ví dụ, Singapore đầu tư mạnh vào hệ thống giáo dục STEM, giúp quốc gia này gặt hái nhiều thành công trong các kỳ thi quốc tế.
    Trong khi đó, mức đầu tư của Hungary vào giáo dục thấp hơn, hạn chế nguồn lực dành cho việc đào tạo học sinh tham dự IMO.
    \ii \textit{Nhân khẩu học và nguồn tài năng}: Cơ cấu dân số ảnh hưởng đến số lượng học sinh có thể tham gia các cuộc thi toán học. Ấn Độ có dân số trẻ đông đảo, mang lại nguồn tài năng dồi dào để đào tạo cho IMO.
    Ngược lại, Hungary đang đối mặt với tỷ lệ sinh giảm, làm giảm số lượng học sinh có thể được tuyển chọn và đào tạo.
    \ii \textit{Thích ứng với các phương pháp giáo dục hiện đại}: Các quốc gia áp dụng các phương pháp giáo dục hiện đại thường có thành tích tốt hơn trong các kỳ thi quốc tế.
    Ví dụ, Hoa Kỳ triển khai các trại huấn luyện chuyên sâu và hội thảo giải toán nâng cao, giúp họ cải thiện đáng kể thứ hạng.
    Trong khi đó, phương pháp đào tạo truyền thống của Hungary có thể chưa theo kịp sự phát triển của các mô hình huấn luyện hiện đại.
\end{itemize}

Hungary đang xem xét các chiến lược của những quốc gia đang thành công tại IMO để xây dựng lại vị thế của mình.
Việc triển khai các chương trình Chương trình Tài năng Quốc gia, Quỹ Vì Trẻ Em Tài Năng Toán Học, duy trì các kỳ thi quốc gia, tham gia các kỳ thi khu vực,
và công khai thông tin đảm bảo cho các kỳ thi tuyển chọn các đội tuyển có đầy đủ sự minh bạch.

\newpage

\section*{References}

\begin{thebibliography}{99}

\bibitem{OKTV} Oktatási Hivatal. "OKTV Versenyfeladatok." Available at: \url{https://www.oktatas.hu/kozneveles/tanulmanyi_versenyek_/oktv_kereteben/versenyfeladatok_javitasi_utmutatok}

\bibitem{AranyDaniel} Bolyai János Matematikai Társulat. "Arany Dániel Matematikai Tanulóverseny." Available at: \url{https://www.bolyai.hu/versenyek-arany-daniel-matematikaverseny/}

\bibitem{Zrinyi} MATEGYE Alapítvány. "Zrínyi Ilona Matematikaverseny." Available at: \url{http://www.mategye.hu/?pid=zrinyi_verseny}

\bibitem{VargaTamas} MATEGYE Alapítvány. "Varga Tamás Országos Matematikaverseny." Available at: \url{http://www.mategye.hu/?pid=vargatamasverseny}

\bibitem{KecskeKupa} MATEGYE Alapítvány. "Kecske Kupa." Available at: \url{http://www.mategye.hu/?pid=tehetsegnap_kecskekupa}

\bibitem{Megyei} MATEGYE Alapítvány. "Megyei Matematikaverseny." Available at: \url{http://www.mategye.hu/?pid=megyeiverseny}

\bibitem{Durer} Dürer Matematikaverseny. "Dürer Matematikaverseny." Available at: \url{https://durerinfo.hu}

\bibitem{NMMV} Nemzetközi Magyar Matematikaverseny. "Nemzetközi Magyar Matematikaverseny." Available at: \url{https://nmmv.berzsenyi.hu/főoldal}

\bibitem{MatHatNelkul} Berzsenyi Dániel Gimnázium. "Matematika Határok Nélkül." Available at: \url{https://msf.berzsenyi.hu/aktuális}

\bibitem{Schweitzer} Bolyai János Matematikai Társulat. "Schweitzer Miklós Matematikai Emlékverseny." Available at: \url{https://www.bolyai.hu/versenyek-schweitzer-miklos-emlekverseny/}

\bibitem{JedlikAnyos} Jedlik Ányos Országos Matematikaverseny. Available at: \url{https://fizikaverseny.lapunk.hu/jedlik-anyos-matematikaverseny-1040038}

\bibitem{EotvosKurschak} Batmath. "Eötvös-Kürschák Competitions." Available at: \url{http://www.batmath.it/matematica/raccolte_es/ek_competitions/ek_competitions.pdf}

\bibitem{IMOSelection} IMO Válogató Rendszer. Available at:
\begin{itemize}
    \ii \url{http://agondolkodasorome.hu/versenynaptar/}
    \ii \url{https://cms.renyi.hu/olimpiak/hu/naptar}
    \ii \url{https://cms.renyi.hu/olimpiak/hu/valogato_szabalyok}
    \ii \url{https://cms.renyi.hu/olimpiak/hu/allas}
    % \ii \url{https://ematlap.hu/tanora-szakkor-2019-6/881-kibol-lesz-diakolimpikon-az-olimpiai-valogatorendszer}
\end{itemize}

\bibitem{Kurschak} Bolyai János Matematikai Társulat. "Kürschák József Matematikai Tanulóverseny." Available at: \url{https://www.bolyai.hu/versenyek-kurschak-jozsef-matematikai-tanuloverseny/}

\bibitem{Komal} Középiskolai Matematikai és Fizikai Lapok. "KöMaL" Available at: \url{https://www.komal.hu/home.h.shtml}

\bibitem{KomalPont} Középiskolai Matematikai és Fizikai Lapok. "A KöMaL pontversenyei." Available at: \url{https://www.komal.hu/verseny.h.shtml}

\bibitem{MEMO} Bolyai János Matematikai Társulat. "MEMO - Közép-európai Matematikai Olimpia." Available at: \url{https://www.bolyai.hu/matematikai-diakolimpiak-memo}

\end{thebibliography}

\end{otherlanguage*}

\end{document}