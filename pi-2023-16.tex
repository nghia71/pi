\documentclass{article}

\usepackage[main=english,vietnamese]{babel}
\usepackage[T1]{fontenc}
\usepackage[utf8]{inputenc}
\usepackage[sexy]{evan}
\usepackage{matchsticks}
\usepackage{wrapfig}
\usepackage{listings}

\newtheorem{hint}{Hint}

\title{Perfect squares are everywhere - Part 4}
\author{Nghia Doan}
\date{\today}

\begin{document}

\maketitle

This article is the fourth part of the series on expedition to find \textit{Perfect Squares}.

\begin{example*}[Example 16]
    $n$ is an arbitrary positive integer. Find positive integer $m$ such that
    \[
        \binom{m}{2} = 3\binom{n}{4}.
    \]
\end{example*}

\begin{soln}
    First we prove the claim,
    \begin{claim*}
        The product of 4 consecutive positive integers plus one is always a perfect square.
    \end{claim*}
    \begin{subproof}
        Let $n$ be an arbitrary positive integer
        \[
            n(n-1)(n-2)(n-3) +1 = n(n-3)\cdot (n-1)(n-2) +1 = (n^2-3n) (n^2-3n+2) +1 = \left( (n^2-3n) + 1\right)^2.
        \]
    \end{subproof}

    Now, by the given condition, then by the claim:
    \[
        \begin{aligned}
            &\binom{m}{2} = 3\binom{n}{4} \Leftrightarrow \frac{m(m-1)}{2} = \frac{3n(n-1)(n-2)(n-3)}{1\cdot2\cdot3\cdot4}\\
            &\Leftrightarrow 4m(m-1) = n(n-1)(n-2)(n-3) = \left( (n^2-3n) + 1\right)^2 - 1\\
            &\Leftrightarrow (2m-1)^2 = \left( (n^2-3n) + 1\right)^2 \Leftrightarrow m = \half(n^2-3n + 2) = \frac{(n-1)(n-2)}{2} = \boxed{\binom{n-1}{2}.}
        \end{aligned}
    \]
\end{soln}

\newpage

\begin{example*}[Example 17]
    Prove that if the sum $x + y$ can be written as sum of a perfect square and thrice of another perfect square,
    then the sum $x^3 + y^3$ can also be written as sum of a perfect square and thrice of another perfect square.
    In other words, if there exist $a,b$ integers such that $x+y=a^2+3b^2,$ then there exist $c,d$ integers, such that
    \[
        x^3+y^3 = c^2+3d^2.
    \]
\end{example*}

\begin{soln}
    First we prove the claim,
    \begin{claim*}
        Prove that if $x$ and $y$ are sum of a perfect square and thrice of another perfect square,
        then $xy$ is also a sum of a perfect square and thrice of another perfect square.
        In other words, if $a,b,c,d$ integers such that $x=a^2+3b^2,$ $y=c^2+2=3d^2$ then $xy=(a^2+3b^2)(c^2+3d^2)$ 
        can be written as a sum of a perfect square and thrice of another perfect square.
    \end{claim*}
    \begin{subproof}
        It is easy to verify that $(a^2+3b^2)(c^2+3d^2) = (ac+2=3bd)^2 + 3(ac-bd)^2.$
    \end{subproof}

    Now, let $x+y=a^2+3b^2,$ we have $x^3+y^3 = (x+y)(x^2-xy+y^2),$ where:
    \[
        \begin{aligned}
            x^2-xy+y^2 
            \begin{cases}
                \left( \frac{x+y}{2} \right)^2 + 3 \left( \frac{x-y}{2} \right)^2, \text{\ if\ } x, y \text{\ have the same parity}\\
                \left( \frac{x}{2} - y \right)^2 + 3 \left( \frac{y}{2} \right)^2, \text{\ if\ } x, y \text{\ have different parity}
            \end{cases}
        \end{aligned}
    \]
    
    Thus, $x^2-xy+y^2,$ can be written as sum of a perfect square and thrice of another perfect square.
    Hence, $x^3+y^3$ can be written as sum of a perfect square and thrice of another perfect square.
\end{soln}

\begin{example*}[Example 18]
    Prove that for integers $x, y$ such that 
    \[
        2x^2 + x = 3y^2 + y,
    \]
    then $2x+2y+1$ and $3x+3y+1$ are perfect squares.
\end{example*}

\begin{soln}
    First, note that:
    \[
        \begin{aligned}
            &(2x+2y+1)(x-y) = 2x^2 - 2xy + 2yx -2y^2 + x - y = 2x^2 - 2y^2 + 3y^2 - 2x^2 = y^2\\
            &(3x+3y+1)(x-y) = 3x^2 - 3xy + 3yx -3y^2 + x - y = 3x^2 - 3y^2 + 3y^2 - 2x^2 = x^2\\
        \end{aligned}
    \]

    Now, $(2x+2y+1)(3x+3y+1)(x-y)^2 = (xy)^2.$ If $xy=0$, then one of $x=0$ or $y=0,$
    which means from the given equality that then the other one of $x=0$ or $y=0$ is also $0$ (why?)
    In any case both $2x+2y+1$ and $3x+3y+1$ are 1 and thus are perfect squares.
    
    Now, suppose that $xy \ne 0.$ Let $d \mid \gcd(2x+2y+1, 3x+3y+1),$ then $d \mid (3x+3y+1) - (2x+2y+1) = x+y.$
    Which means that $d \mid (2x+2y+1) - 2(x+y) = 1.$
    Therefore $\gcd(2x+2y+1, 3x+3y+1) = 1,$ and since $(2x+2y+1)(3x+3y+1)(x-y)^2 = (xy)^2,$
    hence each of $2x+2y+1$ and $3x+3y+1$ is a perfect square.
\end{soln}

\newpage

\begin{example*}[Example 19]
    Prove that the sum of squares of twelve positive consecutive integers is not divisible by the sum of them.
    In other words, if $n$ non-negative integers, then
    \[
        (n+1) + (n+2) + \cdots + (n+12) \not |\ (n+1)^2 + (n+2)^2 + \cdots + (n+12)^2 
    \]
\end{example*}

\begin{soln}[Solution 1]
    First, note that the sum of any three positive consecutive integers is divisible by $3$,
    thus the sum of twelve positive consecutive integers is divisible by $3$, too.
    On the other hand, the sum of squares of three consecutive integers when divided by $3$ leaves a remainder of $2,$
    the sum of twelve positive consecutive integers when divided by $3$ has a remainder $2.$ 
    The conclusion follows.
\end{soln}

\begin{soln}[Solution 2]
    Note that 
    \[
        \begin{aligned}
            &(n+1) + (n+2) + \cdots + (n+12) = 12n+78\\
            &(n+1)^2 + (n+2)^2 + \cdots + (n+12)^2 = 12n^2+156n+650\\
            &\Rightarrow \frac{12n^2 + 156n + 650}{12n+78} = \left( n+ \frac{13}{2} \right)(12n+78) + 143\\
        \end{aligned}
    \]

    The last expression shows that it cannot be an integer for any non-negative integer $n.$
\end{soln}

\begin{example*}[Example 20]
    Let $N$ be a $16-$digit number, where none of its digits can be $0, 1, 4,$ or $9.$
    Prove that there exist some of its consecutive digits such that their product is a perfect square.
\end{example*}

\begin{soln}
    Let $N = \overline{a_1a_2 \ldots a_{16}}.$ Consider the following sequence of 16 terms:
    \[
        a_1,\ a_1a_2,\ ldots,\ a_1a_2 \ldots a_{16} \quad (*)
    \]

    Note that since each of $a_1, a_2, \ldots, a_{16}$ can only be $2,3,5,6,7,8,$ thus each of the term can only be factored as
    \[
        2^{2\alpha_2}+\beta_2 \cdot 3^{2\alpha_3}+\beta_3 \cdot 5^{2\alpha_5}+\beta_5 \cdot 7^{2\alpha_7}+\beta_7,
    \]
    where $\alpha_i$ ($i=2,3,5,7$) are non-negative integers, $\beta_i$ ($i=2,3,5,7$) are 0 or 1.

    Thus, the (*) sequence has a one-on-one correspondence map to a sequence of four-digit binaries $\beta_2 \beta_3 \beta_5 \beta_7.$
    If there is a $0000$ binary, then it is the term in (*) which is the desired sequence of consecutive digits, whose product is a perfect square.
    If there is no such binary, then among 16 terms, two should have the same four-digit binaries $\beta_2 \beta_3 \beta_5 \beta_7.$
    Let assume that they are $a_1a_2 \ldots a_{i}$ and $a_1a_2 \ldots a_{j},$ where $i<j$, then
    \[
        \frac{a_1a_2 \ldots a_{j}}{a_1a_2 \ldots a_{i}} = a_{i+1}a_{i+2} \ldots a_{j}
    \]
    has a $0000$ binary representation of its prime factorization. This product is a perfect square.

    Note that $16$ is the best possible lower limit. For 15, the following number $232523272325232$ is a counter example.
\end{soln}

\end{document}