\documentclass{article}

\usepackage[main=english,vietnamese]{babel}
\usepackage[T1]{fontenc}
\usepackage[utf8]{inputenc}
\usepackage[sexy]{evan}
\usepackage{matchsticks}
\usepackage{wrapfig}
\usepackage{listings}

\title{Sequence}
\author{Ngo Van Minh}
\date{\today}

\begin{document}

\maketitle

A sequence is an enumerated collection of objects (numbers, letters, functions, etc) in which repetitions are allowed and the order of their apperance matters
\[
    1,\ 2,\ 3,\ 4,\ 5, ...
\]

The members of a sequence are called terms or elements.
The sequence is called a finite sequence, if there are a \textit{finite} amount of terms in a sequence. 
In other words, the sequence ends. Not all sequence ends, such sequence is called \textit{infinite.}
When using variables to represent a sequence,
we often use the same letter with different subscripts to represent the terms.
For example, we might represent a sequence with $5$ terms as 
\[ 
    a_1, a_2, a_3, a_4, a_5.
\]

The expression $\{a_n\}$ denotes the sequence,
where $a_n$ denotes a generic $n^{\text{th}}$ element of the sequence.

Some sequences have obvious patterns, such as

\textbf{The sequence of natural numbers}: $0,\ 1,\ 2,\ 3,\ 4, \ldots$

\textbf{The sequence of odd numbers}: $1,\ 3,\ 5,\ 7,\ 9, \ldots$

\textbf{The sequence of prime numbers}: $1,\ 3,\ 5,\ 7,\ 9, \ldots$

\textbf{The sequence of square numbers}: $1,\ 4,\ 9,\ 16,\ 25,\ 36, \ldots$ The $i^{\text{th}}$ is $a_i = i^2.$

\textbf{The sequence of cubic numbers}: $1,\ 8,\ 27,\ 64,\ 125, \ldots$ The $i^{\text{th}}$ is $a_i = i^3.$

\textbf{Fibonacci sequence}: $1,\ 1,\ 2,\ 3,\ 5,\ 8,\ 13,\ 21, \ldots$ The sequence starts with two terms $1$ and $1,$
then each following term is equal to the sum of two previous terms.

\textbf{Arithmetic sequence}: $\{a_i\},$ where $a_i = a_1 + (i-1) \times d.$
$d$ is the common difference between two consecutive terms.
\textit{For example, $2,\ 5,\ 8,\ 11,\ 14, \ldots$ is an arithmetic sequence with $a_1 = 2$ and $d=3.$}

\textbf{Geometric sequence}: $\{a_i\},$ where $a_i = a_{i-1} \times r.$
$r$ is the common ratio between two consecutive terms.
It is easy to see that $a_i = a_1 \times r^{i-1}.$

\textbf{Sequence as a combination of twin numbers sequences}: a combination of two sequences $a_1,\ a_2,\ a_3, \ldots,$
and $b_1,\ b_2,\ b_3, \ldots$ results in $a_1,\ b_1,\ a_2,\ b_2,\ a_3,\ b_3, \ldots.$

\textbf{Mixed sequence}: a sequence without any particular rule or pattern.

\textbf{Sequence of words}: a sequence of words that has some logics based on their meanings.

\newpage

\begin{problem*} \textbf{What is the missing number?}
    \[
        \_,\ 1,\ 2,\ 6,\ 24,\ 120,\ 720.
    \]
\end{problem*}

\begin{soln}
    By observation
    \[
        \begin{aligned}
            &1 = 1!,\ 2 = 1 \times 2 = 2!,\ 6 = 1 \times 2 \times 3 = 3!, 24 = 1 \times 2 \times 3 \times 4 = 4!,\\
            &120 = 1 \times 2 \times 3 \times 4 \times 5 = 5!, 720 = 1 \times 2 \times 3 \times 4 \times 5 \times 6 = 6!, 
        \end{aligned}
    \]
    
    Thus, the missing term is $0!.$ The answer is $\boxed{1.}$
\end{soln}

\begin{problem*} \textbf{What is the missing letter?}
    \[
        \_,\ \text{O},\ \text{T},\ \text{T},\ \text{F},\ \text{F},\ \text{S},\ \text{S},\ \text{E},\ \text{N},\ \text{T}.
    \]
\end{problem*}

\begin{soln}
    By observation, from the second term, each letter is the first letter of each word representing the first ten postitive integers
    \[
        \underline{\text{O}}\text{ne},\ \underline{\text{T}}\text{wo},\ \underline{\text{T}}\text{hree},\ \underline{\text{F}}\text{our},\
        \underline{\text{F}}\text{ive},\ \underline{\text{S}}\text{ix},\ \underline{\text{S}}\text{even},\ \underline{\text{E}}\text{ight},\
        \underline{\text{N}}\text{ine},\ \underline{\text{T}}\text{en.}
    \]
    
    Thus, the missing term is Zero. The answer is \boxed{Z.}
\end{soln}

\begin{problem*} \textbf{Find the next term of the sequence}
    \[
        13,\ 57,\ 91,\ 11,\ 31,\ 51,\ \_
    \]
\end{problem*}

\begin{soln}
    By placing the terms one after another, we obtain
    \[
        \underline{1}\ \underline{3}\ \underline{5}\ \underline{7}\ \underline{9}\ \underline{11}\ \underline{13}\ \underline{15}\ 1\ldots
    \]
    
    By observation, it is a sequence of odd numbers.
    Thus, the missing term is $71$. The answer is $\boxed{71.}$
\end{soln}

\begin{problem*} \textbf{What is the missing letter?}
    \[
        \text{B},\ \text{C},\ \text{E},\ \text{G},\ \text{K},\ \text{M}, \_
    \]
\end{problem*}

\begin{soln}
    The letters in the alphabet form a sequence with their positions in the sequence shown as below
    \begin{figure}[h]
        \centering
        \begin{minipage}[t]{6.5cm}
            \begin{tabular}{c|c|c|c|c}
                \text{A} & \text{B} & \text{C} & $\ldots$ & \text{Z} \\ \hline
                $1$ & $2$ & $3$ & $\ldots$ & $26$
            \end{tabular}
        \end{minipage}
    \end{figure}

    Now, by looking at the letters in the given sequence with their position in the alphabet
    \begin{figure}[h]
        \centering
        \begin{minipage}[t]{6.5cm}
            \begin{tabular}{c|c|c|c|c|c}
                \text{B} & \text{C} & \text{E} & \text{G} & \text{K} & \text{M} \\ \hline
                $2$ & $3$ & $5$ & $7$ & $11$ & $13$
            \end{tabular}
        \end{minipage}
    \end{figure}

    It is easy to see that the sequence of their positions in the alphabet is a sequence of consecutive prime numbers, starting from $2.$
    Thus, the missing letter is the one at the position $17$ in the alphabet, which is Q. The answer is \boxed{Q.}
\end{soln}

\begin{problem*} \textbf{Find the next term of the sequence}
    \[
        311,\ 220,\ 233,\ 112,\ 202,\ 331,\ \_
    \]
\end{problem*}

\begin{soln}
    By placing the terms one after another, we obtain
    \[
        \underline{31122023}\ \underline{31122023}\ \underline{31}\ldots
    \]
    
    The pattern $31122023$ (December, 31, 2023) seems to repeat. 
    Thus, by this observation, the missing term is $122$. The answer is $\boxed{122.}$
\end{soln}

\begin{problem*} \textbf{Find the next term of the sequence}
    \[
        1,\ 1,\ 1,\ 3,\ 2,\ 5,\ 3,\ 7,\ 5,\ 9,\ 8,\ 11,\ 13,\ 13,\ \_
    \]
\end{problem*}

\begin{soln}
    By observation, it can be seen that the sequence is a combination of twin numbers sequences
    \[
        \begin{aligned}
            &1,\ 1,\ 2,\ 3,\ 5,\ 8,\ 13, \ldots\\
            &1,\ 3,\ 5,\ 7,\ 9,\ 11,\ 13, \ldots
        \end{aligned}
    \]
    
    The first sequence is the Fibonacci sequence and the second one is the sequence of consecutive odd numbers.
    Thus, by this observation, the missing term is the term after $13$ in the Fibonacci sequence, which is $8+13=21$.
    The answer is $\boxed{21.}$
\end{soln}

\begin{problem*} \textbf{Find the missing term of the sequence}
    \[
        365824,\ \_ ,\ 85636,\ 5617,\ 658,\ 613,\ 64
    \]
\end{problem*}

\begin{soln}
    By observation, the first term in the sequence can be split at the tens digits the into two parts,
    \[
        3658 \mid 24
    \]

    By reversing the digits of the first part and adding the digits of the second part, we obtain
    \[
        8563 \mid 6 \Longrightarrow 85636
    \]

    This is the second term. Using this observation for the terms we have
    \[
        \begin{aligned}
            &5617 \rightarrow 56 \mid 17 \rightarrow 65 \mid 8 \Longrightarrow 658\\
            &658 \rightarrow 6 \mid 58 \rightarrow 6 \mid 13 \Longrightarrow 613\\
            &613 \rightarrow 6 \mid 13 \rightarrow 6 \mid 4 \Longrightarrow 64
        \end{aligned}
    \]

    So the discovered rule seems to hold. Now, by applying this to the second term
    \[
        85636 \rightarrow 856 \mid 36 \rightarrow 658 \mid 9 \Longrightarrow 6589
    \]
    
    Thus, the missing term is $6589.$ The answer is $\boxed{6589.}$
\end{soln}

\begin{problem*} \textbf{Find the next term of the sequence}
    \[
        5,\ 6,\ 6,\ 7,\ 8,\ 10,\ 13,\ 18,\ \_
    \]
\end{problem*}

\begin{soln}[First solution]
    By observation, from the second term, by subtracting each term from the previous one, we obtain
    \[
        1,\ 0,\ 1,\ 1,\ 2,\ 3,\ 5,\ \ldots
    \]
    
    This is a Fibonacci sequence. The next term in this sequence is $8.$ 
    Thus missing term in the original sequence is $18+8=26.$
    The answer is $\boxed{26.}$
\end{soln}

\begin{soln}[Second solution]
    By observation, by subtracting $5$ from each term, we obtain
    \[
        0,\ 1,\ 1,\ 2,\ 3,\ 5,\ 8,\ 13,\ \ldots
    \]
    
    This is a Fibonacci sequence. The next term in this sequence is $8+13=21.$ 
    Thus missing term in the original sequence is $21+5=26.$
    The answer is $\boxed{26.}$
\end{soln}

Now, lets try to complete some sequences.

\begin{exercise*}
    Find the missing term in each of the sequence below
    \[
        \begin{aligned}
            &1,\ 2,\ 4,\ 7,\ 12,\ \_\\
            &0,\ 2,\ 6,\ 12,\ 22,\ 38,\ \_\\
            &0,\ 1,\ 10,\ 11,\ 100,\ 101,\ 110,\ 111,\ \_\\
            &6,\ 12,\ 15,\ 21,\ 24,\ 30,\ 33,\ 39,\ \_\\
            &1,\ 11,\ 21,\ 1211,\ 111221,\ 312211,\ \_
        \end{aligned}
    \]
\end{exercise*}

\end{document}