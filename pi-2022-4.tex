\documentclass{article}

\usepackage[main=english,vietnamese]{babel}
\usepackage[T1]{fontenc}
\usepackage[utf8]{inputenc}
\usepackage[sexy]{evan}
\usepackage{matchsticks}
\usepackage{wrapfig}
\usepackage{listings}

\title{The Power of Mathematical Reasoning - Part One}
\author{Nghia Doan \& Catherine Doan}
\date{\today}

\begin{document}

\maketitle

In this article, we discuss mathematical reasoning by a number of examples.

\section*{It is so obvious, $\ldots$ or is it?}

Let's start with a few classic examples.
These problems are simple, but don't let them fool you.

\begin{example*}[Who lied?]
    \label{example:pi-2022-4-p1}
    Two American Indians were sitting on a log - a big Indian and a little Indian.
    The big Indian was smoking and the little Indian played with a toy.
    The little Indian was the son of the big Indian,
    but the big Indian was not the father of the little Indian.

    How is it possible?
\end{example*}

\begin{soln}[\nameref{example:pi-2022-4-p1}]
    The fact that the big Indian was smoking does not mean that the big Indian was a man.
    Now, the little Indian was the son of the big Indian, and if the big Indian was not his father
    then the big Indian was \framebox{his mother}.
\end{soln}

\begin{example*}[Whose picture I am looking at?]
    \label{example:pi-2022-4-p2}
    A man was looking at a portrait. Someone asked him "Whose picture are you looking at?"
    He replied: "Brothers and sisters have I none, but this man's father is my father's son."
    (\textit{This man's father means, of course, the father of the man in the picture.})

    Whose picture was the man looking at?
\end{example*}

\begin{soln}[\nameref{example:pi-2022-4-p2}]
    The man made a statement that contains two parts, each a seperate statement.
    The first one means that he had neither brothers nor sisters.
    The second one means that the father in his sentence is a son of his father, meaning one of his brothers or himself.
    Because he has no brother, therefore the father is himself.
    Thus, he is the father of the man in the paintng and the man in the painting is \framebox{his son.}
\end{soln}

\begin{example*}[How could she set the clock?]
    \label{example:pi-2022-4-p3}
    The clock in Lilian's house has stopped. Lilian wanted to set the correct time,
    but she was unable to do that, since there is the only one clock in the house.
    That evening, she went to visit her friend, Bianca,
    who has an excellent wall clock that always shows correct time.
    When Lilian came home after the visit, she was able to set the correct time on her clock.

    How did she do it?
\end{example*}

\begin{soln} \nameref{example:pi-2022-4-p3}
    Lilian checked the time displayed on her clock twice: \textit{when she left the house} and \textit{when she came back.}
    The difference of these two will be equal to the total time when she was away,
    which is \textit{twice the length of the trip in time} between the two houses.
    By dividing by two, she had \textit{the time needed for a trip}. 

    Now, lets assume that she walked to Bianca house, took a look at the clock and promptly left.
    This way, Lilian knew \textit{what time it was she left Bianca's house}.
    By adding the time needed for a trip to that she knew what time it was and then set the clock.
\end{soln}

\begin{example*}[Defeat King Haggard]
    \label{example:pi-2022-4-p4}
    There are ten wells in the far, far away Dark Forest.
    The wells are marked by signs with number from 1 to 10.
    The water from these wells look and taste like normal water;
    however, they were all poisoned.
    The only known antidote is water from a higher-numbered well.
    (\textit{For example, if you had a sip of water from the well number 6,
    you could save yourself by drinking from the well 7, 8, 9, or 10.})

    Unfortunately, there is no antidote for the water from the $10^{\text{th}}$ well.
    While the first nine wells can be easily visited,
    the $10^{\text{th}}$ well is inside the castle of King Haggard, the ruler of the Dark Forest.
    
    Prince Martin challenged the King to a duel:
    \textit{Each participant should drink a cup of water offered by the opponent.}
    Of course, King Haggard agreed. His plan was to use his $10^{\text{th}}$ well;
    because the water from this well help him survive any drink offered by Prince Martin,
    and it will definitely kill the prince.

    The duel took place as planned. Each participant drank the cup of water offered by the opponent.
    Everybody surprised and delighted to see that Prince Martin lived and King Haggard died.
    
    How did this happen?
\end{example*}

\begin{soln} \nameref{example:pi-2022-4-p4}
    Prince Martin \textit{gave King Haggard a drink of clear water.}
    Thus instead of curing him, the water from the $10^{\text{th}}$ well he drank then poisoned him.
    Prince Martin, before the duel, drank some water from any of the wells with a number smaller than $10.$
    Therefore \textit{Haggard's cup of water from the $10^{\text{th}}$ well actually cured Prince Martin.}
\end{soln}

\section*{If this is true then $\ldots$, if this is false then $\ldots$}

\begin{example*}[Who is younger?]
    \label{example:pi-2022-4-p7}
    A brother and a sister were once asked who was younger. "I am older," said the brother.
    "I am younger," said the sister.

    Now, if you knew that \textit{at least one of them lied,}
    then who was younger?
\end{example*}

There are two different approaches to this problem.

\begin{soln}[\nameref{example:pi-2022-4-p7}]
    The first one is to \textit{evaluate all possible truths} and see which one matches the given conditions.
    There are two cases: the brother was younger, or the sister was younger.  
    
    If the brother was younger, then the brother lied, and so did the sister.
    If the sister was younger, then both of them told the truth, which would not be possible.

    Thus, \framebox{the brother was younger, and they both lied.}
\end{soln}

\begin{soln}[\nameref{example:pi-2022-4-p7}]
    Now, the second approach is to \textit{evaluate all possible assumptions},
    in this case whether the brother (and then the sister) told the truth.
    
    If the brother told the truth, then the sister did too, then
    this contradicts the condition that at least one of them lied.
    If the brother lied, then the sister lied too, so
    this satisfies the condition, and therefore it is the correct one.

    Thus, \framebox{the brother was younger, and they both lied.}
\end{soln}

\section*{Exercises}

\begin{exercise*}[Which one is correct?]
    \label{exercise:pi-2022-4-p5}
    Which phrase is correct: "the yolk \textit{is} white" or "the yolk \textit{are} white."?
\end{exercise*}

\begin{soln} \nameref{exercise:pi-2022-4-p5}
    Actually, the yolk is \textit{yellow.}
\end{soln}

\begin{exercise*}[How can they see each other?]
    \label{exercise:pi-2022-4-p6}
    Two camels were facing in the opposite direction.
    One was facing due east and the other one was facing due west.
    How can they manage to see each other, without walking, turning around, or even moving their heads?
\end{exercise*}

\begin{soln} \nameref{exercise:pi-2022-4-p6}
    The two camels were \textit{facing each other} the whole time, hence facing in opposite directions.
\end{soln}

\end{document}