\documentclass{article}

\usepackage[main=english,vietnamese]{babel}
\usepackage[T1]{fontenc}
\usepackage[utf8]{inputenc}
\usepackage[sexy]{evan}
\usepackage{matchsticks}
\usepackage{wrapfig}
\usepackage{listings}

\newtheorem{hint}{Hint}

\title{Paths on grids}
\author{Nghia Doan}
\date{\today}

\begin{document}

\maketitle

In this article, some problems of paths on boards are discussed.
They highlight the relations of cells in same column or row.

\begin{example*}[Who is the taller one?]

    One hundred students are positioned in $10 \times 10$ grids,
    each of the rows and columns contains exactly $10$ students.
    From each of the $10$ columns the \textit{shortest} student is selected,
    and the \textit{tallest} of these $10$ students is tagged as $T$.
    Next the \textit{tallest} student from each rows is selected,
    and from these $10$ students the \textit{shortest} is tagged as $S$.
    Which of the two tagged students is the taller if they are two different people?
\end{example*}

\begin{soln}
    If $S$ and $T$ are in the same column, then $T$ is among the shortests of each column,
    so $T$ is the shortest in its own column, thus $T$ is shorter than $S$.
    If $S$ and $T$ are in the same row, then $S$ is among the tallests of each column,
    so it is the tallest in its own row, thus $T$ is shorter than $S$.

    \begin{figure}[h]
        \centering
        \begin{tabular}{|c|c|c|c|c|}
            \hline
            {\color[HTML]{FFFFFF} X} &                                                  &                          &                                                  &                          \\ \hline
                                    &                                                  &                          & \cellcolor[HTML]{FFFFFF}{\color[HTML]{3531FF} T} &                          \\ \hline
            {\color[HTML]{FFFFFF} X} &                                                  &                          &                                                  &                          \\ \hline
                                    & \cellcolor[HTML]{FFFFFF}{\color[HTML]{FE0000} S} &                          & \cellcolor[HTML]{EFEFEF}I                        &                          \\ \hline
            {\color[HTML]{FFFFFF} X} &                                                  & {\color[HTML]{FFFFFF} X} &                                                  & {\color[HTML]{FFFFFF} X} \\ \hline
        \end{tabular}
    \end{figure}

    If $S$ and $T$ are in different rows and colums,
    then there exists $I$ at the intersection of the row of $S$ and the column of $T$.
    By definition, $T$ is shorter than $I$ and $I$ is shorter than $S$,
    thus $T$ is shorter than $S$.

    Therefore, $\boxed{S}$ is the taller one.
\end{soln}

\newpage

\begin{example*}[How many ways to form the name?]

    Thanh filled a triangle of squares with the letters of her name, as shown below.
    She counted all the $5-$letter paths that form her name T-H-A-N-H,
    each starts from the T letter in the middle of the bottom row,
    then goes left, right, or up. An example is also shown in the diagram.
    What number did she get?
\end{example*}

\begin{figure}[h]
    \centering
    \begin{tabular}{ccccccccc}
        &   &   &   & H                                                &                                                  &                                                  &   &   \\
        &   &   & H & N                                                & H                                                &                                                  &   &   \\
        &   & H & N & A                                                & \cellcolor[HTML]{3166FF}{\color[HTML]{FFFFFF}  N } & \cellcolor[HTML]{3166FF}{\color[HTML]{FFFFFF}  H } &   &   \\
        & H & N & A & \cellcolor[HTML]{3166FF}{\color[HTML]{FFFFFF}  H } & \cellcolor[HTML]{3166FF}{\color[HTML]{FFFFFF}  A } & N                                                & H &   \\
        H & N & A & H & \cellcolor[HTML]{3166FF}{\color[HTML]{FFFFFF}  T } & H                                                & A                                                & N &  H 
    \end{tabular}  
\end{figure}

\begin{soln}
    Consider the \textit{"half"} triangle shown in the diagram below.
    It is easy to see that in each path, there is two choices at each steps.
    One example is shown in the figure, when two As can be chosen after a $H$.

    \begin{figure}[h]
        \centering
        \begin{tabular}{ccccl}
            &   &                                                  &                                                  &  H                                                 \\
            &   &                                                  &  H                                                 &  N                                                 \\
            &   &  H                                                 &  N                                                 &  A                                                 \\
            &  H  &  N                                                 & \cellcolor[HTML]{FD6864}{\color[HTML]{FFFFFF}  A } &  H                                                 \\
            H  &  N  & \cellcolor[HTML]{FD6864}{\color[HTML]{FFFFFF}  A } & \cellcolor[HTML]{3166FF}{\color[HTML]{FFFFFF}  H } & \cellcolor[HTML]{3166FF}{\color[HTML]{FFFFFF}  T }
        \end{tabular}
    \end{figure}

    Therefore there are $2^4=16$ paths for a half triangle.
    In total there are $2\cdot 16-1=31,$
    because the vertical path formed by all the squares in the $T$ column
    is shared by both half triangles.

    Thus, the number that Thanh got is \framebox{$31.$}
\end{soln}

\newpage

\begin{example*}[What would be the largest number?]

    In each square of the $n \times n$, $n \ge 4$ chessboard,
    Antoine writes a number according to the following rules:
    \begin{enumerate}[label=\roman*.,topsep=0pt, partopsep=0pt, itemsep=0pt]
        \ii The number in each square is one of $1,2,\ldots,n^2$.
        \ii The two numbers, in two neighbouring squares that share the same side, differ less than 3.
        \ii The number $3$ is written in one of the square.
    \end{enumerate}
    
    What is the largest number could Antoine writes on the board?
\end{example*}

\begin{soln}
    Let $m$ be the maximal number that can be written on the board.

    \begin{figure}[h]
        \centering
        \begin{tabular}{|
            >{\columncolor[HTML]{FFFFFF}}c |
            >{\columncolor[HTML]{FFFFFF}}c |
            >{\columncolor[HTML]{FFFFFF}}c |
            >{\columncolor[HTML]{96FFFB}}c |
            >{\columncolor[HTML]{FFFFFF}}c |}
            \hline
            {\color[HTML]{FFFFFF} X} &                                                           &                          &                                   &                          \\ \hline
            \cellcolor[HTML]{96FFFB} & \cellcolor[HTML]{96FFFB}{\color[HTML]{3531FF} \textbf{3}} & \cellcolor[HTML]{96FFFB} & {\color[HTML]{3531FF} \textbf{i}} & \cellcolor[HTML]{96FFFB} \\ \hline
            {\color[HTML]{FFFFFF} X} &                                                           &                          &                                   &                          \\ \hline
                                    &                                                           &                          & {\color[HTML]{3531FF} \textbf{m}} &                          \\ \hline
            {\color[HTML]{FFFFFF} X} &                                                           & {\color[HTML]{FFFFFF} X} &                                   & {\color[HTML]{FFFFFF} X} \\ \hline
        \end{tabular}
    \end{figure}
    
    In the figure above, there are atmost $n-1$ squares on the row of $3$,
    not including the square containing $3$, 
    from the number $3$ to the intersection $i$ with the column of $m$,
    Similarly there are atmost $n-1$ squares on the column of $m$,
    not including the square containing $m$, 
    from the intersection $i$ with the column of $m$ to the number $m$.
    Thus the difference between $m$ and $3$ can not be more than
    $2\cdot (n-1)+ 2\cdot (n-1)=4(n-1)$.
    
    Therefore, the largest value of $m$ can be $3+4(n-1)=\boxed{4n-1.}$
\end{soln}

\newpage

\begin{example*}[Colouring the paths]

    In the $4 \times 4$ grid below each cell at row $i$ and column $j$ contains the value equal to $i \times j.$
    A \textit{path} in the grid is a sequence of squares,
    such that consecutive squares share an edge and no square occurs twice in the sequence.
    Furthermore, the \textit{score} of a path is the sum of the point values of all squares in the path.

    Determine the highest possible score of a path that begins with the bottom left corner of the grid 
    (where the number $1$ stands) and ends with the top right corner (where the number $16$ stands.)
\end{example*}

\begin{figure}[h]
    \centering
    \begin{minipage}[t]{6.5cm}
        \centering
        \begin{tabular}{|c|c|c|c|}
            \hline
            \cellcolor[HTML]{EFEFEF}4 & 8                         & \cellcolor[HTML]{EFEFEF}12 & 16                         \\ \hline
            3                         & \cellcolor[HTML]{EFEFEF}6 & 9                          & \cellcolor[HTML]{EFEFEF}12 \\ \hline
            \cellcolor[HTML]{EFEFEF}2 & 4                         & \cellcolor[HTML]{EFEFEF}6  & 8                          \\ \hline
            1                         & \cellcolor[HTML]{EFEFEF}2 & 3                          & \cellcolor[HTML]{EFEFEF}4  \\ \hline
        \end{tabular}
    \end{minipage}
\end{figure}

\begin{soln}
    Let us colour the squares of the grid with a chessboard pattern of black and white,
    starting with black in the bottom left-hand corner, as show below on the left diagram.
    A path in the grid will alternate between black and white squares,
    beginning and ending on black.
    Thus, any path cannot contain all $8$ white squares.

    \begin{figure}[h]
        \centering
        \begin{minipage}[t]{7cm}
            \centering
            \begin{tabular}{|c|c|c|c|}
                \hline
                4                         & \cellcolor[HTML]{C0C0C0}8 & 12                        & \cellcolor[HTML]{C0C0C0}16 \\ \hline
                \cellcolor[HTML]{C0C0C0}3 & 6                         & \cellcolor[HTML]{C0C0C0}9 & 12                         \\ \hline
                2                         & \cellcolor[HTML]{C0C0C0}4 & 6                         & \cellcolor[HTML]{C0C0C0}8  \\ \hline
                \cellcolor[HTML]{C0C0C0}1 & 2                         & \cellcolor[HTML]{C0C0C0}3 & 4                          \\ \hline
            \end{tabular}
            \caption{Alternate colouring}
        \end{minipage}
        \quad
        \begin{minipage}[t]{6cm}
            \centering
            \begin{tabular}{|cccc|}
                \hline
                4                       & 8                      & 12                     & 16 \\ \cline{2-4} 
                3                       & 6                      & 9                      & 12 \\ \cline{1-3}
                2                       & 4                      & \multicolumn{1}{c|}{6} & 8  \\ \cline{2-2}
                \multicolumn{1}{|c|}{1} & \multicolumn{1}{c|}{2} & 3                      & 4  \\ \hline
                \end{tabular}
            \caption{Path with maximal value}
        \end{minipage}     
    \end{figure}

    The path from $1$ to $16$ with highest possible score should contains all squares
    except for the square containing $2$ at the bottom row, as show above in the right diagram.
    The sum of the values for all the squares in this path is 
    \[ 
        (1+2+3+4)+(2+4+6+8)+(3+6+9+12)+(4+8+12+16) = (1+2+3+4)(1+2+3+4) = 10^2 = 100,
    \]
    
    Thus, the score of the path is $100-2=\boxed{98}.$
\end{soln}

\end{document}