\documentclass{article}

\usepackage[main=english,vietnamese]{babel}
\usepackage[T1]{fontenc}
\usepackage[utf8]{inputenc}
\usepackage[sexy]{evan}
\usepackage{matchsticks}
\usepackage{wrapfig}
\usepackage{listings}

\begin{otherlanguage*}{vietnamese}
\title{Các Kỳ Thi Toán Học Tại Hungary\\ \quad \\Phần 1 - Các kỳ thi toán học cấp thành phố/tỉnh}
\end{otherlanguage*}

\author{Nghia Doan}
\date{\today}

\begin{document}

\begin{otherlanguage*}{vietnamese}

\maketitle

\section{Giới thiệu chung về các kỳ thi toán Hungary}

Toán học luôn đóng vai trò quan trọng trong hệ thống giáo dục Hungary. Quốc gia này có một truyền thống toán học lâu đời và đã đào tạo nên nhiều nhà toán học xuất sắc.
Các kỳ thi toán học tại Hungary không chỉ giúp phát hiện và nuôi dưỡng tài năng mà còn đóng vai trò là cánh cửa để học sinh tham gia các đấu trường quốc tế
như Olympic Toán học Quốc tế (IMO), Olympiad Toán học Nữ Châu Âu (European Girls' Mathematical Olympiad - EGMO \cite{EGMO}),
Olympiad Toán học Trung Âu (Middle European Mathematical Olympiad - MEMO \cite{MEMO}),
và Kỳ thi Toán học Quốc tế dành cho học sinh nói tiếng Hungary (Nemzetközi Magyar Matematikaverseny - NMMV \cite{NMMV}).

Hệ thống các kỳ thi toán học tại Hungary vô cùng phong phú và đa dạng, đến mức \textbf{học sinh gần như có thể tham gia một cuộc thi mỗi tuần}.
Từ các kỳ thi cấp địa phương của Quỹ Vì Trẻ Em Tài Năng Toán Học (Matematikában Tehetséges Gyermekekért - MATEGYE \cite{MATEGYE}),
Tạp chí Toán học Trung học Hungary (Középiskolai Matematikai Lapok - KöMaL \cite{KoMaL}), và Kỳ thi Toán học Arany Dániel (Arany Dániel Matematikai Tanulóverseny \cite{AranyDaniel})
đến các kỳ thi cấp quốc gia như Kỳ thi Học Sinh Giỏi Toán Quốc gia Hungary (Országos Középiskolai Tanulmányi Verseny - OKTV \cite{OKTV}) 
và Kỳ thi Toán học Kürschák József (Kürschák József Matematikai Tanulóverseny \cite{Kurschak}),
mỗi kỳ thi đều có lịch trình chặt chẽ và độ khó khác nhau, giúp học sinh có cơ hội rèn luyện liên tục

Ngoài ra, các kỳ thi quốc tế như MEMO, NMMV và IMO cũng tạo ra những sân chơi mang tầm vóc lớn hơn,
giúp học sinh Hungary có thể thử sức và phát triển kỹ năng toán học trong suốt cả năm.
Với sự dày đặc của các cuộc thi, học sinh không chỉ trau dồi kiến thức mà còn rèn luyện tư duy sáng tạo, khả năng giải quyết vấn đề và bản lĩnh thi đấu,
để có cơ hội đạt vị thế cao trên đấu trường quốc tế.

Bài viết này sẽ giới thiệu hệ thống các kỳ thi toán học tại Hungary theo ba cấp độ: các kỳ thi cấp thành phố/tỉnh, các kỳ thi toán cấp quốc gia quan trọng
và những kỳ thi liên quan trực tiếp đến quá trình tuyển chọn đội tuyển tham dự IMO, EGMO, MEMO.

\newpage

\section{Các kỳ thi toán học cấp thành phố/tỉnh}

\subsection{Kỳ thi Toán cấp tỉnh MATEGYE}

Kỳ thi Toán cấp tỉnh do Quỹ MATEGYE tổ chức là một kỳ thi hai vòng dành cho học sinh từ lớp 3 đến lớp 12 tại nhiều tỉnh của Hungary.
Mục tiêu của kỳ thi là tạo điều kiện cho học sinh thử sức, phát triển tư duy toán học và so tài cùng bạn bè đồng trang lứa.
Kỳ thi diễn ra tại 14 tỉnh, với mỗi tỉnh có một đơn vị đồng tổ chức để hỗ trợ việc tổ chức thi.

Kỳ thi mở rộng cho tất cả học sinh thuộc các trường học trong tỉnh đăng ký, không giới hạn số lượng thí sinh từ mỗi trường.
Kỳ thi được chia thành các hạng mục theo cấp lớp: học sinh lớp 3 và 4 thi chung một hạng mục,
học sinh lớp 5 đến lớp 8 được chia thành hai hạng mục dành riêng cho trường tiểu học và trung học (gimnázium),
trong khi học sinh lớp 9 đến lớp 12 thi theo hai hạng mục riêng biệt dành cho trung học phổ thông (gimnázium) và trung học kỹ thuật (technikum).
Đối với học sinh theo học chương trình trung học kéo dài năm năm, các em sẽ thi theo cấp lớp tương ứng với nội dung toán học mình đang học.

\textbf{Cấu trúc và quy trình thi}

Kỳ thi bao gồm hai vòng thi với độ khó tăng dần.
\begin{enumerate}[topsep=0pt, partopsep=0pt, itemsep=0pt]
    \ii Vòng đầu tiên diễn ra thường vào đầu tháng 12 tại chính trường học của thí sinh. Đề thi và đáp án do MATEGYE cung cấp,
    và bài thi sẽ được giáo viên của trường chấm theo hướng dẫn chính thức. Kết quả vòng này sẽ quyết định học sinh nào được mời tham gia vòng hai.
    \ii Vòng hai diễn ra thường vào đầu tháng 2 năm sau tại các trường do đơn vị đồng tổ chức của từng tỉnh lựa chọn.
    Ở vòng này, bài làm sẽ được chấm bởi hội đồng giám khảo chuyên môn do MATEGYE chỉ định.
\end{enumerate}

Mỗi vòng thi bao gồm 5 bài toán tự luận, yêu cầu học sinh trình bày lời giải chi tiết và lập luận chặt chẽ. Nội dung đề thi được xây dựng phù hợp với từng cấp lớp.
\begin{itemize}[topsep=0pt, partopsep=0pt, itemsep=0pt]
    \ii Đối với lớp 3 và 4, các bài toán tập trung vào số học cơ bản, bài toán đố, quy luật số học và hình học trực quan.
    \ii Học sinh lớp 5 đến lớp 8 làm các bài toán về số học, đại số sơ cấp, hình học phẳng, tổ hợp và tư duy logic.
    \ii Ở cấp trung học từ lớp 9 đến lớp 12, đề thi có độ khó cao hơn, gồm số học, đại số nâng cao (phương trình, bất đẳng thức), hình học không gian, tổ hợp và xác suất.
\end{itemize}

Thời gian làm bài của thí sinh được quy định theo cấp lớp: học sinh lớp 3 và 4 có 60 phút,
học sinh lớp 5 đến lớp 8 có 90 phút, và học sinh lớp 9 đến lớp 12 có 120 phút để hoàn thành bài thi.
Trong quá trình làm bài, học sinh lớp 3 đến lớp 8 chỉ được sử dụng bút, thước kẻ, compa và thước đo góc,
trong khi học sinh lớp 9 đến lớp 12 được phép dùng bảng công thức và máy tính bỏ túi.

Học sinh đăng ký trực tuyến thường vào đầu tháng 10 tại \url{http://www.mategye.hu}. Lệ phí tham gia là 1500 Ft (100.000 VND)/học sinh,
được thanh toán dựa trên hóa đơn do MATEGYE cung cấp sau khi hoàn tất đăng ký.

\textbf{Chấm điểm và giải thưởng}

Kết quả kỳ thi sẽ được công bố thường vào đầu tháng 3 trong một buổi lễ trao giải do đơn vị tổ chức tại từng tỉnh quyết định.
Học sinh được xếp hạng dựa trên điểm số đạt được trong vòng hai.

Giải thưởng được phân bổ dựa trên số lượng thí sinh tham gia. Tối đa ba thí sinh xuất sắc nhất trong mỗi hạng mục sẽ nhận cúp,
trong khi các thí sinh đạt thành tích cao sẽ nhận giấy chứng nhận và phần thưởng.
Ở cấp lớp 3 và 4, cũng như lớp 5 và 6, tối đa ba thí sinh xuất sắc nhất sẽ được vinh danh.
Đối với học sinh lớp 7 và 8 (gimnázium) và cấp kỹ thuật, số lượng thí sinh đạt giải có thể lên đến sáu người,
trong khi với học sinh lớp 9 đến lớp 12 (gimnázium), số lượng thí sinh đạt giải có thể lên đến mười người.

Những giáo viên có học sinh đạt thành tích cao nhất cũng sẽ được vinh danh.
Đặc biệt, các học sinh xuất sắc trong hạng mục tiểu học và trung học
có thể được lựa chọn tham gia Kỳ thi Toán học Quốc tế dành cho học sinh nói tiếng Hungary (NMMV) nếu kỳ thi này được tổ chức.

\textbf{Ý nghĩa kỳ thi}

Kỳ thi Toán cấp tỉnh không chỉ là một sân chơi toán học cho tất cả các học sinh ở mọi nơi mà còn là cơ hội để học sinh thử thách bản thân, phát triển tư duy và chuẩn bị cho các kỳ thi toán chuyên sâu hơn.
Với hệ thống tổ chức chặt chẽ và quy trình đánh giá minh bạch, kỳ thi này đóng vai trò quan trọng trong việc thúc đẩy giáo dục toán học tại Hungary.

\bigbreak

\noindent\rule{16.5cm}{0.4pt}

\textbf{Một số bài thi tiêu biểu}

\bigbreak

\begin{problem*}[2016-2017, vòng 2, lớp 8, bài 5 \cite{2016-2017-2f-8o-p5}]
    Một chiếc đồng hồ có kim chỉ giờ và kim phút vuông góc với nhau hai lần trong khoảng thời gian từ 5 giờ đến 6 giờ.
    Hỏi có bao nhiêu phút trôi qua từ lần đầu tiên kim đồng hồ ở vị trí vuông góc đến lần thứ hai?
\end{problem*}

\begin{problem*}[2016-2017, vòng 2, lớp 9, bài 6 \cite{2016-2017-2f-9o-p6}]
    Hai số nguyên dương có tổng, khi cộng thêm hiệu, tích và thương của chúng, ta được kết quả là 32. Hỏi hai số đó là bao nhiêu?
\end{problem*}

\begin{problem*}[2016-2017, vòng 2, lớp 10, bài 6 \cite{2016-2017-2f-10s-p6}]
    Một tam giác vuông có hai cạnh góc vuông dài 6 và 8 đơn vị.
    \begin{enumerate}[topsep=0pt, partopsep=0pt, itemsep=0pt]
        \ii Bán kính của đường tròn nội tiếp là bao nhiêu đơn vị?
        \ii Bán kính của đường tròn tiếp xúc với cạnh huyền và hai đường thẳng chứa hai cạnh góc vuông là bao nhiêu đơn vị?
    \end{enumerate}
\end{problem*}

\begin{problem*}[2013-2014, vòng 2, lớp 11, bài 6 \cite{2013-2014-2f-11o-p5}]
    Giải phương trình: 
    \[
        \sqrt{\left(17+12\sqrt{2}\right)^x} + \sqrt{\left(17-12\sqrt{2}\right)^x} = \frac{10}{3}.
    \]
\end{problem*}

\begin{problem*}[2013-2014, vòng 2, lớp 12, bài 6 \cite{2013-2014-2f-12s-p6}]
    Tìm giá trị của biểu thức sau: 
    \[
        \sqrt{1+\frac{1}{1^2}+\frac{1}{2^2}} + \sqrt{1+\frac{1}{2^2}+\frac{1}{3^2}} + \cdots + \sqrt{1+\frac{1}{2013^2}+\frac{1}{2014^2}}.
    \]
\end{problem*}

\begin{remark*}
    Lời giải các bài trên không được công bố công khai.
\end{remark*}

\newpage

\subsection{Cúp Con Dê - Kecske Kupa}

Kecske Kupa là một kỳ thi toán đồng đội dành cho học sinh từ lớp 5 đến lớp 8, do Quỹ MATEGYE tổ chức.
Kỳ thi diễn ra qua ba vòng với thể thức độc đáo, khuyến khích tư duy sáng tạo và phản xạ nhanh.
Các đội thi gồm bốn học sinh cùng trường và cùng khối lớp, phối hợp để giải quyết các bài toán theo một hình thức thi đặc biệt.

\textbf{Hình thức và nội dung kỳ thi}

Mỗi đội sẽ giải các bài toán thuộc bốn chủ đề toán học, gồm Kecskeszámtan (Số học Kecske), Kecskegebra (Đại số Kecske),
Kecskemetria (Hình học Kecske) và Kecskegyetem (Tư duy tổng hợp Kecske). Mỗi chủ đề có tám bài toán, được in trên giấy màu khác nhau để phân biệt.

Các bài toán được thiết kế để kiểm tra khả năng tư duy logic, lập luận chặt chẽ và chiến lược giải quyết vấn đề. Nội dung bài thi bao gồm nhiều dạng toán khác nhau:
\begin{itemize}[topsep=0pt, partopsep=0pt, itemsep=0pt]
    \ii Số học và lý thuyết số: Các phép toán cơ bản, số nguyên tố, tính chất chia hết, dãy số và hệ thập phân.
    \ii Đại số và phương trình: Biểu thức đại số, phương trình, bất đẳng thức, hệ phương trình, dãy số và quy luật số học.
    \ii Hình học và đo lường: Bài toán hình học phẳng và không gian, diện tích, thể tích, định lý Pythagoras, định lý Thales và các bài toán tối ưu hóa.
    \ii Tổ hợp và xác suất: Nguyên lý đếm, hoán vị, tổ hợp, xác suất cơ bản, lập bảng và phân tích chiến thuật tối ưu.
    \ii Tư duy logic và chiến lược: Câu đố toán học, trò chơi chiến lược, phân tích mô hình, suy luận logic phức tạp.
\end{itemize}

Tại mỗi vòng thi, đội sẽ nhận bài đầu tiên của từng chủ đề và có thể tự chọn thứ tự giải quyết. Sau khi hoàn thành một bài, đội sẽ nộp bài cho ban giám khảo để chấm điểm ngay.
Nếu đáp án đúng, đội nhận điểm và tiếp tục với bài tiếp theo cùng chủ đề. Nếu sai, đội bị trừ 1 điểm và có thể thử lại hoặc chuyển sang bài mới.

Một bài toán có thể thử tối đa sáu lần. Nếu sai cả sáu lần, đội sẽ bị trừ 6 điểm và buộc phải chuyển sang bài tiếp theo.
Một số bài toán không có đáp án đúng trong hệ thống, yêu cầu đội phải phát hiện điều này sớm để tránh mất điểm do thử lại quá nhiều lần.

Điểm số của các bài toán tăng dần theo mức độ khó. Các bài toán đơn giản có điểm số từ 5-8 điểm, bài trung bình từ 9-14 điểm,
trong khi các bài toán khó nhất có thể đạt từ 16-20 điểm. Điều này đòi hỏi đội thi phải có chiến lược hợp lý trong việc lựa chọn bài toán để tối đa hóa điểm số.

\textbf{Lịch trình kỳ thi}

Kỳ thi diễn ra qua ba vòng:
\begin{enumerate}[topsep=0pt, partopsep=0pt, itemsep=0pt]
    \ii Vòng 1 diễn ra thường vào đầu tháng 12 tại các điểm thi địa phương.
    \ii Vòng 2 tổ chức thường vào đầu tháng 3 năm sau,
    gồm phần thi chính và phiên đấu giá Kecske Tallérs (đồng bạc con dê) – đơn vị điểm thưởng đặc biệt mà các đội có thể dùng để đổi lấy phần thưởng.
    \ii Vòng chung kết được tổ chức thường vào đầu tháng 5 tại Kecskemét, nơi các đội xuất sắc nhất từ vòng 2 tranh tài để giành chiến thắng chung cuộc.
\end{enumerate}

Tất cả đội đăng ký đều có thể tham gia hai vòng đầu. Kết quả từ hai vòng này sẽ quyết định đội nào vào chung kết.
Danh sách đội vào chung kết được xác định dựa trên tổng điểm số và thứ hạng trong từng khu vực.

Học sinh đăng ký trực tuyến thường vào cuối tháng 10 tại www.mategye.hu. Lệ phí tham gia là 2000 Ft (140.000 VND)/học sinh, tức 8000 Ft (550.000 VND)/đội.

Mỗi trường có thể đăng ký nhiều đội. Trong suốt giải đấu, mỗi đội có thể thay đổi thành viên một lần,
nhưng người thay thế không được từng tham gia thi với đội khác trong cùng năm học.

\textbf{Đánh giá và giải thưởng}

Điểm số của các đội được cập nhật trực tiếp trên màn hình trong suốt kỳ thi, giúp đội thi theo dõi vị trí của mình và điều chỉnh chiến thuật kịp thời.

Kết quả được xét riêng theo từng cấp lớp. Hai đội đứng đầu mỗi cấp lớp tại từng địa điểm thi sẽ tự động vào chung kết nếu có đủ số lượng đội tham gia.
Ngoài ra, những đội có tổng điểm cao nhất toàn quốc cũng có thể được chọn vào vòng chung kết.

Các đội tham gia vòng 1 và vòng 2 sẽ nhận Kecske Tallérs, dùng để đấu giá phần thưởng sau vòng 2.
Những đội xuất sắc nhất trong vòng này sẽ nhận chứng nhận và Kecske Tallérs thưởng.

Ở vòng chung kết, đội đứng đầu mỗi cấp lớp nhận quà lưu niệm và chứng nhận, trong khi các đội còn lại cũng được trao chứng nhận thành tích.

\textbf{Ý nghĩa kỳ thi}

Kỳ thi không chỉ kiểm tra khả năng toán học mà còn khuyến khích tinh thần đồng đội, chiến lược và tư duy phản xạ.
Hình thức thi linh hoạt giúp học sinh phát triển kỹ năng hợp tác, ra quyết định nhanh và làm quen với cách giải toán sáng tạo trong áp lực thời gian.

\begin{remark*}
    Các đề thi không được công bố công khai.
\end{remark*}

\newpage

\subsection{Kỳ thi Toán - Khoa học Đồng đội Dürer}

Kỳ thi Dürer là một trong những kỳ thi học thuật đồng đội uy tín nhất tại Hungary, tập trung vào ba môn Toán học, Vật lý và Hóa học.
Được tổ chức lần đầu vào năm 2007, kỳ thi khuyến khích tư duy sáng tạo, khả năng giải quyết vấn đề và làm việc nhóm hiệu quả.
Đây không phải là một kỳ thi cá nhân mà là sân chơi dành cho các đội thi từ hai đến ba học sinh,
nơi các thành viên phải phối hợp chặt chẽ để tìm ra lời giải chính xác và tối ưu nhất.
Gần đây kỳ thi được tổ chức với sự hỗ trợ của Chương trình Tài năng Quốc gia (Nemzeti Tehetség Program).

\textbf{Đối tượng tham gia và phân loại}

Kỳ thi dành cho các đội học sinh từ lớp 5 đến lớp 12, với hai nhóm chính:
\begin{itemize}[topsep=0pt, partopsep=0pt, itemsep=0pt]
    \ii KisDürer: Dành cho học sinh tiểu học (lớp 5-8).
    \ii Dürer: Dành cho học sinh trung học (lớp 9-12).
\end{itemize}

Mỗi đội có thể lựa chọn thi theo một trong ba môn: Toán học, Vật lý hoặc Hóa học. Các đội thi thuộc nhóm Dürer (trung học) sẽ đối mặt với những bài toán có độ khó cao hơn,
đòi hỏi khả năng tư duy trừu tượng và vận dụng nhiều kỹ thuật nâng cao hơn so với nhóm KisDürer (tiểu học).

\textbf{Cấu trúc và nội dung kỳ thi}

Kỳ thi diễn ra qua hai vòng: vòng loại và vòng chung kết.
\begin{enumerate}[topsep=0pt, partopsep=0pt, itemsep=0pt]
    \ii Vòng loại: Các đội sẽ giải một bộ đề gồm các bài toán tự luận thuộc lĩnh vực Toán, Vật lý hoặc Hóa học.
    Đề thi yêu cầu không chỉ nắm vững kiến thức mà còn phải có tư duy logic, lập luận chặt chẽ và kỹ năng giải quyết vấn đề theo nhóm.
    \ii Vòng chung kết: Những đội có kết quả tốt nhất ở vòng loại sẽ được mời tham gia vòng chung kết.
    Tại đây, các đội sẽ đối mặt với những thử thách phức tạp hơn, bao gồm cả bài thi lý thuyết và bài tập thực hành.
    Đề thi không chỉ kiểm tra khả năng tư duy toán học - khoa học mà còn yêu cầu sự phối hợp nhịp nhàng giữa các thành viên để đưa ra chiến lược giải quyết bài toán hiệu quả nhất.
\end{enumerate}
	
\textbf{Các loại bài toán}

Các bài toán trong kỳ thi Dürer không chỉ kiểm tra kiến thức mà còn yêu cầu thí sinh phối hợp tư duy sáng tạo và vận dụng kiến thức vào thực tiễn.
Một số dạng bài toán thường gặp bao gồm:
\begin{itemize}[topsep=0pt, partopsep=0pt, itemsep=0pt]
    \ii Toán học: Chứng minh bất đẳng thức, bài toán tổ hợp, số học. Hình học phẳng và không gian, yêu cầu tính diện tích, thể tích, chứng minh quan hệ hình học.
    Phương trình hàm, đa thức và lý thuyết đồ thị.
    \ii Vật lý: Cơ học, điện từ học, nhiệt động lực học. Ứng dụng các định luật vật lý để giải quyết bài toán thực tế.
    Một số bài toán yêu cầu thiết kế mô hình hoặc tính toán dựa trên số liệu thực nghiệm.
    \ii Hóa học: Phản ứng hóa học, cân bằng phương trình, hóa lý. Phân tích hợp chất, tính toán hóa học dựa trên số liệu thực nghiệm.
    Một số bài toán yêu cầu hiểu biết về hóa học ứng dụng.
\end{itemize}

Ví dụ, một bài toán hình học có thể yêu cầu các đội thiết kế một logo 2D sử dụng các hình tam giác đều và hình vuông, sao cho chu vi của logo đạt 13 cm.
Một bài toán khác có thể yêu cầu tính diện tích của một viên gạch có các cạnh là cung tròn, với chiều cao của viên gạch là 12 cm.
Những bài toán này không chỉ kiểm tra kiến thức mà còn khuyến khích học sinh phối hợp suy luận, chia sẻ chiến lược và hỗ trợ lẫn nhau trong quá trình tìm lời giải.

\textbf{Mục tiêu của kỳ thi}

Với cấu trúc thi đặc biệt và các bài toán có độ thử thách cao, kỳ thi Dürer đã trở thành một sự kiện quan trọng cho những học sinh yêu thích toán học và khoa học.
Kỳ thi Dürer không chỉ là một bài kiểm tra kiến thức mà còn tạo ra một môi trường học tập thú vị, thúc đẩy tư duy sáng tạo và tinh thần đồng đội trong học sinh.
Thông qua các thử thách đa dạng, kỳ thi giúp học sinh phát triển kỹ năng tư duy phản biện, làm việc nhóm và khả năng giải quyết vấn đề theo hướng hợp tác hiệu quả.

Ngoài ra, kỳ thi còn giúp các em chuẩn bị tốt hơn cho các kỳ thi học thuật quốc gia và quốc tế, cũng như phát triển niềm đam mê nghiên cứu khoa học.
Những học sinh có thành tích xuất sắc trong kỳ thi Dürer thường tiếp tục tham gia các kỳ thi lớn như Olympic Toán Quốc tế (IMO),
Olympic Vật lý Quốc tế (IPhO) hoặc Olympic Hóa học Quốc tế (IChO).

\bigbreak

\noindent\rule{16.5cm}{0.4pt}

\textbf{Một số bài thi tiêu biểu}

\bigbreak

\begin{problem*}[2024-2025, vòng chung kết, mức C, bài 6 \cite{18KC}]
    Trên một bàn cờ $4 \times 4$, có một quân mã đứng trên một ô (không có quân cờ nào khác). Hai người chơi lần lượt thay nhau di chuyển quân mã.
    Không được di chuyển đến ô mà quân mã đã từng đi qua trước đó, bao gồm cả ô xuất phát. Người thua cuộc là người không thể thực hiện nước đi hợp lệ.
    Bạn có chiến lược nào để luôn thắng nếu được chọn đi trước hay đi sau?
\end{problem*}

\begin{problem*}[2024-2025, vòng chung kết, mức D, bài 4 \cite{{18KD}}]
    Một vị vua tàn nhẫn đã xây dựng một nhà tù với 36 phòng giam được sắp xếp theo dạng lưới $6 \times 6$, trong đó mỗi cặp phòng giam liền kề đều bị ngăn cách bởi một bức tường.
    Một số phòng đã được chỉ định cho tù nhân, nhưng mỗi phòng chỉ thuộc về tối đa một tù nhân.

    Về sau, nhà vua cảm thấy mình quá tàn nhẫn, nên quyết định phá bỏ một số bức tường sao cho bất kỳ phòng nào cũng có thể đi đến bất kỳ phòng nào khác.
    Tuy nhiên, ông không muốn tù nhân quá thoải mái, nên muốn giữ lại ít nhất một bức tường
    giữa bất kỳ hai tù nhân nào ban đầu nằm trên cùng một hàng hoặc cột để họ không thể nhìn thấy nhau từ vị trí của mình.

    Tối đa có bao nhiêu tù nhân có thể bị giam giữ nếu nhà vua phá tường theo cách thỏa mãn các điều kiện trên?
\end{problem*}

\begin{problem*}[2016-2017, vòng chung kết, mức E, bài 5 \cite{18KC}]
    Tồn tại hay không vô số các đường thằng sao cho: không có hai đường nào song song với nhau, không có ba đường nào đồng quy,
    và các giao điểm của các đường thẳng này đều cách mỗi đường thằng một khoảng cách có độ dài nguyên dương?
\end{problem*}

\begin{remark*}
    Lời giải các bài trên có thể tìm thấy tại: \cite{18KC_mo}, \cite{18KD_mo}, và \cite{18KE_mo}.
\end{remark*}

\newpage

\section*{Tham khảo}

\begin{thebibliography}{99}
    \bibitem{EGMO} European Girls’ Mathematical Olympiad, \url{https://www.egmo.org}
    \bibitem{MEMO} Middle European Mathematical Olimpiad, \url{https://www.memo-official.org/MEMO/}
    \bibitem{NMMV} Nemzetközi Magyar Matematikaverseny, \url{https://nmmv.berzsenyi.hu/főoldal}
    \bibitem{MATEGYE} Matematikában Tehetséges Gyermekekért, \url{http://www.mategye.hu}
    \bibitem{KoMaL} A KöMaL pontversenyei, \url{https://www.komal.hu/verseny.h.shtml}
    \bibitem{AranyDaniel} Arany Dániel Matematikai Tanulóverseny,\\ \url{https://www.bolyai.hu/versenyek-arany-daniel-matematikaverseny/}
    \bibitem{OKTV} OKTV Versenyfeladatok,\\ \url{https://www.oktatas.hu/kozneveles/tanulmanyi_versenyek_/oktv_nyito}
    \bibitem{Kurschak} Kürschák József Matematikai Tanulóverseny,\\ \url{https://www.bolyai.hu/versenyek-kurschak-jozsef-matematikai-tanuloverseny/}
    \bibitem{Megyei} Megyei Matematikaverseny, \url{http://www.mategye.hu/?pid=megyeiverseny}
    \bibitem{KecskeKupa} Kecske Kupa, \url{http://www.mategye.hu/?pid=tehetsegnap_kecskekupa}
    \bibitem{Durer} Dürer Matematikaverseny, \url{https://durerinfo.hu}
    \bibitem{2016-2017-2f-8o-p5} \url{http://www.mategye.hu/download/bkmi/2016-2017/2_8.pdf}
    \bibitem{2016-2017-2f-9o-p6} \url{http://www.mategye.hu/download/bkmi/2016-2017/2_9g.pdf}
    \bibitem{2016-2017-2f-10s-p6} \url{http://www.mategye.hu/download/bkmi/2016-2017/2_10s.pdf}
    \bibitem{2013-2014-2f-11o-p5} \url{http://www.mategye.hu/download/bkmi/2013/feladatsor_2013_2_ford_11_oszt_gimn.pdf}
    \bibitem{2013-2014-2f-12s-p6} \url{http://www.mategye.hu/download/bkmi/2013/feladatsor_2013_2_ford_12_oszt_gimn.pdf}
    \bibitem{18KC} \url{https://durerinfo.hu/wp-content/uploads/2025/02/18KC.pdf}
    \bibitem{18KC_mo} \url{https://durerinfo.hu/wp-content/uploads/2025/02/18KC_mo.pdf}
    \bibitem{18KD} \url{https://durerinfo.hu/wp-content/uploads/2025/02/18KD.pdf}
    \bibitem{18KD_mo} \url{https://durerinfo.hu/wp-content/uploads/2025/02/18KD_mo.pdf}
    \bibitem{18KE} \url{https://durerinfo.hu/wp-content/uploads/2025/02/18KE.pdf}
    \bibitem{18KE_mo} \url{https://durerinfo.hu/wp-content/uploads/2025/02/18KE_mo.pdf}
\end{thebibliography}

\end{otherlanguage*}

\end{document}