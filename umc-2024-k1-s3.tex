\documentclass[8pt,xcolor=table,dvipsnames]{beamer}
\usepackage{pgfpages}
\usepackage{yhmath}
\newcommand{\Mod}[1]{\ (\mathrm{mod}\ #1)}
\providecommand{\half}{\frac{1}{2}}
\newcommand{\dg}{^\circ}
\newcommand{\arc}[1]{\wideparen{#1}}
\usetheme{Madrid}

\title{Combinatorial Geometry}
\subtitle{UMC K1, 2024}
\author{Nghia Doan}
\institute{MCC Club \& Competitions}
\date{\today}

\begin{document}

\section{Points, segments, and lines}

\begin{frame}[t]
    \frametitle{Combinatorial Geometry}
    \framesubtitle{Points, segments, and lines - Example 1}
    \begin{example}[HC-2021-SM2-R3-P11]
        On the blackboard six segments are drew. Every two segments intersect one another at a point.
        The $1^{\text{st}}$ segment contains three of the intersections, the $2^{\text{nd}}$ segment contain $4$ of the intersections,
        the $3^{\text{rd}},$ $4^{\text{th}},$ and $5^{\text{th}}$ segments each contains five intersections.
        What segments does the $6^{\text{th}}$ segment intersect with?
    \end{example}
    
    \begin{center}
        \includegraphics[width=5cm]{../Learning-Problem-Solving-2nd-Edition/svg/pdf/hc-2021-2-3-11.pdf}
    \end{center}

    \onslide<2->Each of the $3^{\text{rd}},$ $4^{\text{th}},$ and $5^{\text{th}}$ are to intersect all the remaining segments.

    \onslide<3->Therefore the $1^{\text{st}}$ segment intersects with the $3^{\text{rd}},$ $4^{\text{th}},$ and $5^{\text{th}}$ segments only.
    
    \onslide<4->Thus, the $2^{\text{nd}}$ segment intersect with the $3^{\text{rd}},$ $4^{\text{th}},$ $5^{\text{th}},$ and $6^{\text{th}}$ segments.

    \onslide<5->Hence, the $6^{\text{th}}$ intersects with the \framebox{$2^{\text{nd}},$ $3^{\text{rd}},$ $4^{\text{th}},$ and $5^{\text{th}}$} segments.
\end{frame}

\begin{frame}[t]
    \frametitle{Combinatorial Geometry}
    \framesubtitle{Points, segments, and lines - Example 2}
    \begin{example}[PCT 2021/Mar/MT/3]
        There are $9$ distinct lines on the plane, no two of them are parallel, and no three of them meet at a single point.
        How many non-overlapping regions they divide the plane into?
    \end{example}
    
    \bigbreak
    \onslide<2->Count the regions by adding one line each time.
    With one line, the plane is divided into two non-overlapping regions.

    \bigbreak
    \onslide<3->Let assume there are $n-1$ lines on the plane and we add the $n^{\text{th}}$ line.
    When adding a new line, it should intersect all $n-1$  existing lines, therefore it is cut into $n$ parts.

    \bigbreak
    \onslide<4->These newly added parts divide $n$ existing regions, adding $n$ new regions.
    Therefore, the number of regions is \[ 2+2+3+4+\cdots+n=1+\frac{1}{2}{n(n+1){2}} = \frac{n^2+n+2}{2}. \]

    \onslide<5->For $n=9,$ \[ \frac{n^2+n+2}{2} = \frac{81+9+2}{2} = \boxed{46.} \]
\end{frame}

\begin{frame}[t]
    \frametitle{Combinatorial Geometry}
    \framesubtitle{Points, segments, and lines - Example 3}
    \begin{example}[AMC 12 2019/A/8]
        For a set of four distinct lines in a plane, there are exactly $N$ distinct points that lie on two or more of the lines.
        What is the sum of all possible values of $N$?
    \end{example}

    \begin{overprint}
        \onslide<2>The problem is equivalent to find the possible numbers of intersection of $4$ distinct lines. 

        \bigbreak
        The maximum number of intersections of $4$ lines is the number of pairs of lines can be chosen from 4 lines, or $\binom{4}{2}=6$. 
        The minimum number of intersections of $4$ lines, obviously, is $0.$
        \begin{center}
            \includegraphics[width=5cm]{../Learning-Problem-Solving-2nd-Edition/png/amc-12-2019-a-8.png}
        \end{center}
        The diagram shows examples for any number between $0$ and $6$, except $2$.
    
        \onslide<3>Now, assume that there are two intersections $A$ and $B$ so that all four lines go through them.

        \bigbreak
        \textit{Case 1:} if there is no line through both $A$ and $B$. The two lines that intersecting at $A$ must be pairwise parallel
        with the two lines that intersecting at $B$. In that case there will be more intersections among the non-parallel pairs.
    
        \bigbreak
        \textit{Case 2:} if there is a line through both $A$ and $B$. The other line passing through $A$ must be parallel
        with the other lines passing through $B$. The fourth line shall pass through only $A$ or $B$ and cannot be parallel
        with those parallel lines. In that case more intersection points shall exist.
    
        \bigbreak
        Hence, the sum of possible numbers of intersections is \framebox{$0+1+3+4+5+6=19.$}
    \end{overprint}
\end{frame}

\begin{frame}[t]
    \frametitle{Combinatorial Geometry}
    \framesubtitle{Points, segments, and lines - Example 4}
    \begin{example}[HC-2022-SM1-R10-P6]
        A \textit{close path} $A_1A_2 \ldots A_5$ contains $5$ segments $A_1A_2, A_2A_3, \ldots, A_5A_1$
        that has $5$ \textit{self-intersection} points .
        This close path has the \textit{maximum} number of self-intersection points.
        
        Find the \textit{maximum} number of self-intersection points
        for a $9-$segment close path $A_1A_2 \ldots A_9.$
    \end{example}
    \begin{center}
        \includegraphics[width=5.5cm]{../Learning-Problem-Solving-2nd-Edition/svg/pdf/hc-2022-1-10-6.pdf}
    \end{center}
\end{frame}

\begin{frame}[t]
    \frametitle{Combinatorial Geometry}
    \framesubtitle{Points, segments, and lines - Example 4}
    For two consecutive points, says $A_i$ and $A_{i+1},$ 
    except $A_{i-1} A_i, A_i A_{i+1},$ and $A_{i+1} A_{i+2}$ (where $A_0 \equiv A_n,$ and $A_{n+1} \equiv A_1$)
    the remaining $n-3$ segments of the path $A_1 A_2 \ldots A_n A_1$ 
    can intersect with $A_i A_{i+1}$ at at most $n-3$ points. 
    \onslide<2->Thus the maximum number of intersections is $\half n(n-3).$
    For a $n=9$, the maximum number of intersections is $\boxed{27.}$
    \begin{center}
        \includegraphics[width=6cm]{../Learning-Problem-Solving-2nd-Edition/svg/pdf/hc-2022-1-10-6-2.pdf}
    \end{center}
\end{frame}

\begin{frame}[t]
    \frametitle{Combinatorial Geometry}
    \framesubtitle{Points, segments, and lines - Example 5}
    \begin{example}[MIC-2022-SM2-R1-S9]
        On a straight line there are $10$ distinct points.
        For every pair of points,
        Anna \textit{marks} the midpoint of the segment connected them.
        What is the minimal number of distinct \textit{marked} points?
    \end{example}
    \begin{overprint}
        \onslide<2>Let $A$ and $B$ be the points with the greatest distance between them.
        Connect point $A$ by segments with all other points except $B$.
        The midpoints of the obtained $10 - 2 = 8$ segments do not coincide, 
        otherwise the second endpoints of the segments would coincide too,
        and are situated inside a circle centred at $A$ with radius $\frac{AB}{2}.$
        \textit{In the diagram below an example is shown with the given points colored red and
        the midpoints slightly moved out of their positions for better visualization.}    
        \begin{center}
            \includegraphics[width=8cm]{../Learning-Problem-Solving-2nd-Edition/svg/pdf/hc-2022-1-1-13.pdf}
        \end{center}
        \onslide<3>Similarly there are $8$ distinct midpoints situated in a circle centred at $B$ of radius
        with radius $\frac{AB}{2}.$ The two circles have exactly one common point,
        which is the midpoint of the segment $AB.$ Thus, together with the midpoint of $AB,$
        we have implicitly constructed $8 + 8 + 1 = \boxed{17}$ midpoints.
        \begin{center}
            \includegraphics[width=8cm]{../Learning-Problem-Solving-2nd-Edition/svg/pdf/hc-2022-1-1-13.pdf}
        \end{center}
        \textit{All given points lie on the same straight line at $0, 2, 4, \ldots, 18,$ so with a constant step between them. 
        It is easy to see that there are exactly $17$ distinct midpoints at $1, 2, 3, \ldots, 17.$}
    \end{overprint}
\end{frame}

\begin{frame}[t]
    \frametitle{Combinatorial Geometry}
    \framesubtitle{Points, segments, and lines - Example 6}
    \begin{example}[PCT 2021/Mar/MT/12]
        The main diagonal of the $5 \times 3$ rectangle passes through $7$ squares.
        The main diagonal of the $6 \times 4$ rectangle passes through $8$ squares.
       What is the number of squares passed through by the main diagonal of a $42 \times 30$ rectangle?
    \end{example}
    \onslide<2->When the diagonal passes from one square to the next,
    it passes from one row to the next or from one column to the next if it does not pass through a corner.

    \onslide<3->For an $m \times n$ rectangle, if the diagonal passes through no corner, then it passes through the top left square,
    then $m-1$ squares (one for each remaining row), and then $n-1$ squares (one for each remaining column), in total
    \[
        1+(m-1)+(n-1)=m+n-1
    \]
    \onslide<4->If the diagonal passes through one corner, then we must subtract one from the above sum.
    The number of corners depends on the common factors of $m$ and $n$.
    
    \onslide<5->In fact it is one less than the greatest common divisor of them. Hence the number of squares the diagonal passes through is,
    \[
        m+n-1-(\gcd(m,\ n)-1)=m+n-\gcd(m,\ n)
    \]
    \onslide<6->With $(m,n)=(42,30)$, the answer is \framebox{$42+30-\gcd(42,\ 30)=72-6=66.$}
\end{frame}

\section{Circles}

\begin{frame}[t]
    \frametitle{Combinatorial Geometry}
    \framesubtitle{Circles - Example 1}
    \begin{example}[MIC-2022-SM2-R3-J2]
        Twelve points are equally spaced on the circumference of a circle.
        How many chords can be drawn that connect pairs of these points
        and which are longer than the radius of the circle but shorter than its diameter?  
    \end{example}
    \begin{overprint}
        \onslide<2>We must first determine which diagonals are greater than the radius of the circle and less than the diameter.
        Diagonals such as $AC$ are equal in length to the radius.
        This can be seen by noting that $ACEGIK$ is a regular hexagon,
        and the sides of a regular hexagon are equal in length to the radius of the circumscribed circle.
        Diagonals such as $AG$ are diameters of the the circle.
        \begin{center}
            \includegraphics[width=4cm]{../Learning-Problem-Solving-2nd-Edition/svg/pdf/mic-2022-2-r3-2.pdf}
        \end{center}
        \onslide<3>This leaves diagonals like $AD$, $AE$, and $AF$, which are longer than $AC$ and shorter than $AG$.
        \begin{center}
            \includegraphics[width=4cm]{../Learning-Problem-Solving-2nd-Edition/svg/pdf/mic-2022-2-r3-2.pdf}
        \end{center}
        From each vertex there are $6$ such diagonals, for a total of $6(12)=72$.
        Hence, there are $\frac{72}{2}=\boxed{36}$ diagonals longer than the radius of the circle
        and less than the diameter. 
    \end{overprint}
\end{frame}

\begin{frame}[t]
    \frametitle{Combinatorial Geometry}
    \framesubtitle{Circles - Example 2}
    \begin{example}[HC-2022-SM2-R2-P9]
        In a ceremony of the guild, the master asked the members to form $10$ lines,
        each line consists of $9$ persons, so that he can stand in \textit{a place that has the same distance to every row.}    
        The guild has $81$ members., including the master.
        How can they do that?
    \end{example}
    \onslide<2->The key here is that \textit{the master is equidistant from each row}, not each member.
    
    \onslide<3->Thus, there should be $10$ rows, all are at the same distant from a point.
    Thus these are $10$ chords of a circle with the same length.
    \onslide<4->\textit{To arrange $80$ persons into $10$ chords, each has $9$ persons, 
    we create a regular polygon with $9$ persons on each side.}
    \begin{center}
        \includegraphics[width=4.2cm]{../Learning-Problem-Solving-2nd-Edition/svg/pdf/hc-2022-2-2-9.pdf}
    \end{center}
\end{frame}

\begin{frame}[t]
    \frametitle{Combinatorial Geometry}
    \framesubtitle{Circles - Example 3}
    \begin{example}[MIC-2021-SM2-R1-P7]
        $10$ vertices are placed on a circle.
        What is the maximum number of line segments, connecting two of the given vertices,
        that can be drawn such that no two intersect each other, except at the vertices?
    \end{example}
    \onslide<2->The $10$ sides and $d$ diagonals divide the $10-$gon into $k$ triangles.
    
    \onslide<3->If we count the segments by the triangles, each triangles has three segments as sides,
    each segment is counted once if it is a side of the $10-$gon,
    and twice if it is a diagonal (shared by two triangles), so
    \[
        3k = 10 + 2d \quad(1)
    \]
    
    \onslide<4->On the other hand, by connecting the segments, we create a so-called \textit{planar graph},
    by Euler's graph formula $E-V+F=2$ for $10$ vertices, $d+10$ edges, and $k+1$ faces:
    \[
        10 - (d+10) + (k+1) = 2 \Rightarrow k = d+1 \quad(2)
    \]
    Therefore, from (1) and (2), $d=7,\ k=8.$
    
    \onslide<5->Thus, the number of segment is \framebox{$d+10=17.$}
\end{frame}

\begin{frame}[t]
    \frametitle{Combinatorial Geometry}
    \framesubtitle{Circles - Example 4}
    \begin{example}[IMO 1975/5]
        Can $1975$ distinct points be drawn on a circle of radius $1$,
        so that the distance (measured on the chord) between any two points is a rational number?
    \end{example}
    \begin{overprint}
        \onslide<2>We prove that $n$ ($n\ge 2$) distinct points can be placed on a circle of radius $1$,
        such that the distance between any two points is a rational number.
    
        \bigbreak
        For 2 points, it is simple to see that the two vertices of diameter is such a pair of points.
        
        \bigbreak
        For 3 points, for a Pythagorean triple $(a_1,b_1,c_1)$ of integers: $c_1^2=a_1^2+b_1^2$,
        a point $P_1$ can be chosen on the semicircle $\arc{AB}$ diameter 2, 
        such that both $P_1A$ and $P_1B$ are rational:
        \[
            AB=2,\ P_1A=\frac{2a}{c},\ P_1B=\frac{2b}{c} \Rightarrow P_1A^2+P_1B^2 = \frac{4(a^2+b^2)}{c^2}=AB^2
            \Rightarrow P_1 \in \arc{AB}.
        \]
        \onslide<3>Therefore, by using $n \ge 1$ distinct Pythagorean triples $(a_i,b_i,c_i)$, where
        $\forall i=1,\ldots,n,\ c_i^2=a_i^2+b_i^2$,
        $n$ distinct points $P_1,P_2,\ldots,P_n$ can be chosen on the semicircle $\arc{AB}$,
        so that all $P_iA$, $P_iB$ are rationals.
        \begin{center}
            \includegraphics[width=4cm]{../Learning-Problem-Solving-2nd-Edition/asy/pdf/imo-1975-5-1.pdf}
        \end{center}
        For any $1 \le i \ne j \le n$, $ABP_jP_i$ is a cyclic quadrilateral,
        by the Ptolemy Theorem,
        \[
            AB,\ P_iA,\ P_iB,\ P_iA,\ P_jB\ \text{are rational}\ \Rightarrow
            P_iP_j = \frac{P_iB \cdot P_jA - P_iA \cdot P_jB}{AB} \text{are rational}
        \]
    \end{overprint}
\end{frame}

\section{Topology}

\begin{frame}[t]
    \frametitle{Combinatorial Geometry}
    \framesubtitle{Topology - Example 1}
    \begin{example}[HC-2022-SM2-R3-P11]
        Find the minimal value $m$ so that no matter how you partition
        a unit square into two distinct sets of points,
        the diameter of one of the sets is at least $m.$ 
        \textit{Diameter of a set of points is the largest distance between two points of the set.}
    \end{example}
    \begin{overprint}
        \onslide<2>Let $ABCD$ be the unit square, $E$ and $F$ be the midpoints of $CD$ and $DA.$
        We prove that $m$ is equal to the length of $AE.$
        Let assume that $ABCD$ can be divided into two sets $\mathcal{G}$ and $\mathcal{R}$ such that
        \textit{none of the sets has a diameter larger than or equal to $AE.$}
        \begin{center}
            \includegraphics[width=7cm]{../Learning-Problem-Solving-2nd-Edition/svg/pdf/hc-2022-2-3-11-2.pdf}
        \end{center}    
        \onslide<3>WLOG, let $A \in \mathcal{G},$ then $E \not \in \mathcal{G},$ thus $E \in \mathcal{R},$
        therefore $B \in \mathcal{G},$ so $F \in \mathcal{R},$ hence $C \in \mathcal{G},$
        but then the diameter of $\mathcal{G}$ is at least $AC > AE.$
        Contradiction.
        \begin{center}
            \includegraphics[width=7cm]{../Learning-Problem-Solving-2nd-Edition/svg/pdf/hc-2022-2-3-11-2.pdf}
        \end{center}    
        The diagram on the right of the figure above shows
        the case when such diameter is equal to $AE$
        Therefore one of the sets has a diameter larger than or equal to $AE.$
    \end{overprint}
\end{frame}

\begin{frame}[t]
    \frametitle{Combinatorial Geometry}
    \framesubtitle{Topology - Example 2}
    \begin{example}[LPS V2B/2.44]
        Prove that there exist three consecutive vertices $A,B,C$ in every convex $n-$gon ($n \ge 3$),
        such that the circumcircle of $\triangle ABC$ covers the whole $n-$gon.
    \end{example}
    \begin{overprint}
        \onslide<2>Consider all circles through three vertices of the $n-$gon. There is a finite number of such circles.
        Let $\Omega$ be the maximal such circle. We will prove the following statements:
        (1) $\Omega$ covers the whole $n-$gon; (2) $\Omega$ passes through three consecutive vertices.

        For the first statement, suppose that vertex $A'$ lies outside $\Omega.$
        \begin{center}
            \includegraphics[width=6cm]{../Learning-Problem-Solving-2nd-Edition/asy/pdf/lps-v2b-2-44.pdf}
        \end{center}
        Because $\Omega$ is the circumcircle of $\triangle ABC$, where $A,B,C$ are denoted such that $A,B,C,A'$ form convex quadrilateral.
        Easy to see that the circumcircle of $\triangle A'BC$ is larger than $\Omega$, see the left diagram.
        This is contradiction.
        \onslide<3>For the second statement, let $A,B,C$ be vertices on $\Omega$.
        Suppose there exists vertex $A'$ between $B$, $C$ and $A'$ not on $\Omega$.
        \begin{center}
            \includegraphics[width=6cm]{../Learning-Problem-Solving-2nd-Edition/asy/pdf/lps-v2b-2-44.pdf}
        \end{center}
        Because of the first statement, $\Omega$ covers the whole $n-$gon, so $A'$ is inside $\Omega.$
        Hence the circumcircle of $\triangle A'BC$ is larger than $\Omega,$ see the right diagram.
        This is a contradiction.    
    \end{overprint}
\end{frame}

\end{document}
