\documentclass{article}

\usepackage[main=english,vietnamese]{babel}
\usepackage[T1]{fontenc}
\usepackage[utf8]{inputenc}
\usepackage[sexy]{evan}
\usepackage{matchsticks}
\usepackage{wrapfig}
\usepackage{listings}

\begin{otherlanguage*}{vietnamese}
    \title{Các Kỳ Thi Toán Học Tại Hungary\\ \quad \\Phần 5 - Lịch sử và thành tích của Hungary tại Olympic Toán Quốc tế (IMO)}
\end{otherlanguage*}

\author{Nghia Doan}
\date{\today}

\begin{document}

\begin{otherlanguage*}{vietnamese}

\maketitle

Hungary có một lịch sử đáng tự hào tại Olympic Toán Quốc tế (IMO) \cite{IMO_team}, thể hiện truyền thống toán học mạnh mẽ và cam kết nuôi dưỡng tài năng trẻ.

\textbf{Thành tích tổng quát}

\begin{itemize}[topsep=0pt, partopsep=0pt, itemsep=0pt]
    \ii \textbf{Lần tham gia đầu tiên}: 1959
    \ii \textbf{Số lần tham gia}: 64
    \ii \textbf{Bảng thành tích huy chương}:
    \begin{itemize}[topsep=0pt, partopsep=0pt, itemsep=0pt]
        \ii Vàng: 88
        \ii Bạc: 174
        \ii Đồng: 116
        \ii Giải khuyến khích: 10
    \end{itemize}
\end{itemize}

\textbf{Những thành tựu nổi bật}
\begin{itemize}[topsep=0pt, partopsep=0pt, itemsep=0pt]
    \ii Hungary giành vị trí số một toàn đoàn sáu lần vào các năm 1961, 1962, 1969, 1970, 1971 và 1975.
    \ii Đội tuyển Hungary năm 1971 đặc biệt ấn tượng với bốn thí sinh giành huy chương vàng và bốn thí sinh giành huy chương bạc.
    Đáng chú ý, bảy trong số những thành viên này đến từ một trường trung học danh tiếng ở Budapest, và ba trong số những thí sinh giành huy chương vàng sau này đã có sự nghiệp học thuật xuất sắc.
\end{itemize}

\textbf{Những cá nhân xuất sắc}

\begin{itemize}[topsep=0pt, partopsep=0pt, itemsep=0pt]
    \ii \textbf{László Lovász}: Tham gia IMO bốn lần (1963-1965), đạt ba huy chương vàng, hai giải đặc biệt, và một huy chương bạc. 
    Ông sau này trở thành một nhà toán học nổi tiếng, giữ chức Chủ tịch Liên minh Toán học Quốc tế và có nhiều đóng góp quan trọng trong lĩnh vực tổ hợp và khoa học máy tính lý thuyết.
    \ii \textbf{József Pelikán}: Tham gia IMO bốn lần (1963-1966), đạt ba huy chương vàng, hai giải đặc biệt, và một huy chương bạc.  
    Sau đó, ông giữ nhiều vai trò quan trọng trong công tác tổ chức IMO, bao gồm việc làm Chủ tịch Hội đồng Cố vấn của kỳ thi.
\end{itemize}

\textbf{Các giai đoạn thành tích của Hungary tại IMO}
\begin{itemize}[topsep=0pt, partopsep=0pt, itemsep=0pt]
    \ii Giai đoạn đầu cho đến trước năm 1988 (1959–1987): Hungary là một trong những quốc gia sáng lập IMO vào năm 1959.
    Trong giai đoạn này, đội tuyển Hungary luôn nằm trong nhóm dẫn đầu, giành vị trí số một toàn đoàn sáu lần.
    Đặc biệt, năm 1975, Hungary giành giải nhất đồng đội dù không có huy chương vàng cá nhân nào, với thành tích năm huy chương bạc và ba huy chương đồng.
    Tính trung bình, thứ hạng đội tuyển Hungary là 3 với 42V, 76B, 52Đ trong 27 năm.
    \ii Giai đoạn chuyển tiếp (1988-2004): Những năm 1990 và đầu những năm 2000 là thời kỳ chuyển đổi của Hungary tại IMO.
    Dù vẫn duy trì thành tích tốt, nhưng sự cạnh tranh gia tăng khi ngày càng nhiều quốc gia tham gia.
    Trong giai đoạn này, Hungary vẫn nằm trong nhóm 10 đội tuyển mạnh nhất, với nhiều cá nhân giành huy chương vàng.
    Thành tích nổi bật như vị trí thứ 2 vào năm 1997 và thứ 3 hai năm 1996, 1998.
    Tính trung bình, thứ hạng đội tuyển Hungary là 8 với 28V, 47B, 21Đ trong 17 năm.
    \ii Thành tích gần đây (2005-2024): Trong những năm gần đây, Hungary vẫn tiếp tục duy trì thành tích đáng kể tại IMO.
    Đội tuyển thường xuyên nằm trong top 30, với một số thành tích nổi bật như vị trí thứ 8 vào năm 2024.
    Nhiều thí sinh cá nhân cũng xuất sắc giành huy chương tại các kỳ thi gần đây.
    Tính trung bình, thứ hạng đội tuyển Hungary là 19 với 18V, 51B, 43Đ trong 20 năm.
\end{itemize}

\textbf{Thành tích đội tuyển Hungary tại IMO suy giảm}

Hungary từng là một trong những quốc gia hàng đầu tại IMO, nhưng thứ hạng của đội tuyển đã giảm dần theo thời gian.
Khi so sánh với các quốc gia hiện đang có thành tích tốt hơn, có thể thấy một số yếu tố quan trọng đã góp phần vào sự suy giảm này.

\begin{itemize}[topsep=0pt, partopsep=0pt, itemsep=0pt]
    \ii \textit{Hệ thống giáo dục và mức lương giáo viên}: Mức lương giáo viên và hệ thống giáo dục vững chắc đóng vai trò quan trọng trong việc thu hút
    và giữ chân những giáo viên chất lượng cao.
    Các quốc gia như Trung Quốc và Hàn Quốc có hệ thống giáo dục chuyên sâu với sự hỗ trợ mạnh mẽ từ chính phủ, đảm bảo mức lương hấp dẫn cho giáo viên.
    Trong khi đó, hệ thống giáo dục của Hungary gặp phải những thách thức, bao gồm mức lương giáo viên chưa đủ cạnh tranh, ảnh hưởng đến chất lượng giảng dạy.
    \ii \textit{Đầu tư kinh tế vào giáo dục}: Các quốc gia dành nhiều ngân sách hơn cho giáo dục thường có thành tích học thuật tốt hơn.
    Ví dụ, Singapore đầu tư mạnh vào hệ thống giáo dục STEM, giúp quốc gia này gặt hái nhiều thành công trong các kỳ thi quốc tế.
    Trong khi đó, mức đầu tư của Hungary vào giáo dục thấp hơn, hạn chế nguồn lực dành cho việc đào tạo học sinh tham dự IMO.
    \ii \textit{Nhân khẩu học và nguồn tài năng}: Cơ cấu dân số ảnh hưởng đến số lượng học sinh có thể tham gia các cuộc thi toán học.
    Ấn Độ có dân số trẻ đông đảo, mang lại nguồn tài năng dồi dào để đào tạo cho IMO.
    Ngược lại, Hungary đang đối mặt với tỷ lệ sinh giảm, làm giảm số lượng học sinh có thể được tuyển chọn và đào tạo.
    \ii \textit{Thích ứng với các phương pháp giáo dục hiện đại}: Các quốc gia áp dụng các phương pháp giáo dục hiện đại thường có thành tích tốt hơn trong các kỳ thi quốc tế.
    Ví dụ, Hoa Kỳ triển khai các trại huấn luyện chuyên sâu và hội thảo giải toán nâng cao, giúp họ cải thiện đáng kể thứ hạng.
    Trong khi đó, phương pháp đào tạo truyền thống của Hungary có thể chưa theo kịp sự phát triển của các mô hình huấn luyện hiện đại.
\end{itemize}

Hungary đang xem xét các chiến lược của những quốc gia đang thành công tại IMO để xây dựng lại vị thế của mình.
Việc triển khai các chương trình Chương trình Tài năng Quốc gia, Quỹ Vì Trẻ Em Tài Năng Toán Học, duy trì các kỳ thi quốc gia, tham gia các kỳ thi khu vực,
và công khai thông tin đảm bảo cho các kỳ thi tuyển chọn các đội tuyển có đầy đủ sự minh bạch.

\newpage

\section*{References}

\begin{thebibliography}{99}
    \bibitem{IMO_team} Đội tuyển IMO Hungary, \url{https://www.imo-official.org/country_team_r.aspx?code=HUN}
    \bibitem{IMO_team} Bảng vinh danh,\\ \url{https://www.imo-official.org/country_hall.aspx?code=HUN&column=participations&order=desc}
\end{thebibliography}

\end{otherlanguage*}

\end{document}