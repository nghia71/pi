\documentclass{article}

\usepackage[main=english,vietnamese]{babel}
\usepackage[T1]{fontenc}
\usepackage[utf8]{inputenc}
\usepackage[sexy]{evan}
\usepackage{matchsticks}
\usepackage{wrapfig}
\usepackage{listings}

\newtheorem{hint}{Hint}

\title{Solving Forty Two Problems by the Induction Principle - Part V}
\author{Nghia Doan}
\date{\today}

\begin{document}

\maketitle

\begin{problem}[Problem Twenty Three]
    The sequence $(a_n)$ is defined by,
    \[
        a_1 = \half, a_{n+1} = \frac{2n-1}{2n+2} a_n
    \]
    
    Prove that $a_1 + a_2 + \cdots + a_n < 1$ for $n \ge 1.$
\end{problem}

\begin{soln}
    Consider $s_n = a_1 + a_2 + \cdots + a_{n-1} + 2n a_{n},\ \forall n\ge 2,$ then $,$
    so 
    \[
        a_1 = \half,\ a_2 = \frac{1}{4} \frac{1}{2},\ 
        \Rightarrow s_2 = a_1 + 2(2)a_2 = \half + \half = 1.
    \]

    We prove by induction $s_n = 1, \forall n\ge 2,$ with the base case verified.
    \[
        s_{n+1} = a_1 + a_2 + \cdots + a_{n} + 2(n+1)a_{n+1} = s_{n} + (1-2n)a_n + 2(n+1)a_{n+1}
        = s_{n} + (1-2n)a_n + (2n-1)a_n = s_n = 1.
    \]

    Therefore the hypothesis is true, thus
    \[
        a_1 + a_2 + \cdots + a_n  = s_n - (2n-1)a_n < 1.
    \]
\end{soln}

\begin{problem}[Problem Twenty Four]
    Show that for any positive integer $n$
    \[
        \sum_{r=1}^{n} \frac{1}{r} \binom{n}{r} = \sum_{r=1}^{n} \frac{2^r-1}{r} \quad (*)
    \]
\end{problem}

\begin{soln}
    For $n=1,$ both sides become 1, thus true.

    Assuming that it is true for $n \ge 1,$ then
    \[
        \begin{aligned}
            \sum_{r=1}^{n+1} \frac{1}{r} \binom{n+1}{r} - \sum_{r=1}^{n} \frac{1}{r} \binom{n}{r}
            &= \sum_{r=1}^{n} \frac{1}{r} \left( \binom{n+1}{r} - \binom{n}{r} \right) + \frac{1}{n+1} \binom{n+1}{n+1}\\
            &= \sum_{r=1}^{n} \frac{1}{r} \left( \binom{n}{r-1} \right) + \frac{1}{n+1}
            = \frac{1}{n+1} \left( \sum_{r=1}^{n}  \binom{n+1}{r} \right) + \frac{1}{n+1}\\
            &= \frac{1}{n+1} \left( \sum_{r=0}^{n}  \binom{n+1}{r}  -2 + 1\right)
            = \frac{2^{n+1} -1}{n+1} 
        \end{aligned}
    \]

    Thus 
    \[
        \sum_{r=1}^{n+1} \frac{1}{r} \binom{n+1}{r} = \sum_{r=1}^{n} \frac{1}{r} \binom{n}{r} + \frac{2^{n+1} -1}{n+1} 
        = \sum_{r=1}^{n+1} \frac{2^r-1}{r} 
    \]
\end{soln}

\begin{problem}[Problem Twenty Five]
    \label{problem:23-24-s5-o-p23}
    Find all functions $f:\ \ZZ \rightarrow \RR$ such that:
    \[
        f(1) = \frac{5}{2}, f(m)f(n) = f(m+n) + f(m-n),\ \forall m,n \in \ZZ.
    \]
\end{problem}

\begin{soln}
    We prove that $f(n) = 2^n + 2^{-n},\ \forall n \ge 0 \quad (*)$
    First 
    \[
        f(0)f(1) = 2f(1) \Rightarrow f(0) = 2
    \]

    Assuming that (*) is true for all $k \le n,$ then
    \[
        f(n)f(1) = f(n+1) + f(n-1) \Rightarrow f(n) = f(n)f(1) - f(n-1) = 2^{n+1} + 2^{-(n+1)}
    \]

    For $n$ negative, note that $f(0)f(n) = f(n) + f(-n) \Rightarrow f(-n) = f(n),$ in other words $f$ is an even function, thus
    $f(-n) = f(n) = 2^n + 2^{-n},\ \forall n \in \ZZ.$
\end{soln}

\begin{problem}[Problem Twenty Six]
    Let $n \ge 1$ be a positive integer and let $x_1, x_2, \ldots, x_n$ be real numbers such that
    $0 \le x_n \le x_{n-1} \le x_2 \le x_1.$ Let
    \[
        \begin{aligned}
            &s_n = x_1 - x_2 + \cdots + (-1)^n x_{n-1} + (-1)^{n+1}x_n\\
            &S_n = x_1^2 - x_2^2 + \cdots + (-1)^n x_{n-1}^2 + (-1)^{n+1}x_n^2
        \end{aligned}
    \]

    Prove that $s_n^2 \le S_n.$
\end{problem}

\begin{soln}
    First we prove that 
    \begin{claim*}
        $s_n \ge 0,$ for all $n \ge 1.$
    \end{claim*}
    \begin{subproof}
        For the base case $s_1 = x_1 \ge 0.$ Assume that $s_n \ge 0.$
        
        \textit{Case 1:} $n$ is even
        \[
            s_n \ge 0, x_{n+1} \ge 0 \Rightarrow s_{n+1} = s_n + (-1)^{n+2}x_{n+1} = s_n + x_{n+1} \ge 0.
        \]

        \textit{Case 2:} $n$ is odd
        \[
            s_n \ge 0, x_{n} \ge x_{n+1} \Rightarrow s_{n+1} = s_{n-1} + (-1)^{n+1}x_{n-1} + (-1)^{n+2}x_{n+1} = s_n + x_{n} - x_{n+1} \ge 0.
        \]
    
        Thus $s_n \ge 0,\ \forall n\ge 1.$
    \end{subproof}
    
    Now we proof the problem statement by induction.

    For the base case $n=1,$ $S_1 = s_1^2 \ge s_1^2.$ For $n=2$
    \[
        S_2 = x_1^2 - x_2^2 = (x_1-x_2)(x_1 + x_2) \ge (x_1-x_2)^2 = s_2^2.
    \]

    Let assume that the hypothesis is true for $n,$ then
    
    \textit{Case 1:} $n=2k$
    \[
        \begin{aligned}
            &s_{2k+2}^2 = (s_{2k+1} - x_{2k+2})^2
            = s_{2k+1}^2 - 2 s_{2k+1}x_{2k+2}  + x_{2k+2}^2\\
            &\le S_{2k+1} - x_{2k+2}^2 + 2 x_{2k+2}(s_{2k+1} -x_{2k+2})
            = S_{2k+2} - 2s_{2k+2}x_{2k+2}\\
            &\le S_{2k+2}.
        \end{aligned}
    \]

    \textit{Case 2:} $n=2k+1$
    \[
        \begin{aligned}
            &s_{2k+3}^2 = (s_{2k+1} - x_{2k+2} + x_{2k+3})^2
            = s_{2k+1}^2 - 2 s_{2k+1}(x_{2k+2} - x{2k+3}) + (x_{2k+2} - x_{2k+3})^2\\
            &\le S_{2k+1} - x_{2k+2}^2 + x_{2k+3}^2 - 2 s_{2k+1}(x_{2k+2} - x{2k+3}) + 2x_{2k+2}^2 - 2x_{2k+2}x_{2k+3}\\
            &= S_{2k+1} - x_{2k+2}^2 + x_{2k+3}^2 - 2(s_{2k+1} - x_{2k+2})(x_{2k+2} - x{2k+3})\\
            &= S_{2k+3} - 2s_{2k+2}(x_{2k+2} - x{2k+3})
            \le S_{2k+3}
        \end{aligned}
    \]
\end{soln}

\begin{problem}[Problem Twenty Seven]
    Let $n \ge 1$ be a non-negative integer. Prove that for all real number $x$,
    \[
        \sum_{k=0}^{n} | \sin{(2^k x)} | \le 1 + \frac{\sqrt{3}}{2} n.
    \]
\end{problem}

\begin{soln}
    We prove that
    \begin{claim*}
        \[ 
            2|\sin{x}| + |\sin{2x}| \le \frac{3\sqrt{3}}{2}.
        \]
    \end{claim*}
    \begin{subproof}
        \[
            \begin{aligned}
                2|\sin{x}| + |\sin{2x}| 
                &= 2|\sin{x}|(1+|\cos{x}|)\\
                &= 2\sqrt{\left(1-(|\cos{x}|)^2\right)\left(1+|\cos{x}| \right)^2}
                = 2\sqrt{\left(1-|\cos{x}|\right)\left(1+|\cos{x}| \right)^3}\\
                & = \frac{2}{\sqrt{3}} \sqrt{3(1-|\cos{x}|)(1+|\cos{x}|)(1+|\cos{x}|)(1+|\cos{x}|)}\\
                &\le \frac{2}{\sqrt{3}} \sqrt{\left( \frac{3(1-|\cos{x}|)+ (1+|\cos{x}|) + (1+|\cos{x}|) + (1+|\cos{x}|)}{4} \right)^4}\\
                &= \frac{2}{\sqrt{3}} \left(\frac{3}{2}\right)^2 = \frac{3\sqrt{3}}{2}
            \end{aligned}
        \]
    \end{subproof}

    For the base case for $n=1,$
    \[
        \begin{aligned}
            &\sum_{k=0}^{n} | \sin{(2^k x)} | = \left(\frac{2}{3}|\sin{x}| + \frac{1}{3}|\sin{2x}|\right) +  \left(\frac{1}{3}|\sin{x}| + \frac{2}{3}|\sin{2x}|\right)
            \le \frac{\sqrt{3}}{2} + \left(\frac{1}{3} + \frac{2}{3}\right) = 1 + \frac{\sqrt{3}}{2} n.
        \end{aligned}
    \]

    Furthermore, from the base case, by replacing $x$ with $2^n x,$
    \[
        \frac{2}{3}|\sin{2^n x}| + \frac{1}{3}|\sin{2^{n+1} x}| \le \frac{\sqrt{3}}{2}.
    \]

    Now, let's assume that the inequality stands for $n-1$,
    \[
        \begin{aligned}
            &\sum_{k=0}^{n-1} | \sin{(2^k x)} | + \frac{2}{3}|\sin{2^n x}| + \frac{1}{3}|\sin{2^{n+1} x}| \le \frac{\sqrt{3}}{2}(n-1) + \frac{\sqrt{3}}{2}\\
            &\Rightarrow \sum_{k=0}^{n+1} | \sin{(2^k x)} | \le \frac{\sqrt{3}}{2}(n) + \frac{1}{3}|\sin{2^n x}| + \frac{2}{3}|\sin{2^{n+1} x}| \le \frac{\sqrt{3}}{2}(n+1) 
        \end{aligned}
    \]
\end{soln}

\end{document}