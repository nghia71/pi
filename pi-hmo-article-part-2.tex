\documentclass{article}

\usepackage[main=english,vietnamese]{babel}
\usepackage[T1]{fontenc}
\usepackage[utf8]{inputenc}
\usepackage[sexy]{evan}
\usepackage{matchsticks}
\usepackage{wrapfig}
\usepackage{listings}

\begin{otherlanguage*}{vietnamese}
\title{Các Kỳ Thi Toán Học Tại Hungary\\ \quad \\Phần 2 - Các kỳ thi toán học cấp quốc gia}
\end{otherlanguage*}

\author{Nghia Doan}
\date{\today}

\begin{document}

\begin{otherlanguage*}{vietnamese}

\maketitle

\section{Kỳ thi Toán học Zrínyi Ilona}

Kỳ thi Toán học Zrínyi Ilona (Zrínyi Ilona Matematikaverseny \cite{Zrinyi}) là một trong những cuộc thi toán quan trọng nhất tại Hungary,
do Quỹ MATEGYE \cite{MATEGYE} tổ chức hàng năm.
Mục tiêu chính của kỳ thi là phổ biến toán học, phát triển tư duy logic và kỹ năng giải quyết vấn đề cho học sinh từ lớp 2 đến lớp 12.
Cuộc thi không chỉ tạo điều kiện cho học sinh thử sức mà còn giúp chuẩn bị cho các kỳ thi toán học chuyên sâu hơn.

\textbf{Cấu trúc và thể lệ}

Kỳ thi gồm ba cấp độ: cấp tiểu học (lớp 2-8) thi chung một hạng mục, còn cấp trung học phổ thông (lớp 9-12) chia thành hai hạng mục là gimnázium (trung học phổ thông)
và technikum (trung học kỹ thuật).

Thời gian làm bài được quy định theo cấp lớp. Học sinh lớp 2-4 có 60 phút để hoàn thành 25 câu hỏi, lớp 5-6 có 75 phút để làm 25 câu, và lớp 7-12 có 90 phút để giải 30 câu hỏi.
Các bài thi đều ở dạng trắc nghiệm với 5 phương án trả lời (A, B, C, D, E), trong đó chỉ có một đáp án đúng.

Nội dung câu hỏi bao gồm nhiều lĩnh vực khác nhau trong toán học. Đối với học sinh nhỏ tuổi, bài thi tập trung vào số học, hình học cơ bản và tư duy logic.
Ở cấp độ cao hơn, đề thi có sự kết hợp giữa số học, đại số, hình học, tổ hợp và xác suất, yêu cầu học sinh vận dụng kiến thức linh hoạt và suy luận chặt chẽ.
Một số dạng bài phổ biến trong kỳ thi gồm:
\begin{itemize}[topsep=0pt, partopsep=0pt, itemsep=0pt]
    \ii Số học và lý thuyết số, bao gồm phép toán, ước số, bội số, số nguyên tố và các bài toán chia hết.
    \ii Đại số, với biểu thức, phương trình, bất đẳng thức và quy luật số học.
    \ii Hình học, bao gồm tính diện tích, chu vi, thể tích, quan hệ góc và các bài toán hình học phẳng.
    \ii Tổ hợp và xác suất, liên quan đến đếm, hoán vị, tổ hợp và các bài toán tư duy logic.
    \ii Ứng dụng toán học, với các bài toán về tốc độ, thời gian, tài chính và bài toán thực tế.
\end{itemize}

Hệ thống điểm số được tính theo công thức $4H - R + F$ hoặc $5H + U$, trong đó $H$ (helyes) là số câu trả lời đúng, $R$ (rossz) là số câu sai, $F$ (feladatok) là số câu hỏi,
và $U$ (üres) là số câu không trả lời. Nếu một câu hỏi không được trả lời, thí sinh không bị trừ điểm, nhưng nếu tô hai hoặc nhiều đáp án hoặc có dấu hiệu chỉnh sửa sai quy định,
câu trả lời đó sẽ bị tính là sai.

Thí sinh không được sử dụng bất kỳ công cụ hỗ trợ nào ngoài bút viết và giấy nháp do ban tổ chức cung cấp. Chỉ phiếu trả lời trắc nghiệm được thu lại để chấm điểm.

\textbf{Quy trình và vòng thi}

Kỳ thi diễn ra qua ba vòng với độ khó tăng dần:
\begin{enumerate}[topsep=0pt, partopsep=0pt, itemsep=0pt]
    \ii Vòng 1 tổ chức tại trường học của thí sinh thường vào cuối tháng 11. Các bài thi của học sinh sẽ được chấm tự động, và kết quả công bố vào ngày 09/12/2024.
    \ii Vòng 2 diễn ra thường vào cuối tháng 2 năm sau, với bài thi gồm số lượng câu hỏi tương tự vòng 1 nhưng độ khó cao hơn.
    Chỉ những học sinh đạt tối thiểu 50\% số điểm tối đa hoặc nằm trong nhóm có thành tích cao nhất tại trường mới đủ điều kiện vào vòng 2.
    Danh sách thí sinh đủ điều kiện vào vòng này được công bố vào cuối tháng 12.
    \ii Vòng chung kết toàn quốc được tổ chức thường vào cuối tháng 4, nơi những thí sinh xuất sắc nhất của từng khu vực tranh tài.
    Số suất tham dự vòng chung kết phụ thuộc vào tổng số thí sinh tham gia từ mỗi khu vực.
    Nếu một khu vực có từ 50 đến 149 thí sinh, một thí sinh được chọn vào chung kết; với 150-249 thí sinh sẽ có hai suất, 250-349 thí sinh có ba suất,
    và nếu có từ 350 thí sinh trở lên sẽ có bốn suất.
    Vòng chung kết có độ khó cao hơn hẳn hai vòng trước, đòi hỏi tư duy logic và kỹ năng phân tích bài toán phức tạp.
    Điểm đặc biệt của vòng này là có sự kết hợp giữa các câu hỏi trắc nghiệm và phần thi tự luận nhằm đánh giá tư duy chiến lược của thí sinh.
\end{enumerate}

\textbf{Đánh giá và trao giải}

Kết quả được chấm riêng theo từng cấp lớp và hạng mục thi. Nếu có hai thí sinh có cùng số điểm, thứ tự ưu tiên sẽ dựa trên tiêu chí số câu trả lời sai ít hơn.
Nếu vẫn bằng nhau, hệ thống sẽ xét đến mức độ khó của các bài mà thí sinh giải đúng; bài nào được nhiều thí sinh làm đúng hơn sẽ có trọng số thấp hơn,
còn bài hiếm người giải đúng hơn sẽ có trọng số cao hơn. Nếu vẫn bằng nhau, thí sinh sẽ được xếp đồng hạng.

Những thí sinh đạt thành tích cao nhất trong vòng 2 sẽ nhận chứng nhận và phần thưởng hiện vật.
Ở vòng chung kết, mười thí sinh xuất sắc nhất mỗi cấp lớp sẽ nhận huy chương và giải thưởng đặc biệt, trong khi các thí sinh khác sẽ nhận chứng nhận thành tích.
Lễ trao giải và kết quả cuối cùng sẽ được công bố ngay sau buổi thi.

\textbf{Ý nghĩa và tầm quan trọng}

Kỳ thi Toán học Zrínyi Ilona không chỉ là một cuộc thi mà còn là cơ hội để học sinh thử thách bản thân,
rèn luyện tư duy logic và chuẩn bị cho các kỳ thi toán học tầm cỡ.
Hàng năm, cuộc thi thu hút hàng chục nghìn học sinh tham gia và là một trong những sự kiện toán học quan trọng nhất tại Hungary.
Không chỉ vinh danh những thí sinh xuất sắc, cuộc thi còn góp phần xây dựng nền tảng toán học vững chắc cho thế hệ trẻ,
khuyến khích học sinh Hungary phát triển niềm đam mê với môn toán học.

\bigbreak

\noindent\rule{16.5cm}{0.4pt}

\textbf{Một số bài thi tiêu biểu}

\bigbreak

\begin{problem*}[Ví dụ mẫu, lớp 8, bài 30 \cite{zrinyi_8_o}]
    Anna đã tạo một hình 67-giác lồi từ giấy. Bea cắt hình này thành hai phần bằng một đường thẳng, sau đó tiếp tục cắt một trong các phần nhận được bằng một đường thẳng khác.
    Quá trình này được lặp lại cho đến khi thu được 8 $n-$giác. Giá trị của $n$ là bao nhiêu?
    \[
        (A) \quad 11 \qquad
        (B) \quad 12 \qquad
        (C) \quad 13 \qquad
        (D) \quad 14 \qquad
        (E) \quad 15
    \]
\end{problem*}

\begin{problem*}[Ví dụ mẫu, lớp 10, bài 29 \cite{zrinyi_10_o}]
    András và Balázs khởi hành cùng lúc từ thành phố A đến thành phố B bằng cách đi bộ. András đi nhanh hơn Balázs, mỗi km mất ít hơn Balázs 5 phút.
    Sau khi đi được một phần năm quãng đường, András quay lại thành phố A, nghỉ 10 phút, rồi tiếp tục hành trình đến thành phố B, nơi anh đến cùng lúc với Balázs.
    Khoảng cách giữa hai thành phố A và B là bao nhiêu km, nếu Balázs mất 2,5 giờ để đi hết quãng đường đó?
    \[
        (A) \quad 8 \qquad
        (B) \quad 10 \qquad
        (C) \quad 15 \qquad
        (D) \quad 16 \qquad
        (E) \quad 20
    \]
\end{problem*}

\begin{problem*}[Ví dụ mẫu, lớp 12, bài 28 \cite{zrinyi_12_o}]
    Có bao nhiêu cách để đọc được từ GORDIUSZ từ sơ đồ nếu ta chỉ có thể di chuyển sang phải hoặc xuống dưới, và không được đi cùng một hướng quá hai lần liên tiếp?
    \begin{center}
        \begin{tabular}{|c|ccccccc}
        \hline
        G & \multicolumn{1}{c|}{O} & \multicolumn{1}{c|}{R} & \multicolumn{1}{c|}{D} & \multicolumn{1}{c|}{I} & \multicolumn{1}{c|}{U} & \multicolumn{1}{c|}{S} & \multicolumn{1}{c|}{Z} \\ \hline
        O & \multicolumn{1}{c|}{R} & \multicolumn{1}{c|}{D} & \multicolumn{1}{c|}{I} & \multicolumn{1}{c|}{U} & \multicolumn{1}{c|}{S} & \multicolumn{1}{c|}{Z} &                        \\ \cline{1-7}
        R & \multicolumn{1}{c|}{D} & \multicolumn{1}{c|}{I} & \multicolumn{1}{c|}{U} & \multicolumn{1}{c|}{S} & \multicolumn{1}{c|}{Z} &                        &                        \\ \cline{1-6}
        D & \multicolumn{1}{c|}{I} & \multicolumn{1}{c|}{U} & \multicolumn{1}{c|}{S} & \multicolumn{1}{c|}{Z} &                        &                        &                        \\ \cline{1-5}
        I & \multicolumn{1}{c|}{U} & \multicolumn{1}{c|}{S} & \multicolumn{1}{c|}{Z} &                        &                        &                        &                        \\ \cline{1-4}
        U & \multicolumn{1}{c|}{S} & \multicolumn{1}{c|}{Z} &                        &                        &                        &                        &                        \\ \cline{1-3}
        S & \multicolumn{1}{c|}{Z} &                        &                        &                        &                        &                        &                        \\ \cline{1-2}
        Z &                        &                        &                        &                        &                        &                        &                        \\ \cline{1-1}
        \end{tabular}
    \end{center}    
    \[
        (A) \quad 8 \qquad
        (B) \quad 32 \qquad
        (C) \quad 42 \qquad
        (D) \quad 100 \qquad
        (E) \quad 128
    \]
\end{problem*}

\begin{remark*}
    Đáp án (không kèm lời giải) tại \cite{zrinyi_mo}
\end{remark*}

\newpage

\section{Kỳ thi Toán học Quốc gia Varga Tamás}

Kỳ thi Toán học Varga Tamás (Varga Tamás Országos Matematikaverseny \cite{VargaTamas}) là một trong những cuộc thi toán quan trọng nhất dành cho học sinh trung học cơ sở tại Hungary,
được tổ chức hàng năm nhằm phát hiện và nuôi dưỡng tài năng toán học. Kỳ thi mang tên Varga Tamás (1919-1987), một nhà sư phạm xuất sắc có nhiều đóng góp cho giáo dục toán học.
Được khởi xướng vào năm 1987 bởi các giáo viên từ ELTE Trefort Utcai Gimnázium và Fazekas Mihály Gimnázium,
kỳ thi ban đầu mang tính thử nghiệm và chính thức trở thành một phần quan trọng của hệ thống giáo dục Hungary từ năm học 1990-1991.

Mô hình ba vòng thi được định hình từ những năm đầu tiên, với sự điều chỉnh để phù hợp với hệ thống giáo dục thay đổi.
Từ năm 1993, khi các trường trung học 6 năm và 8 năm xuất hiện, kỳ thi bắt đầu phân chia thành hai hạng mục:
một dành cho học sinh có số giờ học toán tiêu chuẩn và một dành cho học sinh có chương trình toán nâng cao.
Năm 2001, kỳ thi được chuyển giao cho Viện Dịch vụ Sư phạm và Chuyên môn Tỉnh Fejér (Fejér Megyei Pedagógiai Szakszolgálati és Szakmai Intézet \cite{FMPDSzSzI}) để tổ chức và quản lý.

\textbf{Cấu trúc và thể lệ}

Kỳ thi dành riêng cho học sinh lớp 7 và 8, với hai hạng mục: nhóm tiêu chuẩn (học tối đa 4 giờ toán mỗi tuần) và nhóm nâng cao (học hơn 4 giờ toán mỗi tuần).
Cả hai nhóm đều thi chung ở vòng đầu, nhưng từ vòng hai sẽ được xét tách biệt.

Kỳ thi gồm ba vòng với độ khó tăng dần:
\begin{enumerate}[topsep=0pt, partopsep=0pt, itemsep=0pt]
    \ii Vòng 1 (trường học): Tổ chức vào tháng 10 hoặc 11 tại các trường đăng ký. Thí sinh làm bài thi trong 90 phút, gồm 5-6 bài toán tự luận, với mức độ từ cơ bản đến nâng cao.
    Bài thi không chỉ kiểm tra kiến thức chương trình phổ thông mà còn đánh giá tư duy logic, kỹ năng lập luận và khả năng sáng tạo của học sinh.
    \ii Vòng 2 (khu vực/megyei): Tổ chức vào tháng 1, dành cho những thí sinh đạt tối thiểu 50\% số điểm tối đa ở vòng 1.
    Ở vòng này, học sinh được phân vào hai hạng mục dựa trên số giờ học toán hàng tuần. Bài thi kéo dài 120 phút, với 6-8 bài toán tự luận, đòi hỏi khả năng suy luận cao hơn,
    bao gồm các chủ đề như số học, đại số, hình học và tổ hợp.
    \ii Vòng chung kết (toàn quốc): Diễn ra vào tháng 3 hoặc 4, dành cho khoảng 50 thí sinh xuất sắc nhất trong hạng mục nâng cao và 100 thí sinh từ hạng mục tiêu chuẩn.
    Bài thi vòng này kéo dài 150 phút, gồm 8-10 bài toán, với các câu hỏi thử thách hơn, yêu cầu thí sinh đưa ra lời giải chi tiết và chặt chẽ.
    Các bài toán thường có sự kết hợp của nhiều lĩnh vực toán học, từ số học, hình học, tổ hợp đến bất đẳng thức và phương trình hàm.
\end{enumerate}

Điểm số của từng bài toán thường dao động từ 4 đến 10 điểm, tùy vào độ khó và mức độ chi tiết của lời giải.
Nếu lời giải có sai sót nhỏ nhưng vẫn thể hiện tư duy đúng, thí sinh có thể nhận điểm một phần.

\textbf{Đánh giá và trao giải}

Bài thi vòng 1 được chấm bởi giáo viên của trường, vòng 2 do hội đồng khu vực chấm,
và vòng chung kết do hội đồng giám khảo quốc gia gồm các chuyên gia từ Hội Toán học Bolyai János (Bolyai János Matematikai Társulat - BJMT \cite{BJMT}) đảm nhiệm.

Điểm số được tính dựa trên số lượng bài giải đúng và mức độ hoàn chỉnh của lời giải. Nếu có hai thí sinh đạt cùng số điểm, thứ hạng sẽ được quyết định dựa trên:
\begin{enumerate}[topsep=0pt, partopsep=0pt, itemsep=0pt]
    \ii Số bài giải đúng hoàn chỉnh nhiều hơn.
    \ii Điểm số đạt được trong các bài khó nhất.
    \ii Tổng điểm ưu tiên, dựa trên số lượng thí sinh giải được mỗi bài (bài càng khó, điểm ưu tiên càng cao).
    Nếu vẫn bằng nhau, thí sinh sẽ đồng hạng.
\end{enumerate}

Những thí sinh xuất sắc tại vòng 2 sẽ nhận chứng nhận và phần thưởng hiện vật. Ở vòng chung kết, ba thí sinh hàng đầu mỗi hạng mục nhận huy chương và giải thưởng đặc biệt.
Các thí sinh còn lại được trao chứng nhận thành tích.

\textbf{Ý nghĩa và ảnh hưởng}

Từ khi ra đời, kỳ thi đã trở thành bệ phóng cho nhiều tài năng toán học Hungary. Mỗi năm, hàng nghìn học sinh tham gia vòng đầu tiên,
và nhiều em sau này đã gặt hái thành công tại các cuộc thi toán quốc tế, bao gồm Olympic Toán Quốc tế (IMO).
Cuộc thi không chỉ giúp phát hiện tài năng mà còn góp phần nâng cao chất lượng dạy và học toán trên toàn quốc.

Với truyền thống lâu đời và uy tín, kỳ thi Toán học Varga Tamás tiếp tục đóng vai trò quan trọng trong hệ thống giáo dục Hungary,
khuyến khích học sinh rèn luyện tư duy logic, khám phá vẻ đẹp của toán học và chuẩn bị cho những thử thách lớn hơn trong tương lai.

\bigbreak

\noindent\rule{16.5cm}{0.4pt}

\textbf{Một số bài thi tiêu biểu}

\bigbreak

\begin{problem*}[2024-2025, lớp 7, nhóm 2, vòng cấp tỉnh, bài 3 \cite{vt_7_o_2}]
    Bốn tam giác đều bằng nhau có các đỉnh được gán các số 1, 2, 3 sao cho mỗi số xuất hiện đúng một lần trong mỗi tam giác.
    Sau đó, bốn tam giác này được ghép lại để tạo thành một tam giác lớn theo sơ đồ.
    Các số tại các đỉnh trùng nhau được cộng lại, và kết quả được ghi tại trung điểm của ba cạnh của tam giác lớn. Đáng chú ý là ba trung điểm này nhận cùng một tổng.
    \begin{center}
        \includegraphics[width=3cm]{varga-tamas-2425-2f-7o-2k-3.pdf}
    \end{center}
    Tổng đó có thể là bao nhiêu?
\end{problem*}

\begin{problem*}[2024-2025, lớp 8, nhóm 1, vòng cấp tỉnh, bài 3 \cite{vt_8_o_1}]
    Cho hình chữ nhật \( ABCD \) có cạnh \( AB \) dài gấp ba lần cạnh \( BC \). Gọi \( E \) là điểm chia ba cạnh \( AB \), gần \( A \) hơn, và \( F \) là trung điểm của cạnh \( AD \).
    Biết diện tích tam giác \( \triangle EFC \) là \( 16 \) cm\(^2\), hãy xác định độ dài của cạnh \( AB \).
\end{problem*}

\begin{problem*}[2024-2025, lớp 8, nhóm 2, vòng cấp tỉnh, bài 5 \cite{vt_8_o_2}]
    Xét một đường tròn và chọn \( 2n \) điểm trên đó (\( n \geq 3 \)) sao cho độ dài các cung giữa hai điểm kề nhau chỉ có ba giá trị khác nhau
    và không có hai cung liên tiếp nào có cùng độ dài. Các điểm này được tô màu luân phiên đỏ và xanh, sao cho có \( n \) điểm đỏ và \( n \) điểm xanh.  
    
    Chứng minh rằng đa giác \( n \) cạnh tạo bởi các điểm xanh và đa giác \( n \) cạnh tạo bởi các điểm đỏ có diện tích bằng nhau và chu vi bằng nhau!
\end{problem*}

\begin{remark*}
    Đề và lời giải cùng nằm trong các tệp tại các đường dẫn tại \cite{vt_7_o_2}, \cite{vt_8_o_1}, và \cite{vt_8_o_2}.
\end{remark*}

\newpage

\section{Kỳ thi Toán học Arany Dániel}

Kỳ thi Toán học Arany Dániel (Arany Dániel Matematikai Tanulóverseny \cite{AranyDaniel}) mang tên Arany Dániel (1863-1945), một giáo viên toán học người Hungary,
người đã sáng lập Tạp chí Toán học Trung học Hungary (Középiskolai Matematikai Lapok - KöMaL \cite{KoMaL}) vào năm 1893.
Cuộc thi khởi nguồn từ Kỳ thi Toán Trung học Toàn quốc, tổ chức lần đầu vào năm 1947 và chính thức mang tên Arany Dániel từ năm 1951.
Từ năm 1977, Kỳ thi Toán học Kalmár László, dành cho học sinh lớp 9 và 10 tại Hungary, được nhập chung vào kỳ thi này.

\textbf{Hệ thống thi và thể lệ}

Kỳ thi Arany Dániel được chia thành hai cấp độ: KEZDŐK (Người mới bắt đầu) và HALADÓK (Người nâng cao).
Cấp độ KEZDŐK dành cho học sinh lớp 9 và 10, được chia thành ba nhóm theo số giờ học toán bắt buộc mỗi tuần.
Trong khi đó, cấp độ HALADÓK chỉ dành cho học sinh lớp 10 và cũng được phân thành ba nhóm tùy thuộc vào chương trình học toán chuyên sâu hay không.

\textbf{Cấu trúc và quy trình thi}

Kỳ thi diễn ra qua ba vòng thi, với độ khó tăng dần.
\begin{enumerate}[topsep=0pt, partopsep=0pt, itemsep=0pt]
    \ii Vòng đầu tiên, còn gọi là vòng trường (iskolai forduló), diễn ra ngay tại trường của thí sinh. Đề thi gồm 4-5 bài toán tự luận, tùy theo cấp độ của thí sinh.
    Mỗi bài có điểm số khác nhau, thường dao động từ 5 đến 10 điểm, tùy vào độ khó. Các bài toán chủ yếu thuộc các lĩnh vực số học, đại số, hình học và tổ hợp,
    với yêu cầu giải thích chi tiết lời giải. Giáo viên toán tại trường sẽ chấm bài dựa trên hướng dẫn chính thức, và những thí sinh đạt đủ số điểm yêu cầu sẽ được ghi danh vào vòng tiếp theo.
    \ii Vòng hai, còn gọi là vòng khu vực/quốc gia (második forduló), thử thách thí sinh với 5 bài toán có độ khó cao hơn. Đề thi có cấu trúc tương tự vòng đầu,
    nhưng các bài toán yêu cầu phương pháp giải sáng tạo và lập luận chặt chẽ hơn. Điểm tối đa cho mỗi bài có thể lên đến 15 điểm.
    Sau khi hoàn thành bài thi, tất cả bài làm sẽ được gửi về Bolyai János Matematikai Társulat,
    nơi hội đồng giám khảo cấp quốc gia chấm điểm và lựa chọn những thí sinh xuất sắc nhất vào vòng chung kết.
    \ii Vòng chung kết (Döntő) diễn ra tại một địa điểm tập trung, nơi các thí sinh có thành tích cao nhất trong vòng hai tranh tài.
    Ở vòng này, thí sinh phải giải 3 bài toán có độ phức tạp cao, mỗi bài thường có thang điểm từ 10 đến 20 điểm, tùy vào mức độ thử thách.
    Các bài toán không chỉ yêu cầu tư duy toán học sâu sắc mà còn đòi hỏi khả năng trình bày lập luận chặt chẽ. Hội đồng giám khảo cấp quốc gia sẽ chấm bài và công bố kết quả cuối cùng.
\end{enumerate}

\textbf{Quy định và công cụ được phép sử dụng}

Thời gian làm bài ở mỗi vòng thi là 240 phút. Trong suốt kỳ thi, thí sinh không được sử dụng máy tính, điện thoại hay bất kỳ thiết bị điện tử nào.
Tuy nhiên, các em được phép mang theo sách giáo khoa, tài liệu toán học in sẵn, bảng công thức và sổ tay toán học để tham khảo.

\textbf{Ý nghĩa của kỳ thi}

Kỳ thi Arany Dániel không chỉ là một cuộc thi toán thông thường mà còn có ý nghĩa giáo dục sâu sắc.

Trước hết, cuộc thi giúp phát hiện và bồi dưỡng tài năng toán học trẻ. Đây là cơ hội để học sinh trung học thử sức với những bài toán mang tính thách thức cao,
phát triển tư duy logic và sáng tạo. Những thí sinh xuất sắc thường tiếp tục tham gia các kỳ thi lớn hơn như OKTV (Országos Középiskolai Tanulmányi Verseny – Kỳ thi Học Sinh Giỏi Toán Quốc gia Hungary).

Bên cạnh đó, kỳ thi góp phần thúc đẩy tinh thần cạnh tranh lành mạnh, giúp học sinh làm quen với áp lực thi cử, rèn luyện kỹ năng quản lý thời gian và tư duy chiến lược.
Nhiều thí sinh đạt thành tích cao trong kỳ thi này sau đó tham gia các kỳ thi quốc tế.

Về mặt định hướng nghề nghiệp, kỳ thi đóng vai trò là một bước đệm quan trọng cho những học sinh có nguyện vọng theo đuổi các ngành khoa học – công nghệ.
Nhiều thí sinh từng đạt giải trong kỳ thi này sau đó đã trở thành nhà toán học, kỹ sư, nhà khoa học và giảng viên đại học.

Ngoài ra, kỳ thi còn tạo ra một sân chơi trí tuệ, khuyến khích học sinh yêu thích toán học.
Ngay cả những học sinh không theo đuổi toán chuyên sâu vẫn có thể tận hưởng niềm vui khi khám phá kiến thức và rèn luyện tư duy logic.
Không chỉ mang tính thi đấu, kỳ thi còn giúp học sinh trải nghiệm “vẻ đẹp” của toán học thông qua những bài toán thú vị và các phương pháp giải sáng tạo.

\bigbreak

\noindent\rule{16.5cm}{0.4pt}

\textbf{Một số bài thi tiêu biểu}

\bigbreak

\begin{problem*}[2023-2024, nhập môn, nhóm I-II, vòng 2, bài 1 \cite{AD_2024}]
    Ádám tính tổng:  
    \[
        10^{2023} + 10^{2018} + 10^{2013} + \dots + 10^{18} + 10^{13} + 10^8 + 10^3
    \]
    và viết ra kết quả. Hỏi Ádám đã viết bao nhiêu chữ số \( 0 \)?
\end{problem*}

\begin{problem*}[2023-2024, nâng cao, nhóm II, vòng 1, bài 2 \cite{AD_2024}]
    Giải hệ phương trình nghiệm nguyên:
    \[
        \left\{
            \begin{array}{rcl}
                x^2 - y^2 - z^2 &=& 1\\
                -x+y+z &=& -3
            \end{array}
        \right.
    \]
\end{problem*}

\begin{problem*}[2023-2024, nâng cao, nhóm III, vòng 2, bài 3 \cite{AD_2024}]
    Có thể chọn bốn số thực sao cho thỏa mãn cả bốn điều kiện sau đây không?  
    \begin{enumerate}[topsep=0pt, partopsep=0pt, itemsep=0pt]
        \ii Tồn tại ba số trong đó có tổng và tích đều là số hữu tỷ.  
        \ii Tồn tại ba số trong đó có tổng là số hữu tỷ nhưng tích là số vô tỷ.  
        \ii Tồn tại ba số trong đó có tổng là số vô tỷ nhưng tích là số hữu tỷ.
        \ii Tồn tại ba số trong đó có tổng và tích đều là số vô tỷ.
    \end{enumerate}
\end{problem*}

\begin{remark*}
    Đề và lời giải cùng nằm trong tệp tại đường dẫn tại \cite{AD_2024}.
\end{remark*}

\newpage

\section{Kỳ thi Toán học Quốc tế dành cho học sinh nói tiếng Hungary (NMMV)}

Kỳ thi Toán học Quốc tế dành cho học sinh nói tiếng Hungary (Nemzetközi Magyar Matematikaverseny - NMMV \cite{NMMV})
là một sự kiện học thuật thường niên dành cho học sinh các quốc gia nơi có giảng dạy toán học bằng tiếng Hungary.
Kỳ thi không chỉ là một sân chơi trí tuệ mà còn là cơ hội để học sinh giao lưu,
trao đổi kinh nghiệm và củng cố tinh thần đoàn kết giữa các cộng đồng người Hungary trên thế giới.

\textbf{Lịch sử và mục tiêu}

Ý tưởng về kỳ thi NMMV xuất phát từ thầy giáo toán Mihály Bencze từ Brașov (Romania) vào năm 1991 tại Hội nghị Rátz László ở Szeged.
Với sự nỗ lực của thầy György Oláh, kỳ thi đầu tiên đã được tổ chức và từ đó trở thành sự kiện thường niên.
NMMV được tổ chức luân phiên giữa Hungary và các quốc gia khác như Romania, Slovakia, Ukraina và các nước Nam Tư cũ.

Mục tiêu chính của cuộc thi là tạo ra một môi trường chung cho học sinh nói tiếng Hungary có cơ hội gặp gỡ, thử sức với các bài toán đầy thử thách,
từ đó phát triển tư duy logic, kỹ năng giải quyết vấn đề và làm giàu kiến thức toán học.

\textbf{Cấu trúc và thể lệ cuộc thi}

Kỳ thi NMMV được tổ chức theo hình thức thi cá nhân và chia thành ba nhóm theo độ tuổi:
\begin{enumerate}[topsep=0pt, partopsep=0pt, itemsep=0pt]
    \ii Nhóm A: Học sinh lớp 11-12
    \ii Nhóm B: Học sinh lớp 9-10
    \ii Nhóm C: Học sinh lớp 7-8
\end{enumerate}

Mỗi nhóm sẽ thi trong 4 giờ, với 4 bài toán tự luận. Đề thi yêu cầu học sinh không chỉ vận dụng kiến thức cơ bản mà còn cần tư duy sáng tạo và khả năng lập luận chặt chẽ.

Các bài toán trong kỳ thi được thiết kế để phản ánh tinh thần toán học Hungary, với sự cân bằng giữa các chủ đề:
\begin{itemize}[topsep=0pt, partopsep=0pt, itemsep=0pt]
    \ii Số học: Chứng minh chia hết, số nguyên tố, dãy số.
    \ii Đại số: Phương trình, bất phương trình, đa thức.
    \ii Hình học: Chứng minh hình học, tính diện tích, thể tích, góc.
    \ii Tổ hợp: Xác suất, hoán vị, đếm, chiến lược tối ưu.
\end{itemize}

\textbf{Quy trình đánh giá và trao giải}

Bài thi được chấm theo thang điểm 0-7 cho mỗi bài toán, với tổng điểm tối đa là 28 điểm.
Ban giám khảo đánh giá dựa trên mức độ chính xác của lời giải, cách lập luận và cách trình bày của thí sinh.

Dựa trên tổng điểm, thí sinh được xếp hạng và trao các giải thưởng.
Ngoài giải cá nhân, giải thưởng Urbán János được trao hàng năm cho một thí sinh xuất sắc từ Hungary và một thí sinh từ nước ngoài,
nhằm vinh danh những học sinh có thành tích đặc biệt xuất sắc trong kỳ thi.

\textbf{Tầm quan trọng và ý nghĩa của kỳ thi}

NMMV không chỉ là một cuộc thi toán học mà còn là một cầu nối văn hóa, giúp học sinh từ các quốc gia khác nhau có cơ hội gặp gỡ, học hỏi lẫn nhau và cùng nhau phát triển trong lĩnh vực toán học.
Cuộc thi góp phần giữ gìn và phát huy truyền thống giáo dục toán học xuất sắc của Hungary, đồng thời giúp học sinh chuẩn bị cho các kỳ thi toán học quốc gia và quốc tế.

Với lịch sử hơn ba thập kỷ phát triển, NMMV tiếp tục là một sân chơi trí tuệ uy tín, tạo cơ hội cho học sinh không chỉ thể hiện năng lực toán học mà còn giao lưu,
kết nối và phát triển tư duy logic trên một nền tảng giáo dục vững chắc.

\bigbreak

\noindent\rule{16.5cm}{0.4pt}

\textbf{Một số bài thi tiêu biểu}

\bigbreak

\begin{problem*}[2023-2024, lớp 9, bài 4 \cite{nmmv_9o}]
    Giải phương hệ phương trình sau:
    \[
        \left\{
            \begin{array}{rcl}
                |x| + |y| + z &=& 2021\\
                |x| + y + |z| &=& 2023\\
                x + |y| + |z| &=& 2025\\
            \end{array}
        \right.
    \]
    và viết ra kết quả. Hỏi Ádám đã viết bao nhiêu chữ số \( 0 \)?
\end{problem*}

\begin{problem*}[2023-2024, lớp 10, bài 4 \cite{nmmv_10o}]
    Chứng minh rằng, với mọi cách đặt 2023 điểm trong một hình vuông \( 241 \times 241 \),
    luôn có thể tìm được một hình vuông con kích thước \( 200 \times 200 \) sao cho nó chứa ít nhất 675 điểm.
\end{problem*}

\begin{problem*}[2023-2024, lớp 12, bài 3 \cite{nmmv_12o}]
    Xét tam giác vuông \( ABC \) có đường tròn nội tiếp và đường tròn ngoại tiếp có bán kính lần lượt là \( r \) và \( R \). Gọi \( CD \) là đường cao ứng với cạnh huyền \( AB \).  
    
    Dựng hình vuông \( CEFG \) có cạnh bằng \( CD \), trong đó các đỉnh \( E \) và \( G \) lần lượt nằm trên các đoạn \( AC \) và \( BC \).
    Gọi \( T \) là diện tích phần giao nhau của tam giác \( ABC \) và hình vuông \( CEFG \), còn \( t \) là phần diện tích của hình vuông \( CEFG \) không bị tam giác \( ABC \) che phủ.  
    \begin{enumerate}[topsep=0pt, partopsep=0pt, itemsep=0pt]
        \ii Chứng minh rằng:
        \[
            \frac{t}{T} = \frac{r}{2R}.
        \]
        \ii Giá trị \( \frac{t}{T} \) có thể thay đổi trong khoảng nào?
    \end{enumerate}
\end{problem*}

\begin{remark*}
    Lời giải nằm trong các tệp tại các đường dẫn tại \cite{nmmv_9o_mo}, \cite{nmmv_10o_mo}, và \cite{nmmv_12o_mo}.
\end{remark*}

\newpage

\section{Kỳ thi Học sinh Giỏi Quốc gia (OKTV)}

Kỳ thi Học Sinh Giỏi Toán Quốc gia Hungary (Országos Középiskolai Tanulmányi Verseny - OKTV \cite{OKTV}) 
là một trong những cuộc thi học thuật quan trọng nhất dành cho học sinh trung học tại Hungary.
Kỳ thi này không chỉ đánh giá năng lực học tập mà còn giúp học sinh giành được lợi thế trong quá trình tuyển sinh đại học.

\textbf{Điều kiện tham gia}

OKTV dành cho học sinh trung học tại Hungary, bao gồm cả học sinh chính thức và khách.
Học sinh phải theo học hệ chính quy và đang học ở năm cuối hoặc gần cuối cấp theo chương trình của trường.
Để đủ điều kiện, học sinh phải học hoặc đã hoàn thành môn thi theo chương trình giảng dạy, hoặc đã thi đỗ kỳ thi phân loại của môn đó.
Đối với các môn ngoại ngữ, những học sinh có ngôn ngữ thi là tiếng mẹ đẻ hoặc đã sống một khoảng thời gian dài tại quốc gia sử dụng ngôn ngữ đó sẽ không được tham gia.

Mỗi học sinh có thể đăng ký nhiều môn nhưng chỉ có thể thi một hạng mục trong mỗi môn. Quy trình đăng ký được thực hiện thông qua trường học và phải hoàn tất thời hạn trong tháng 9.
Sau khi học sinh đăng ký, hiệu trưởng hoặc người được ủy quyền sẽ gửi thông tin lên hệ thống OKTV trước thời hạn trong tháng 9.

\textbf{Cấu trúc kỳ thi}

OKTV bao gồm hai hoặc ba vòng thi, tùy theo từng môn học. Đối với các môn có hai vòng, học sinh sẽ thi vòng đầu tại trường, nơi giáo viên sẽ chấm bài theo hướng dẫn chấm điểm của trung ương.
Những thí sinh xuất sắc nhất sẽ vào vòng chung kết, nơi đề thi khó hơn và kiểm tra toàn diện kiến thức cũng như khả năng tư duy logic.

Những môn như Toán, Vật lý, Hóa học, Tin học, Sinh học, Ngữ văn và Lịch sử có ba vòng thi.
\begin{enumerate}[topsep=0pt, partopsep=0pt, itemsep=0pt]
    \ii Vòng đầu tiên được tổ chức tại trường, nơi học sinh làm bài theo đề thi trung tâm trong khoảng 120 đến 180 phút.
    \ii Những học sinh có kết quả cao nhất sẽ tiếp tục vào vòng hai, diễn ra tại các trung tâm thi cấp khu vực với thời gian làm bài kéo dài từ 180 đến 240 phút.
    Vòng này có mức độ khó cao hơn và có thể bao gồm các bài nghiên cứu hoặc phân tích chuyên sâu.
    \ii Những thí sinh xuất sắc nhất sẽ lọt vào vòng chung kết, được tổ chức tập trung tại Budapest hoặc các trung tâm thi lớn khác.
    Đề thi vòng chung kết thường yêu cầu khả năng tư duy sáng tạo, vận dụng kiến thức một cách linh hoạt và giải quyết các bài toán hoặc tình huống phức tạp.
\end{enumerate}

\textbf{Hình thức thi}

Tùy theo môn học, kỳ thi có thể bao gồm thi viết, thi nói hoặc thi thực hành.

Trong các môn khoa học và xã hội, học sinh sẽ làm bài thi viết với thời gian từ 3 đến 4 giờ.
Môn Toán, Vật lý, Hóa học thường có 5 đến 7 bài toán phức tạp, đòi hỏi tư duy logic và kỹ năng giải quyết vấn đề.
Ngữ văn và Lịch sử có phần viết luận, trong đó học sinh phải phân tích tài liệu hoặc trình bày lập luận một cách thuyết phục.

Các môn ngoại ngữ yêu cầu học sinh thực hiện bài kiểm tra nghe - nói, trong khi các môn nghệ thuật hoặc sân khấu có thể yêu cầu thí sinh trình bày một phần trình diễn hoặc phân tích tác phẩm.
Đối với các môn khoa học ứng dụng như Tin học, Sinh học, Hóa học, học sinh có thể phải thực hiện các bài thực hành, thí nghiệm hoặc lập trình trực tiếp.

Mặc dù thí sinh không được sử dụng máy tính cá nhân, điện thoại hoặc các thiết bị điện tử, họ vẫn có thể mang theo tài liệu tham khảo bằng giấy,
bao gồm sách giáo khoa và bảng công thức nếu được phép.

\textbf{Chấm điểm và xếp hạng}

Hệ thống chấm điểm của kỳ thi OKTV được thiết kế để đảm bảo tính khách quan và minh bạch. Trong vòng đầu tiên, bài thi được chấm bởi giáo viên trường theo hướng dẫn từ Bộ Giáo dục.
Từ vòng hai trở đi, bài thi được chấm bởi hội đồng giám khảo quốc gia theo thang điểm chi tiết. Đối với các bài luận hoặc bài toán tự luận, mỗi bài thi sẽ được ít nhất hai giám khảo chấm độc lập.
Nếu có sự chênh lệch đáng kể, bài thi sẽ được chấm lại bởi giám khảo thứ ba.

Sau vòng chung kết, 30 học sinh có kết quả cao nhất sẽ được xếp hạng theo ba nhóm: Nhóm 1-10 gồm những học sinh xuất sắc nhất, Nhóm 11-20 và Nhóm 21-30.
Nếu có nhiều thí sinh có cùng số điểm, thứ hạng sẽ được quyết định dựa trên điểm từ các vòng trước. Những học sinh trong top 3 sẽ nhận được chứng nhận danh dự từ Bộ Giáo dục.

Ngoài danh hiệu và giải thưởng, kết quả OKTV có thể mang lại lợi thế lớn trong tuyển sinh đại học.
Nhiều trường đại học hàng đầu tại Hungary, như Đại học Eötvös Loránd (ELTE), Đại học BME, Đại học Szeged,
công nhận thành tích OKTV như một phần tiêu chí xét tuyển trực tiếp hoặc cộng điểm ưu tiên.
Đặc biệt, các thí sinh đạt thứ hạng cao trong môn Toán, Lý, Hóa, Sinh, Tin học có cơ hội được tuyển thẳng vào các chương trình danh giá.

\textbf{Ý nghĩa của kỳ thi}

OKTV không chỉ là một kỳ thi đánh giá năng lực học thuật mà còn là một sân chơi thử thách, giúp học sinh phát triển tư duy, trau dồi kỹ năng nghiên cứu và chuẩn bị cho tương lai.
Đây là một cơ hội quan trọng cho những ai mong muốn khẳng định bản thân và giành được lợi thế trong hành trình học thuật và sự nghiệp sau này.

\bigbreak

\noindent\rule{16.5cm}{0.4pt}

\textbf{Một số bài thi tiêu biểu}

\bigbreak

\begin{problem*}[2023-2024, nhóm I, vòng chung kết, bài 1 \cite{mat1_flap_d}]
    Cho một cấp số cộng \( (a_n) \) không phải là hằng số (\( n \in \mathbb{N}^+ \)) và một hàm số được xác định trên tập số thực bởi quy tắc:
    \[
        f(x) = x^3 + a_{4} x^2 + a_{20} x + a.
    \]
    
    Biết rằng các nghiệm của phương trình \( f(x) = 0 \) là \( a_9, a_{10} \) và \( a_{11} \). Hãy tìm số hạng đầu tiên và công sai của cấp số cộng này.
\end{problem*}

\begin{problem*}[2023-2024, nhóm II, vòng chung kết, bài 3 \cite{mat2_flap_d}]
    Xét một tứ giác nội tiếp \( ABCD \) có độ dài các cạnh là \( AB = a \), \( BC = b \), \( CD = c \), và \( DA = d \).  
    
    Ký hiệu \( r_a \) là bán kính của đường tròn tiếp xúc ngoài với cạnh \( AB \) tại một điểm bên trong tứ giác,
    đồng thời tiếp xúc với hai đường thẳng chứa các cạnh \( BC \) và \( AD \). Tương tự, định nghĩa các bán kính \( r_b \), \( r_c \), và \( r_d \).  
    
    Chứng minh rằng:
    \[
        \frac{1}{r_a} + \frac{1}{r_b} + \frac{1}{r_c} + \frac{1}{r_d} \geq \frac{4}{\sqrt{abcd}}.
    \]
\end{problem*}

\begin{problem*}[2023-2024, nhóm III, vòng chung kết, bài 3 \cite{mat3_flap_d}]
    Cho \( k \) và \( \ell \) là các số nguyên dương, dãy số nguyên dương \( a_1 \leq a_2 \leq \dots \leq a_k \) và \( b_1 \leq b_2 \leq \dots \leq b_\ell \).
    \( q(x) \) là một đa thức có hệ số nguyên và bậc ít nhất là 1.  
    
    Giả sử rằng với mọi số nguyên dương \( n \), số $a_1^n + a_2^n + \dots + a_k^n + q(n)$ luôn là ước của số $b_1^n + b_2^n + \dots + b_\ell^n + q(n).$
    
    Chứng minh rằng \( k = \ell \) và \( a_i = b_i \) với mọi \( 1 \leq i \leq k \).
\end{problem*}

\begin{remark*}
    Lời giải nằm trong các tệp tại các đường dẫn tại \cite{mat1_javut_d}, \cite{mat2_javut_d}, và \cite{mat3_javut_d}.
\end{remark*}

\newpage

\section*{Tham khảo}

\begin{thebibliography}{99}
    \bibitem{Zrinyi} Zrínyi Ilona Matematikaverseny, \url{http://www.mategye.hu/?pid=zrinyi_verseny}
    \bibitem{MATEGYE} Matematikában Tehetséges Gyermekekért, \url{http://www.mategye.hu}
    \bibitem{zrinyi_8_o} \url{http://www.mategye.hu/download/zrinyi/minta_feladatsor/1/8_osztaly.pdf}
    \bibitem{zrinyi_10_o} \url{http://www.mategye.hu/download/zrinyi/minta_feladatsor/1/10_osztaly.pdf}
    \bibitem{zrinyi_12_o} \url{http://www.mategye.hu/download/zrinyi/minta_feladatsor/1/12_osztaly.pdf}
    \bibitem{zrinyi_mo} \url{http://www.mategye.hu/download/zrinyi/minta_feladatsor/2/megoldokulcs.pdf}
    \bibitem{VargaTamas} Varga Tamás Országos Matematikaverseny, \url{http://www.mategye.hu/?pid=vargatamasverseny}
    \bibitem{BJMT} Bolyai János Matematikai Társulat, \url{https://www.bolyai.hu}
    \bibitem{FMPDSzSzI} Fejér Megyei Pedagógiai Szakszolgálati és Szakmai Intézet, \url{https://fejermepsz.hu}
    \bibitem{vt_7_o_2} \url{http://www.mategye.hu/download/varga/2425/2_fordulo_7_o_2_kat_javito.pdf}
    \bibitem{vt_8_o_1} \url{http://www.mategye.hu/download/varga/2425/2_fordulo_8_o_1_kat_javito.pdf}
    \bibitem{vt_8_o_2} \url{http://www.mategye.hu/download/varga/2425/2_fordulo_8_o_2_kat_javito.pdf}
    \bibitem{AranyDaniel} Arany Dániel Matematikai Tanulóverseny, \url{https://www.bolyai.hu/versenyek-arany-daniel-matematikaverseny/}
    \bibitem{KoMaL} A KöMaL pontversenyei, \url{https://www.komal.hu/verseny.h.shtml}
    \bibitem{AD_2024} \url{https://www.bolyai.hu/files/AD_2024_feladatok_megoldasok.pdf}
    \bibitem{NMMV} Nemzetközi Magyar Matematikaverseny, \url{https://nmmv.berzsenyi.hu/főoldal}
    \bibitem{nmmv_9o} \url{https://drive.google.com/file/d/1HFmjtRUJ9tcZiCyNaIpCEXhACR-tcV0h/view}
    \bibitem{nmmv_9o_mo} \url{https://drive.google.com/file/d/1ZtY-2SJg9puYShE26yi4J7F3uVIvqI_2/view}
    \bibitem{nmmv_10o} \url{https://drive.google.com/file/d/1tJ87_Ko4PW2HSyxvIVRQXDqe3AW8y2vF/view}
    \bibitem{nmmv_10o_mo} \url{https://drive.google.com/file/d/1-_QkD5-I0cmUXHjqaFGbQSVTijGhJ0yo/view}
    \bibitem{nmmv_12o} \url{https://drive.google.com/file/d/13-Nf-nRMxw11_Zn2C0Sh5qgVJkBWvsVS/view}
    \bibitem{nmmv_12o_mo} \url{https://drive.google.com/file/d/1O3vnJ2x2wN6IK6NnBWmh9saE9gNT16QF/view}
    \bibitem{OKTV} OKTV Versenyfeladatok,\\ \url{https://www.oktatas.hu/kozneveles/tanulmanyi_versenyek_/oktv_nyito}
    \bibitem{mat1_flap_d} \url{https://www.oktatas.hu/pub_bin/dload/kozoktatas/tanulmanyi_versenyek/oktv/oktv2023_2024_donto/mat1_flap_d_oktv_2324.pdf}
    \bibitem{mat1_javut_d} \url{https://www.oktatas.hu/pub_bin/dload/kozoktatas/tanulmanyi_versenyek/oktv/oktv2023_2024_donto/mat1_javut_d_oktv_2324.pdf}
    \bibitem{mat2_flap_d} \url{https://www.oktatas.hu/pub_bin/dload/kozoktatas/tanulmanyi_versenyek/oktv/oktv2023_2024_donto/mat2_flap_d_oktv_2324.pdf}
    \bibitem{mat2_javut_d} \url{https://www.oktatas.hu/pub_bin/dload/kozoktatas/tanulmanyi_versenyek/oktv/oktv2023_2024_donto/mat2_javut_d_oktv_2324.pdf}
    \bibitem{mat3_flap_d} \url{https://www.oktatas.hu/pub_bin/dload/kozoktatas/tanulmanyi_versenyek/oktv/oktv2023_2024_donto/mat3_flap_d_oktv_2324.pdf}
    \bibitem{mat3_javut_d} \url{https://www.oktatas.hu/pub_bin/dload/kozoktatas/tanulmanyi_versenyek/oktv/oktv2023_2024_donto/mat3_javut_d_oktv_2324.pdf}
\end{thebibliography}

\end{otherlanguage*}

\end{document}