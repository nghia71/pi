\documentclass{article}

\usepackage[main=english,vietnamese]{babel}
\usepackage[T1]{fontenc}
\usepackage[utf8]{inputenc}
\usepackage[sexy]{evan}
\usepackage{matchsticks}
\usepackage{wrapfig}
\usepackage{listings}

\begin{otherlanguage*}{vietnamese}
    \title{Các Kỳ Thi Toán Học Tại Hungary\\ \quad \\Phần 3 - Các kỳ thi để lựa chọn các đội tuyển quốc gia}
\end{otherlanguage*}

\author{Nghia Doan}
\date{\today}

\begin{document}

\begin{otherlanguage*}{vietnamese}

\maketitle

\section{Középiskolai Matematikai Lapok - Tạp chí Toán học Trung học Hungary}

Tạp chí Toán học Trung học Hungary (Középiskolai Matematikai Lapok - KöMaL \cite{KoMaL}) là một trong những tạp chí toán học dành cho học sinh trung học lâu đời nhất trên thế giới,
được thành lập vào năm 1893 bởi Arany Dániel. Trong hơn 130 năm hoạt động, KöMaL đã góp phần nuôi dưỡng nhiều thế hệ tài năng toán học Hungary,
nhiều người trong số đó đã đạt thành tích cao tại các kỳ thi toán quốc tế và trở thành những nhà toán học hàng đầu thế giới.

\textbf{Hệ thống bài toán và điểm số}

Một trong những nét đặc trưng của KöMaL là hệ thống bài toán dành cho học sinh giải và gửi lời giải về ban biên tập để được chấm điểm.
Hàng năm, tạp chí xuất bản khoảng 300 bài toán, chia thành nhiều cấp độ khó khác nhau,
từ bài toán cơ bản dành cho học sinh mới tiếp cận đến những bài toán nâng cao thách thức cả sinh viên đại học.

Hệ thống bài toán được chia thành các mục sau:
\begin{itemize}[topsep=0pt, partopsep=0pt, itemsep=0pt]
    \ii Bài toán cơ bản (K/B): Dành cho học sinh trung học cơ sở (K) và trung học phổ thông (B), thường liên quan đến số học, phương trình, bất đẳng thức và tổ hợp.
    \ii Bài toán nâng cao (A): Có độ khó tương đương với bài toán trong kỳ thi IMO, dành cho những học sinh xuất sắc.
    \ii Bài toán hình học (C): Tập trung vào các chủ đề hình học cổ điển, như đường tròn, tam giác, đa giác và tọa độ.
    \ii Bài toán về tư duy logic (D): Các bài toán đòi hỏi suy luận sáng tạo và cách tiếp cận mới.
    \ii Bài toán vật lý (P): Dành cho học sinh yêu thích vật lý, với các bài toán liên quan đến động lực học, điện từ học và cơ học lượng tử cơ bản.
\end{itemize}

Học sinh có thể gửi lời giải của mình qua hệ thống trực tuyến của KöMaL. Mỗi lời giải được giám khảo chấm điểm trên thang 10, với những bài toán khó có thể đạt tới 15-20 điểm.
Cuối năm, những học sinh đạt tổng điểm cao nhất được vinh danh trên tạp chí và có cơ hội nhận các giải thưởng danh giá.

\textbf{Ảnh hưởng và thống kê}

Theo thống kê, mỗi năm KöMaL thu hút khoảng 2000-3000 học sinh từ khắp Hungary và cả một số quốc gia khác. Trong số đó, hơn 500 học sinh thường xuyên gửi bài giải,
và khoảng 100 học sinh đạt thứ hạng cao nhất trong hệ thống xếp hạng của tạp chí. Tạp chí cũng có đóng góp đáng kể vào thành công của Hungary trong các kỳ thi toán quốc tế.

Nhiều nhà toán học Hungary nổi tiếng từng tham gia KöMaL khi còn là học sinh, trong đó có:
\begin{itemize}[topsep=0pt, partopsep=0pt, itemsep=0pt]
    \ii László Lovász, người đoạt Giải thưởng Abel 2021, từng giải bài trên KöMaL từ khi còn là học sinh.
    \ii Paul Erdős, một trong những nhà toán học có ảnh hưởng nhất thế kỷ 20, từng tham gia KöMaL trong thời niên thiếu.
    \ii Endre Szemerédi, người đoạt Giải thưởng Abel 2012, từng là độc giả trung thành của KöMaL và được truyền cảm hứng từ các bài toán của tạp chí.
\end{itemize}

\textbf{KöMaL trong thời đại kỹ thuật số}

Ngày nay, KöMaL vẫn giữ được vị trí quan trọng trong nền giáo dục toán học Hungary. Tạp chí hiện có phiên bản trực tuyến với hàng nghìn bài toán có thể truy cập miễn phí.
Ngoài ra, hệ thống nộp bài trực tuyến giúp học sinh từ khắp nơi trên thế giới có thể tham gia giải bài toán mà không bị giới hạn về địa lý.

Mặc dù trải qua hơn một thế kỷ tồn tại, KöMaL vẫn duy trì triết lý giáo dục toán học của mình: khuyến khích học sinh suy nghĩ sáng tạo,
giải quyết vấn đề một cách độc lập và khám phá vẻ đẹp của toán học. Nhờ vào KöMaL, Hungary tiếp tục duy trì vị thế là một trong những quốc gia có nền giáo dục toán học mạnh trên thế giới.

\bigbreak

\noindent\rule{16.5cm}{0.4pt}

\textbf{Một số bài thi tiêu biểu}

\bigbreak

\begin{problem*}[12/2024, nhóm C, bài 1833 \cite{2024_f}]
    (5 điểm) Giải hệ phương trình sau:
    \[
        \left\{
            \begin{array}{rcl}
                a + c &=& b\\
                a^3 - c &=& b^2\\
                a + b &=& c^3\\
            \end{array}
        \right.
    \]
    trong đó \( a, b, c \) là các số tự nhiên. 
\end{problem*}

\begin{problem*}[12/2024, nhóm B, bài 5424 \cite{2024_f}]
    (4 điểm) Với mọi số nguyên dương \( n \), ký hiệu \( K_n \) là hình nhận được bằng cách lấy một bảng ``cờ vua'' kích thước \( (2n) \times (2n) \),
    rồi cắt bỏ bốn hình vuông kích thước \( (n-1) \times (n-1) \) ở bốn góc, như trong hình minh họa bên dưới.
    \begin{center}
        \includegraphics[width=10cm]{komal-12-2024-b-5454.png}
    \end{center}
    
    Ký hiệu \( a_n \) là số cách lát kín \( K_n \) bằng các domino kích thước \( 2 \times 1 \) mà không có khe hở hay chồng lấn. (Ví dụ: \( a_1 = 2 \), \( a_2 = 8 \)).  
    Chứng minh rằng \( 2a_n \) luôn là một số chính phương với mọi \( n \).
\end{problem*}

\begin{problem*}[12/2024, nhóm A, bài 895 \cite{2024_f}]
    (7 điểm) Gọi một hàm \( f: \mathbb{R} \to \mathbb{R} \) là \textit{tuần hoàn yếu} nếu nó liên tục và thỏa mãn:
    \[
        f(x+1) = f(f(x)) + 1, \quad \forall x \in \mathbb{R}.
    \]
    
    \begin{enumerate}[topsep=0pt, partopsep=0pt, itemsep=0pt]
        \ii Có tồn tại một hàm tuần hoàn yếu thỏa mãn \( f(x) > x \) với mọi \( x \in \mathbb{R} \) không?  
        \ii Có tồn tại một hàm tuần hoàn yếu thỏa mãn \( f(x) < x \) với mọi \( x \in \mathbb{R} \) không?
    \end{enumerate}
\end{problem*}

\begin{remark*}
    Lời giải nằm tại các đường dẫn tại \cite{c_1833}, \cite{b_5424}, và \cite{a_895}.
\end{remark*}

\newpage

\section{Kỳ thi Toán học Kürschák József}

Kỳ thi Toán học Kürschák József (Kürschák József Matematikai Tanulóverseny \cite{Kurschak}) là một trong những cuộc thi toán lâu đời và danh giá nhất tại Hungary,
được tổ chức lần đầu tiên vào năm 1894 bởi Hội Toán học và Vật lý Hungary (tiền thân của Hội Toán học Bolyai János).
Cuộc thi ban đầu được lập ra nhằm vinh danh Bá tước Eötvös Loránd khi ông được bổ nhiệm làm Bộ trưởng Tôn giáo và Giáo dục Công cộng.
Trải qua nhiều giai đoạn phát triển và thay đổi, kỳ thi này trở thành một biểu tượng trong nền toán học Hungary, với nhiều nhà toán học nổi tiếng từng tham gia và đạt thành tích cao.

\textbf{Lịch sử và sự phát triển}

Trong những năm đầu, kỳ thi được gọi là “Kỳ thi Toán học và Vật lý của Hội” và chỉ dành cho học sinh đã tốt nghiệp trung học.
Sau này, nó được đổi tên thành “Kỳ thi Toán học Eötvös Loránd” để vinh danh nhà khoa học này sau khi ông qua đời.
Tuy nhiên, từ năm 1949, cuộc thi chính thức mang tên “Kỳ thi Toán học Kürschák József” nhằm ghi nhận đóng góp to lớn của Kürschák József,
một trong những nhà toán học có ảnh hưởng nhất tại Hungary.

Ban đầu, kỳ thi chỉ diễn ra tại Budapest và Kolozsvár, nhưng do những biến động lịch sử như Thế chiến thứ nhất,
Thế chiến thứ hai và cuộc cách mạng Hungary năm 1956, có một số năm cuộc thi không được tổ chức.
Sau năm 1947, Hội Toán học Bolyai János đã tái tổ chức kỳ thi và mở rộng phạm vi cho cả học sinh trung học.
Từ đó, kỳ thi trở thành một trong những tiêu chuẩn quan trọng để đánh giá và phát hiện tài năng toán học của học sinh Hungary.

\textbf{Cấu trúc và nội dung kỳ thi}

Kỳ thi Toán học Kürschák József diễn ra hàng năm vào khoảng tháng 10 hoặc tháng 11 và chỉ gồm một vòng duy nhất. Mỗi thí sinh có 4 giờ để giải quyết 3 bài toán.
Không giống như nhiều kỳ thi khác, thí sinh không được phép sử dụng bất kỳ tài liệu tham khảo hay công cụ hỗ trợ nào.
Mục tiêu của kỳ thi không chỉ là kiểm tra khả năng giải toán mà còn đánh giá tư duy sáng tạo, cách tiếp cận vấn đề và khả năng lập luận logic của thí sinh.

Các bài toán trong kỳ thi thường được thiết kế sao cho không đòi hỏi kiến thức vượt quá chương trình trung học nhưng lại yêu cầu khả năng tư duy linh hoạt và sáng tạo.
Các lĩnh vực chính thường xuất hiện trong đề thi bao gồm:
\begin{itemize}[topsep=0pt, partopsep=0pt, itemsep=0pt]
    \ii Số học và Lý thuyết số: Các bài toán liên quan đến tính chất số nguyên, chia hết, số chính phương, đồng dư, và các định lý cơ bản như định lý Fermat nhỏ, định lý Wilson.
    Ví dụ, một bài toán có thể yêu cầu chứng minh rằng một dãy số vô hạn có một số vô hạn các số nguyên tố.
    \ii Đại số: Các bài toán về đa thức, phương trình hàm, bất đẳng thức, hệ phương trình.
    Chẳng hạn, một bài toán có thể yêu cầu chứng minh một bất đẳng thức liên quan đến trung bình cộng và trung bình nhân.
    \ii Hình học: Đề thi thường có ít nhất một bài hình học thuần túy, yêu cầu chứng minh một tính chất của đường tròn, tam giác hoặc các phép biến đổi hình học.
    Một ví dụ điển hình là bài toán chứng minh rằng trong một đa giác lồi $n$ cạnh, có thể chọn tối đa $n$ đường chéo sao cho chúng đôi một cắt nhau.
    \ii Tổ hợp: Những bài toán đếm, nguyên lý Dirichlet, lý thuyết đồ thị, sắp xếp và bài toán chu trình Hamilton.
    Một bài toán trong kỳ thi năm 2007 yêu cầu thí sinh áp dụng thuật toán Gale-Shapley để tìm cách ghép đôi ổn định giữa nam và nữ dựa trên sở thích cá nhân.
\end{itemize}

\textbf{Cách chấm điểm và đánh giá}

Các bài làm được chấm dựa trên độ chính xác, sự chặt chẽ của lập luận, mức độ sáng tạo trong cách tiếp cận và trình bày.
Các bài toán không chỉ yêu cầu một đáp án đúng mà còn phải có một lời giải rõ ràng, hợp lý.
Những thí sinh có bài làm xuất sắc nhất sẽ được trao giải thưởng Kürschák, đây là một trong những danh hiệu danh giá nhất trong hệ thống các cuộc thi toán học của Hungary.

Một bài toán có thể được chấm tối đa 10 điểm, tùy thuộc vào mức độ hoàn chỉnh của lời giải.
Nếu một lời giải có hướng tiếp cận đúng nhưng chưa đầy đủ, thí sinh có thể được một phần điểm.
Những lời giải có tính sáng tạo đặc biệt, thể hiện cách nhìn nhận vấn đề mới mẻ thường được đánh giá cao.

Các thí sinh đạt kết quả cao trong kỳ thi thường tiếp tục tham gia các kỳ thi lớn hơn như Olympic Toán Quốc tế (IMO).
Đặc biệt, trong nhiều năm, những học sinh đạt giải cao trong cuộc thi này được miễn kỳ thi đầu vào đại học và được xét tuyển trực tiếp vào các chương trình toán học hàng đầu tại Hungary.

\textbf{Tầm quan trọng và ảnh hưởng}

Kỳ thi Toán học Kürschák József không chỉ giúp phát hiện các tài năng toán học mà còn đóng góp quan trọng vào việc nâng cao chất lượng giáo dục toán học tại Hungary.
Những bài toán được chọn lọc kỹ lưỡng qua từng năm đã trở thành nguồn tài liệu quý giá cho học sinh và giáo viên.

Các bài toán từ kỳ thi đã được tập hợp trong tuyển tập “Matematikai Versenytételek” do Kürschák khởi xướng.
Bộ sách này không chỉ chứa các đề thi mà còn kèm theo nhiều lời giải chi tiết, giúp học sinh hiểu sâu hơn về các phương pháp giải toán.
Tuyển tập này đã được dịch ra nhiều ngôn ngữ, bao gồm tiếng Anh, Nhật, Nga, Romania và Ba Tư, cho thấy tầm ảnh hưởng rộng rãi của cuộc thi.

Những người chiến thắng trong kỳ thi này đã trở thành những nhà toán học hàng đầu, đóng góp đáng kể cho nền toán học thế giới.
Trong số đó có nhiều tên tuổi lớn như Fejér Lipót, Kármán Tódor, Kőnig Dénes, Haar Alfréd và Teller Ede.

\bigbreak

\noindent\rule{16.5cm}{0.4pt}

\textbf{Một số bài thi tiêu biểu}

\bigbreak

\begin{problem*}[2024, bài 1 \cite{k_2024}]
    Cho tứ giác \( ABCD \) được chia thành một số tứ giác nội tiếp sao cho:
    \begin{itemize}[topsep=0pt, partopsep=0pt, itemsep=0pt]
        \ii Không có hai tứ giác nào trong phân hoạch này có điểm trong chung (điểm trong là điểm không nằm trên chu vi).
        \ii Các đỉnh của mỗi tứ giác trong phân hoạch không nằm trên bất kỳ cạnh nào của một tứ giác khác trong phân hoạch, cũng như không nằm trên cạnh nào của tứ giác \( ABCD \).
    \end{itemize}
    Chứng minh rằng \( ABCD \) cũng là một tứ giác nội tiếp.
\end{problem*}

\begin{problem*}[2024, bài 2 \cite{k_2024}]
    Vương quốc Một Chiều trong thời cổ đại nằm dọc theo một đường thẳng. Ban đầu không có thành phố nào. Các thành phố được thành lập lần lượt tại \( n \) vị trí khác nhau.  
    Từ thành phố thứ hai trở đi, mỗi thành phố mới thành lập sẽ được ghép đôi làm \textit{thành phố anh em} với thành phố gần nhất đã tồn tại (nếu có hai thành phố gần nhất,
    chọn thành phố được thành lập trước đó).  

    Trên bản đồ còn sót lại của vương quốc, chỉ hiển thị vị trí các thành phố và khoảng cách giữa chúng, nhưng không có thông tin về thứ tự thành lập.
    Các nhà sử học đang cố gắng xác định xem có thể suy ra từ bản đồ rằng mỗi thành phố có tối đa 41 thành phố anh em hay không.  

    \begin{enumerate}[topsep=0pt, partopsep=0pt, itemsep=0pt]
        \ii Với \( n = 10^6 \), hãy đưa ra một bản đồ sao cho có thể suy ra điều kiện trên.  
        \ii Chứng minh rằng với \( n = 10^{13} \), không có bản đồ nào có thể dẫn đến kết luận này.  
    \end{enumerate}
\end{problem*}

\begin{problem*}[2024, bài 3 \cite{k_2024}]
    Gọi \( p \) là một số nguyên tố và \( H \subseteq \{0, 1, \dots, p-1\} \) là một tập hợp không rỗng.
    Giả sử rằng với mỗi phần tử \( a \in H \), tồn tại hai phần tử khác \( a \), ký hiệu \( b, c \in H \), sao cho:
    \[
        b + c - 2a \equiv 0 \Mod{p}.
    \]
    
    Chứng minh rằng:
    \[
        p < 4k,
    \]
    trong đó \( k \) là số phần tử của tập \( H \).
\end{problem*}

\begin{remark*}
    Đề và lời giải cùng nằm trong tệp tại đường dẫn tại \cite{k_2024}.
\end{remark*}

\newpage

\section{Olympiad Toán học Trung Âu (MEMO) - Sân chơi quốc tế cho học sinh yêu toán}

Olympiad Toán học Trung Âu (Middle European Mathematical Olympiad - MEMO \cite{MEMO}) là một kỳ thi toán học quốc tế dành cho học sinh trung học đến từ các quốc gia Trung Âu.
Kỳ thi được tổ chức lần đầu tiên vào năm 2007, xuất phát từ kỳ thi Toán học Áo-Ba Lan,
với mục tiêu giúp học sinh rèn luyện và tích lũy kinh nghiệm thi đấu quốc tế trước khi tham gia các kỳ thi lớn hơn như Olympic Toán học Quốc tế (IMO).
Vì vậy, MEMO không chỉ là một cuộc thi mà còn là một bước đệm quan trọng trong sự nghiệp toán học của nhiều học sinh trong khu vực.

Kỳ thi diễn ra hàng năm vào tháng 8 hoặc tháng 9 với sự tham gia của 10 quốc gia: Áo, Ba Lan, Cộng hòa Séc, Croatia, Đức, Hungary, Litva, Slovakia, Slovenia và Thụy Sĩ.
Ngoài ra, nước chủ nhà có thể mời thêm một đội tuyển khách. Cuộc thi không mở đăng ký rộng rãi mà chỉ dành cho các đội tuyển được mời, và việc tham gia có thể đi kèm với phí đăng ký.

MEMO có cấu trúc đặc biệt so với các kỳ thi toán học khác, bao gồm cả phần thi cá nhân và phần thi đồng đội, nhấn mạnh không chỉ vào năng lực cá nhân mà còn vào khả năng làm việc nhóm.

\textbf{Cấu trúc kỳ thi và nội dung bài toán}

MEMO gồm hai phần thi: thi cá nhân và thi đồng đội.

Phần thi cá nhân diễn ra trong 5 tiếng, trong đó mỗi thí sinh phải giải 4 bài toán thuộc 4 lĩnh vực: đại số, tổ hợp, hình học và số học.
Các bài toán này có độ khó tương đương với IMO, đòi hỏi tư duy sáng tạo, kỹ năng lập luận chặt chẽ và khả năng trình bày khoa học.

Phần thi đồng đội cũng kéo dài 5 tiếng, trong đó mỗi đội 6 thành viên sẽ cùng nhau giải quyết 8 bài toán (2 bài thuộc mỗi lĩnh vực).
Phần thi này yêu cầu sự phối hợp hiệu quả giữa các thành viên, phân chia công việc hợp lý và tận dụng điểm mạnh của từng người để đạt được kết quả tối ưu.
Đây là điểm khác biệt đáng chú ý so với IMO, nơi phần thi cá nhân là trọng tâm chính.

\textbf{Lựa chọn bài toán và chấm điểm}

Các bài toán của MEMO được đề xuất bởi các quốc gia tham gia và được lựa chọn bởi ban tổ chức của nước chủ nhà.
Danh sách sơ bộ sẽ được gửi đến trưởng đoàn và phó trưởng đoàn ít nhất 3 tuần trước cuộc thi để đánh giá.
Quá trình chọn lọc dựa trên hai tiêu chí: độ hấp dẫn (thang điểm 1-3) và độ khó (thang điểm 1-5). Sau đó, hội đồng giám khảo sẽ họp để chọn bộ đề chính thức.

Bài thi được chấm theo thang điểm 7, giống như IMO. Điểm số được đánh giá dựa trên mức độ chính xác của lời giải, cách trình bày và phương pháp tiếp cận bài toán.
Những bài toán có cách giải sáng tạo hoặc lập luận chặt chẽ sẽ được đánh giá cao hơn.
Sau khi chấm điểm, kết quả sẽ được công bố và học sinh có thể theo dõi tiến trình của mình so với các thí sinh khác.

\textbf{Lịch sử hình thành và phát triển}

MEMO bắt nguồn từ kỳ thi Toán học Áo-Ba Lan, một cuộc thi song phương giữa hai nước.
Năm 2007, kỳ thi này mở rộng quy mô với sự tham gia của Cộng hòa Séc, Croatia, Thụy Sĩ, Slovakia và Slovenia, tạo tiền đề cho sự ra đời của MEMO.
Sau đó, năm 2008, Đức và Hungary gia nhập, và đến năm 2009, Litva trở thành thành viên thứ 10.

Kể từ đó, kỳ thi được tổ chức luân phiên giữa các quốc gia thành viên, với mỗi nước đăng cai một lần.
Việc tổ chức không chỉ nhằm tìm kiếm những tài năng toán học xuất sắc mà còn thúc đẩy sự hợp tác giáo dục giữa các nước Trung Âu trong lĩnh vực toán học.

\textbf{Ý nghĩa và vai trò của MEMO}

MEMO không chỉ là một cuộc thi mà còn là cơ hội giúp học sinh phát triển tư duy toán học và rèn luyện khả năng giải quyết vấn đề.
Việc tham gia kỳ thi giúp học sinh làm quen với môi trường thi đấu quốc tế, nâng cao kỹ năng tư duy sáng tạo và trình bày toán học.

Ngoài ra, kỳ thi cũng tạo ra một cộng đồng toán học quốc tế, nơi học sinh có thể giao lưu, học hỏi từ bạn bè quốc tế và mở rộng tầm nhìn về toán học.
Không ít học sinh từng tham gia MEMO đã tiếp tục theo đuổi sự nghiệp nghiên cứu toán học hoặc đạt thành tích cao trong các kỳ thi quốc tế lớn hơn.

\bigbreak

\noindent\rule{16.5cm}{0.4pt}

\textbf{Một số bài thi tiêu biểu}

\bigbreak

\begin{problem*}[2024, bài 1 \cite{m_2024}]
    Xác định tất cả các số \( k \in \mathbb{N}_0 \) sao cho tồn tại hàm số \( f: \mathbb{N}_0 \to \mathbb{N}_0 \) thoả mãn:
    \[
        f(2024) = k,\ f(f(n)) \leq f(n+1) - f(n), \quad \forall n \in \mathbb{N}_0.
    \]
   
    \textit{Ở đây, \( \mathbb{N}_0 \) là tập hợp tất các các số nguyên không âm.}
\end{problem*}

\begin{problem*}[2024, bài 2 \cite{m_2024}]
    Trên một bảng đen vô hạn có một tờ giấy. Marvin bí mật chọn một đa giác lồi \( P \) có 2024 đỉnh nằm hoàn toàn trong tờ giấy. Tigerin muốn xác định các đỉnh của \( P \).  
    
    Trong mỗi bước, Tigerin có thể vẽ một đường thẳng \( g \) trên bảng đen, sao cho đường thẳng này nằm hoàn toàn bên ngoài tờ giấy.
    Marvin sau đó phản hồi bằng cách vẽ đường thẳng \( h \), song song với \( g \) và là đường gần nhất với \( g \) mà đi qua ít nhất một đỉnh của \( P \).  
    
    Chứng minh rằng tồn tại một số nguyên dương \( n \) sao cho Tigerin luôn có thể xác định được các đỉnh của \( P \) trong nhiều nhất \( n \) bước.
\end{problem*}

\begin{problem*}[2024, bài 3 \cite{m_2024}]
    Cho tam giác nhọn \( ABC \) không cân. Chọn một đường tròn \( \omega \) đi qua \( B \) và \( C \),
    cắt các đoạn thẳng \( AB \) và \( AC \) lần lượt tại \( D \neq A \) và \( E \neq A \). Gọi \( F \) là giao điểm của \( BE \) và \( CD \).  
    
    Gọi \( G \) là điểm trên đường tròn ngoại tiếp tam giác \( ABF \) sao cho \( GB \) là tiếp tuyến của \( \omega \).
    Tương tự, gọi \( H \) là điểm trên đường tròn ngoại tiếp tam giác \( ACF \) sao cho \( HC \) là tiếp tuyến của \( \omega \).  
    
    Chứng minh rằng tồn tại một điểm \( T \neq A \), không phụ thuộc vào cách chọn \( \omega \), sao cho đường tròn ngoại tiếp tam giác \( AGH \) luôn đi qua \( T \).
\end{problem*}

\begin{remark*}
    Lời giải các bài trên có thể tìm thấy tại: \cite{m_2024_s}.
\end{remark*}

\newpage

\section*{Tham khảo}

\begin{thebibliography}{99}
    \bibitem{KoMaL} A KöMaL pontversenyei, \url{https://www.komal.hu/verseny.h.shtml}
    \bibitem{2024_f} \url{https://www.komal.hu/feladat?a=honap&h=202412&t=mat&l=hu}
    \bibitem{c_1833} \url{https://www.komal.hu/feladat?a=feladat&f=C1833&l=hu}
    \bibitem{b_5424} \url{https://www.komal.hu/feladat?a=feladat&f=B5424&l=hu}
    \bibitem{a_895} \url{https://www.komal.hu/feladat?a=feladat&f=A895&l=hu}
    \bibitem{Kurschak} Kürschák József Matematikai Tanulóverseny,\\ \url{https://www.bolyai.hu/versenyek-kurschak-jozsef-matematikai-tanuloverseny/}
    \bibitem{k_2024} \url{https://www.bolyai.hu/files/Kurschak_2024_megoldasok.pdf}
    \bibitem{MEMO} Middle European Mathematical Olimpiad, \url{https://www.memo-official.org/MEMO/}
    \bibitem{m_2024} \url{https://www.bolyai.hu/files/MEMO_2024_I_en.pdf}
    \bibitem{m_2024_s} \url{https://www.bolyai.hu/files/MEMO-2024-SolutionBooklet-2.pdf}
\end{thebibliography}

\end{otherlanguage*}

\end{document}