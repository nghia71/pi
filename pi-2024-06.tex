\documentclass{article}

\usepackage[main=english,vietnamese]{babel}
\usepackage[T1]{fontenc}
\usepackage[utf8]{inputenc}
\usepackage[sexy]{evan}
\usepackage{matchsticks}
\usepackage{wrapfig}
\usepackage{listings}

\newtheorem{hint}{Hint}

\title{Solving Forty Two Problems by the Induction Principle - Part IV}
\author{Nghia Doan}
\date{\today}

\begin{document}

\maketitle

\begin{problem}[Problem Sixteen]
    Prove that for $n \ge 1,$ the number $n!$ has the following property: to any divisor of $n!,$ except $n!,$
    we can add another divisor of $n!$ so that their sum is a divisor of $n!.$
\end{problem}

\begin{soln}
    Consider an induction based on $n \ge 2$ The case $n=2$ is obvious $1+1 = 2!.$
    
    Now, for the inductive step. 
    
    Let $k \mid (n+1)!,$ then there exist $a,b$ so that $k = ab,$ where $a \mid n!, b \mid n+1.$

    \textit{Case 1:} $a <n!,$ then by induction hypothesis,
    \[
        \exists c \mid n!,\ a+c \mid n! \Rightarrow k+cb = ab+cb =(a+c)b \mid n!(n+1) = (n+1)!.
    \]

    \textit{Case 2:} $a =n!,$ then $b < n+1.$
    
    \textit{Case 2a:} $n+1$ is composite, then
    \[
        \begin{aligned}
            &\exists c > 1,\ n+1 = bc \Rightarrow c < n \Rightarrow \frac{n!}{c} \mid n! \Rightarrow \exists d,\  \frac{n!}{c}+d \mid n!\\
            &\Rightarrow k + (n+1)d = n!b + (n+1)d = \frac{(n+1)!}{c} + (n+1)d  =  \left( \frac{n!}{c} + d \right) (n+1) \mid n!(n+1) = (n+1)!
        \end{aligned}
    \]
\end{soln}

\begin{problem}[Problem Seventeen]
    On a circular route, there are $n$ identical cars. Together they have enough gas for one car to make a complete tour.
    Prove that there is a car that can make a complete tour by taking gas from all the cars that is encounters.
\end{problem}

\begin{soln}[Solution One]
    We will prove this by induction on $n.$ It is clearly true for $n=1.$

    Consider $n+1$ cars. Clearly there is a car, say $A$ has enough gas to reach another car, say $B.$ (Why there is such $A$? Prove by contradiction can help.)
    The consider a configuration without $B.$ $A$ now has the total amount of the gas of $A$ and $B.$ It is clear that this configuration satisfies the induction hypothesis.
    Thus, there is a car $C$ which can make a complete tour by taking gas from all the cars that is encounters.
    Now, if $C$ is $A$ then it is equivalent to the case $A$ goes to $B$ and takes $B$ gas and then continue.
    If $C$ is not $A,$ then $C$ will be able to reach $A,$ which then has enough gas to reach $B,$ and then the next car, which is not $A$.
\end{soln}

\begin{soln}[Solution Two]
    Let the cars be $C_1, C_2, \ldots, C_n.$ Let's imagine that we have a similar car $Z$ with a huge tank that has more than enough gas to complete a tour.
    Let this car $Z$ start from the location of some car $C_i$ on the circular route, and takes all the gas from all cars $C_1, \ldots , C_n$ it encounters,
    and complete the tour.
    
    WLOG, let $C_1$ be a car where the level of gas in the car $Z$ is the lowest when $Z$ encounters $C_1.$
    We claim that if the car $C_1$ is the starting point and travel in the same direction as the car $Z,$ then it will be able to
    complete a tour by taking gas from all the cars that it encounters on its way. 
    
    Suppose not, that $C_1$ runs out of gas when getting from car $C_j$ to car $C_{j+1}$ for some $1 \le j \le n - 1.$
    Then the amount of gas required to travel from car $C_1$ to car $C_{j+1}$ is more than the total amount of gas from cars $C_1, C_2, \ldots, C_{j+1}.$
    That implies that the level of gas in the car $Z$ when meeting car $C_j$ is strictly lower than its level of gas when $Z$ encountered $C_1,$
    which is a contradiction to the minimality of the level of gas when $Z$ encounters $C_1.$
    Therefore, $C_1$ can make a complete tour.
\end{soln}

\begin{problem}[Problem Eightteen]
    Santa Claus has $n \ge 2$ different presents and several identical bags. Each bag contains exactly two objects:
    either two bags, or a bag and a presents, or two presents. In particular, the big bag of presents Santa Claus has contains two objects.
    How many ways are there to split the presents into bags?
\end{problem}

\begin{soln}
    We show by induction that the number of ways is $(2n-3)!! = 1 \cdot 3 \cdot 5 \cdots (2n-3),$ and the number of needed bags is $n-1.$
    For $n=2$ we put two presents into a bag. There is no other way.

    Lets assume we have $n+1$ present, and we find a ways to put them in bags as required.
    Then the $(n=1)^{1^\text{st}}$ present is in a bag with another a bag, or another present.
    Let's remove the $(n=1)^{1^\text{st}}$ present, the bag that contains it, but not the content of the bag.
    Then the remaining configuration is a valid way.
    Conversely, given a valid way to distribute $n$ presents,
    we just need to add the $(n=1)^{1^\text{st}}$ present to any of the existing objects - a bag or a present into a new bag.

    Now, by induction, for $n$ present we use  $2n-1$ objects: $n$ present and $n-1$ bags, thus $2n-1$ objects.
    To add a new object we have $2n-1$ choices for any of these $2n-1$ objects and perform the procedure as described above.
    Thus the number of valid ways now is $(2n-3)!! \cdot (2n-1) = \boxed{(2n-1)!!.}$
\end{soln}

\begin{problem}[Problem Nineteen]
    Prove that if the graph $G$ is a tree with $n$ vertices then it has exactly $n-1$ edges.
\end{problem}

\begin{soln}
    Suppose we select a vertex of $G$ and start to walk on the edges from that vertex, forming a trail $v_1, v_2, \ldots$
    The trail cannot self-intersect, as then we would have a circle, which contradicts the existence of the tree.
    Thus the walk should eventually stop ar some vertex of degree 1, otherwise it is still possible to continue the walk.

    Now we prove that the tree contains $n-1$ edges. The case is clear if $n=1.$
    Suppose we have a tree with $n+1$ vertices. As above, we shall find a vertex with degree 1.
    By removing that vertex together with the edge connecting it,
    we obtain again a graph with $n$ vertices. The graph is still connected, and contains no cycle, thus it is a tree.
    Therefore the number of edges before removing the vertex is $n-1+1=n$.
\end{soln}

\begin{problem}[Problem Twenty]
    There are $n \ge 3$ distinct points on the plane, not all of them collinear.
    Prove that these point determine at least $n$ lines (each line through at least two points.)
\end{problem}

\begin{soln}
    We prove the given statement by induction. For $n=3,$ the statement is obvious.
    For $n > 3,$ by Sylvester-Gallai theorem, there is a ;ine passing through only two out of $P_1, P_2, \ldots, P_{n+1}$ points.
    WLOG, let that be the line through $P_nP_{n+1},$ If all $P_1, P_2, \ldots, P_{n}$ are collinear,
    then we have $n$ line through $P_{n+1}$ and each of the points $P_1, P_2, \ldots, P_{n},$ thus altogether $n+1$ lines.
    Otherwise we have $n$ distinct lines through pairs of $P_1, P_2, \ldots, P_{n}.$ These, together with the line $P_nP_{n+1},$ make $n+1$ lines.
\end{soln}

\begin{problem}[Problem Twenty One]
    Let $A_1, A_2, \ldots, A_n$ ($n \ge 4$) be a convex polygon. Denote by $R_i$ the radius of the circumcircle of the triangle $A_{i-1}A_{i}A_{i+1}$ ($i=2,3,\ldots,n$ and $A_{n+1} = A_1$.)
    Prove that if $R_2 = R_3 = \ldots = R_n$ then one can draw a circumcircle for the polygon $A_1A_2\ldots A_n.$
\end{problem}

\begin{lemma*}[Lemma 1]
    If the convex quadrilateral $MNPK$ cannot be inscribed in a circle, but the radii of the circumcircles of the triangles $MNP$ and $MNK$ are equal,
    then $\angle MPN + \angle MKN = 180\dg.$
\end{lemma*}

\begin{lemma*}[Lemma 2]
    Let $ABCDE$ be a convex pentagon such that $R_B = R_C = R_D,$ where $R_B, R_C,$ and $R_D$ denote the radii of the circumcircles of the triangles $ABC,$ $BCD,$ and $CDE,$ respectively.
    Then at least one of the quadrilaterals $ABCD$ or $BCDE$ can be inscribed in a circle.
\end{lemma*}

\begin{lemma*}[Lemma 3]
    Let $ABCDE$ be a convex pentagon such that $R_B = R_C = R_D = R_E$ where $R_B,$ $R_C,$ $R_D,$ and $R_E$
    denote the radii of the circumcircles of the triangles $ABC,$ $BCD,$ $CDE,$ and $DEA,$ respectively.
    Then at least one of the quadrilaterals $ABCD$ or $CDEF$ can be inscribed in a circle.
\end{lemma*}

\begin{soln}
    We leave the proofs of the lemmas for the students. We prove the given statement by induction.
    The case $n=$ is handled by the lemmas. For $n=5,$ applying Lemma 2 and Lemma 3 ($A=F$) we see that at least two of the quadrilaterals
    $A_1A_2A_3A_4,$ $A_2A_3A_4A_5,$ $A_3A_4A_5A_1$ can be inscribed in a circle.
    From this it follows that it is possible to draw a circumcircle for the pentagon $A_1A_2A_3A_4A_5.$
    
    For the induction step, we show that for $n \ge 5,$ then the case $n+1$ stands.
    Consider the quadrilaterals $A_1A_2A_3A_4,$ $A_2A_3A_4A_5,$ $A_3A_4A_5A_6,$ and  $A_4A_5A_6A_7,$ (with $n=6,$ consider $A_7=A_1$.)
    Using Lemma 2 and Lemma 3 we have that a pair of consecutive quadrilaterals in the list can be inscribed in a circle.

    Assume that these are $A_1A_2A_3A_4,$ $A_2A_3A_4A_5$ (the argument is basically same for other cases).
    Then it can be verified that the convex $n-$gon $A_1 A_2 A_4 A_5 \ldots A_n A_{n+1}$ satisfy the conditions, thus can be inscribed in a circle.
    Moreover $A_3$ is also on that circle. This completes our proof.
\end{soln}

\begin{problem}[Problem Twenty Two]
    Let $I$ be an interval, $f,g:\ I \rightarrow \RR$ be two $n$ times differentiable functions. Prove that
    \[
        (fg)^{(n)} = \binom{n}{0}f^{(n)}g + \binom{n}{1}f^{(n-1)}g^{(1)} + \cdots + \binom{n}{n-1}f^{(1)}g^{(n-1)} + \binom{n}{n}fg^{(n)}
    \]
    where $f^{(k)}$ denotes the $k^{\text{th}}$ derivative of $f.$
\end{problem}

\begin{soln}
    The key is the rule for derivative of a product
    \[
        (fg)' = f'g + fg'.
    \]

    The rest is simple with the Pascal Identity
    \[
        \binom{n}{k-1} + \binom{n}{k} = \binom{n+1}{k}.
    \]
\end{soln}

\end{document}