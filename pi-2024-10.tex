\documentclass{article}

\usepackage[main=english,vietnamese]{babel}
\usepackage[T1]{fontenc}
\usepackage[utf8]{inputenc}
\usepackage[sexy]{evan}
\usepackage{matchsticks}
\usepackage{wrapfig}
\usepackage{listings}

\newtheorem{hint}{Hint}

\title{Word Problems}
\author{Nghia Doan}
\date{\today}

\begin{document}

\maketitle

In this session we discuss solving words problem, that is, how to establish equations or systems of equations, and solve them.

\begin{example}[Example One]
    In eight years Henry will be three times the age that Sally was last year. Twenty five years ago their ages added to 83. How old is Henry now?
\end{example}

\begin{soln}
    Let $h$ represent Henry's current age, the $s$ represent Sally's current age. In 8 years, Henry will have age $h+8.$
    Last year Sally had age $s-1.$ Thus 
    \[
        h+8 = 3(s-1) \Rightarrow h = 3s -11 \quad (*)
    \]

    25 years ago their ages were $h-25$ and $s-25.$ Thus
    \[
        (h-25) + (s-25) = 83 \Rightarrow h + s = 133 \Rightarrow h = 133 - s \quad (**)
    \]

    Therefore, from (*) and (**)
    \[
        3s - 11 = 133 - s \Rightarrow 4s = 144 \Rightarrow s = 36,\ h = 133-36 = \boxed{97.}
    \]
\end{soln}

\begin{remark*}
    It is always important to test your answers. Here's how can you do it for the solution above.

    Let's test if these are the correct answers. In eight years, Henry will be $97 + 9 =105,$ which is $105 = 3\times 35,$ or 3 times $36-1=35,$
    which is Sally's age last year. 25 years agon Henry was $97-25=72$ and Sally was $36-15=11,$ thus their sum of ages was $72+11=83.$
    All given conditions are satisfied.
\end{remark*}

\begin{example}[Example Two]
    The lengths of the diagonals of a rhombus are, in inches, two consecutive integers.
    The area of the rhombus is 210 square inches. Find its perimeter, in inches.
\end{example}

\begin{soln}
    Let $x$ and $x+1$ be the lengths of the two diagonals. Then, from the formula of the area for rhombus,
    \[
        \half x(x+1) = 210 \Rightarrow x(x+1) = 420 = 20 \times 21 \Rightarrow x = 20.
    \]

    The side length $s$ of the rhombus is the hypotenuse of a right triangle with legs $10$ and $\frac{21}{2},$
    or
    \[
        \sqrt{10^2 + \left(\frac{21}{2}\right)^2 } = \sqrt{\frac{400 + 441}{4}} = \frac{29}{2}.
    \]

    Therefore the perimeter is $4 \times \frac{29}{2} = \boxed{58}.$
\end{soln}

\begin{example}[Example Three]
    How many gallons of a solution which is 15\% alcohol do we have to mix with a solution that is 35\% alcohol
    to make 250 gallons of a solution that is 21\% alcohol?
\end{example}

\begin{soln}
    Let $x$ represent the number of gallons of 15\% alcohol solution that we need to mix.
    To make 250 gallons, we need to mix the $x$ gallons of 15\% alcohol solution with $250-x$ gallons of the 35\% alcohol solution.
    
    The mix will then contain $(0.15)(x) + (0.35)(250-x)$ gallons of alcohol solution which should be 21\% of the total 250 gallons,
    in other words $(0.21)(250),$ therefore
    \[
        \begin{aligned}
            &(0.15)(x) + (0.35)(250-x) = (0.21)(250) \Rightarrow 15x + 35(250-x) = 21(250)\\
            &\Rightarrow 35(250) - 20x = 21(250) \Rightarrow 20 x = 14(250) \Rightarrow x = \boxed{175}.
        \end{aligned}
    \]
\end{soln}

\begin{example}[Example Four]
    A rectangle has area of 1100. If the length is increased by ten percent and the width is decreased by ten percent, what is the area of the rectangle?
\end{example}

\begin{soln}
    Let $l$ and $w$ be the length and width of the rectangle, then $lw = 1100.$
    Now, if the length is increased by ten percent and the width is decreased by ten percent, then the area of the modified rectangle is,
    \[
        1.1 l \cdot 0.9 w = (1.1 \cdot 0.9)lw = 0.99 lw = 0.99 \cdot 1100 = \boxed{1089.}
    \]
\end{soln}

\begin{example}[Example Five]
    A week ago, Sandy's seasonal Little League batting average was 360.
    After five more at bats this week, Sandy's batting average is up to 400.
    What is the smallest number of hits that Sandy could have had this season?
\end{example}

\begin{remark*}
    \textit{Batting average} is a statistic in cricket, baseball, and softball that measures the performance of batters.
    In baseball, the batting average is defined by the number of hits divided by at bats.
    It is usually reported to three decimal places and read without the decimal:
    A player with a batting average of $.300$ is \textit{batting three-hundred.}
\end{remark*}

\begin{soln}
    Suppose that a week ago Sandy had been at bat $x$ times. Then Sandy would have made $(0.360)x$ hits.
    Suppose that Sandy made $y$ more hits in the next five times at bat.
    That makes a total of $(0.360)x+y$ hits after $x+5$ times at bat, which gives a batting average of 400, thus
    \[
        \frac{(0.360)x+y}{x+5} = 0.400 \Rightarrow (0.360)x+y = (0.4)(x+5) \Rightarrow y = 0.04 x + 2
    \]
    
    We want to find the smallest number of hits, which is 
    \[
        0.36 x + 0.04 x + 2 = 0.4x + 2.
    \]

    Thus we want to find the smallest possible integer $x$ such that both $0.36 x$ and $0.04 x + 2$ are integers.
    Obviously the least possible integer value for the second expression is $3,$ which means $x = 25.$
    Then the least number of hits is $(0.4)25 + 2 = \boxed{12.}$
\end{soln}

\begin{example}[Example Six]
    Fill in numbers in the boxes below so that the sum of the entries in each three consecutive boxes is 2005.
    What is the number that goes into the leftmost box?
\end{example}

\begin{figure}[h]
    \centering
    \begin{tabular}{|l|l|l|l|l|l|l|l|l|}
    \hline
    &  & 999 &  &  &  &  & 888 & \\ \hline
    \end{tabular}
\end{figure}

\begin{soln}
    Suppose that four adjacent boxes contain the numbers $a,b,c,$ and $d$ as read from left to right, see the diagram below.
    \begin{figure}[h]
        \centering
        \begin{tabular}{|l|l|l|l|l|l|l|l|l|}
        \hline
         &  & 999 & $a$ & $b$ & $c$ & $d$ & 888 & \\ \hline
        \end{tabular}
    \end{figure}

    Then 
    \[
        \begin{cases}
            &a+b+c = 2005\\
            &b+c+d = 2005
        \end{cases}
        \Rightarrow a = d
    \]

    In order words, the number in the boxes repeat every three boxes as shown below
    \begin{figure}[h]
        \centering
        \begin{tabular}{|l|l|l|l|l|l|l|l|l|}
        \hline
         $a$ & $b$ & 999 & $a$ & $b$ & $c$ & $a$ & 888 & $c$ \\ \hline
        \end{tabular}
    \end{figure}

    Therefore $b=888,$ $c=999,$ and $a = 2005-(888+999)=\boxed{118.}$
\end{soln}

\begin{example}[Example Seven]
    Bill's age is one third larger than Tracy's age. In 30 years Bill's age will be one eighth larger than Tracy's age. How many years old is Bill?
\end{example}

\begin{proof}
    Let Bill's current age be $b$ and Tracy's current age be $t.$ Then 
    \[
        \begin{cases}
            &b = t \left(1 + \frac{1}{3}\right)\\
            &(b+30) = (t+30)\left(1 + \frac{1}{8}\right)\\
        \end{cases}
        \Rightarrow
        \begin{cases}
            &3b = 4t\\
            &8b+240 = 9t+270\\
        \end{cases}
        \Rightarrow 
        8b = 9\left( \frac{3}{4}\right) b + 30 
        \Rightarrow \frac{5}{4} b =30 \Rightarrow b = \boxed{24.}
    \]
\end{proof}

\begin{example}[Example Eight]
    At a movie theater tickets for adults cost 4 dollars more than tickets for children.
    One afternoon the theater sold 100 more child tickets than adult tickets for a total sales amount of 1475 dollars.
    How much money would the theater have taken in if the same tickets were sold,
    but the costs of the child tickets and adult tickets were reversed?
\end{example}

\begin{soln}
    Let $x$ be the cost of a child ticket, and let $a$ be the number of adult tickets sold.
    Then the cost of an adult ticket is $x+4$ and the number of child tickets sold is $a+100.$
    This means that
    \[
        (x+4)a + x(a+100) = 1475 \Rightarrow 2xa + 4a + 100x = 1475.
    \]

    If the cost of the tickets were reversed, then the theater would have taken in
    \[
        xa + (x+4)(a+100) = 2xa + 4a + 100x + 400 = 1475 + 400 = \boxed{1875.}
    \]
\end{soln}

\begin{example}[Example Nine]
    A rogue spaceship escapes. 54 minutes later the police leave in a spaceship in hot pursuit.
    If the police spaceship travels 12\% faster than the rogue spaceship along the same route,
    how many minutes will it take for the police to catch up with the rogues?
\end{example}

\begin{soln}
    Let $x$ be the number of minutes it will take the police to catch the rogues, and let $r$ be the rate at which the rouges are traveling.
    Then the police are traveling at a rate of $1.12r,$ and the rogues will have traveled for $54+x$ minutes.
    Since distance $=$ time $\times$ rate, thus
    \[
        (54 + x)r = x (1.12r) \Rightarrow 54r = (.12)xr \Rightarrow x = \frac{54}{.12} = \boxed{450.}
    \]
\end{soln}

\begin{example}[Example Ten]
    The sizes of the freshmen class and the sophomore class are in the ratio 5:4.
    The sizes of the sophomore class and the junior class are in the ratio 7:8.
    The sizes of the junior class and the senior class are in the ratio 9:7.
    If these four classes together have a total of 2158 students, how many of the students are freshmen?
\end{example}

\begin{soln}
    Let the sizes of the freshman, sophomore, junior, and senior classes be $w, x, y,$ and $z,$ respectively.
    Then
    \[
        4w = 5x,\ 8x = 7y,\ 7y= 9z,\ w+x+y+z = 2158
    \]

    By computing $x, y, z$ based on $w$ using the first three equations and substitute those into the last one:
    \[
        \begin{aligned}
            &x = \frac{4}{5}w,\ y=\ \frac{8}{7}x = \frac{8}{7} \cdot \frac{4}{5}w , z = \frac{7}{9} \frac{8}{7} \cdot \frac{4}{5}w\\
            &\Rightarrow 2158 = w + \frac{4}{5}w + \frac{32}{35}w + \frac{32}{45}w = \frac{1079}{315} w 
            \Rightarrow w = \frac{315 \cdot 2158}{1079} = \boxed{630.}
        \end{aligned}
    \]
\end{soln}

\end{document}