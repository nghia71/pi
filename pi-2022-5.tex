\documentclass{article}

\usepackage[main=english,vietnamese]{babel}
\usepackage[T1]{fontenc}
\usepackage[utf8]{inputenc}
\usepackage[sexy]{evan}
\usepackage{matchsticks}
\usepackage{wrapfig}
\usepackage{listings}

\title{The Power of Mathematical Reasoning - Part Two}
\author{Nghia Doan \& Catherine Doan}
\date{\today}

\begin{document}

\maketitle

\section*{Casework}

Sometimes, when the problem at hand is more complicated, 
looking at all possible cases, or caseworking, can help to find the correct answer.

\begin{example*}[Who stole what?]
    \label{example:pi-2022-4-p8}
    Father was not happy when he heard that Mother could not make his favourite cake because butter, eggs and milk all were stolen.
    Mother told him that she saw Chipmunk, Groundhog, and Sparrow sneaking out of the kitchen.
    Everyone was carrying something, and it was not clear who was carrying what.
    After a brief investigation all stolen ingredients were found at the places of Chipmunk, Groundhog, and Sparrow.
    Here were what they said,
    \begin{itemize}[topsep=0pt, partopsep=0pt, itemsep=0pt]
        \ii Chipmunk: Groundhog stole the butter.
        \ii Groundhog: Sparrow stole the eggs.
        \ii Sparrow: I stole the milk.
    \end{itemize}
    As it happened, \textit{the one, who stole the butter, told the truth} and \textit{the one, who stole the eggs, lied.}

    Who stole what?
\end{example*}

\begin{soln} \nameref{example:pi-2022-4-p8}
    Let's try casework on \textit{who stole the butter.}
    \begin{itemize}[topsep=0pt, partopsep=0pt, itemsep=0pt]
        \ii \textit{Case 1:} Assume that Sparrow have stolen the butter. 
        Then she would told the truth, but she said she stole the milk. So it was not possible.
        \ii \textit{Case 2:} If Chipmunk stole the butter, then what she said about Groundhog is true,
        thus Groundhog stole the butter, which is a contradiction.
        \ii \textit{Case 3:} Groundhog must have stolen the butter. 
        Thus what Groundhog said was true, so Sparrow stole the eggs, therefore Chipmunk stole the milk.
    \end{itemize}    
    The answer is \framebox{Groundhog stole the butter, Sparrow stole the eggs, and Chipmunk stole the milk.}
\end{soln}

\section*{Knights and Liars, Patients and Doctors, and other stories}

Classic stories have some twists. Inductive reasoning leads us to a general conclusion.
Deductive reasoning applies a general assumption to a specific case.

\begin{example*}[Who killed the dragon?]
    \label{example:pi-2022-4-p10}
    On the Island of Knights and Liars, there are two types of people:
    the knights who always tell the truth and the liars who always lie.

    Three mighty warriors live on the island.
    Their names are Albert, David, and Victor.
    Two of them are Liars, and one is a Knight.
    The friends keep their secrets so nobody know who was what.

    One of their great deeds was a battle with a terrible dragon
    that was terrorizing the students of the Math, Chess, and Codding Club.
    Not much is known about this battle except that the dragon was slayed by a Knight.
    In a recently discovered letter, Albert stated that Victor had slayed the dragon
    and David were merely watching.

    Who actually killed the dragon?
\end{example*}

\begin{soln} \nameref{example:pi-2022-4-p10}
    If Albert is a Knight, then Victor slayed the dragon.
    Because the dragon slayer was a Knight, this would make Victor a Knight,
    which is impossible because only one of them is a Knight.
    Therefore Albert is a Liar, which means that Victor did not kill the dragon.
    
    Thus, \framebox{David was the dragon-slayer.}
\end{soln}

\begin{example*}[Who were removed?]
    \label{example:pi-2022-4-p11}
    In the Mental Hospital of Geniuses there are only doctors and patients.
    Each of these inhabitants is sane or insane, but cannot be both.
    The sane people were a hundred percent accurate in all of their beliefs, 
    and the insane people were a hundred percent inaccurate in all of their beliefs.
    
    After being in the hospital for a while, some patients got cured and became sane.
    Unfortunately some doctors lost their minds and became insane.
    The sane patients and insane doctors, if found, were removed from the hospital.
    
    One day, Inspector Melanie visited the hospital in order to determine who should be removed.
    She interviewed three people $A$, $B$, and $C$:
    \begin{itemize}[topsep=0pt, partopsep=0pt, itemsep=0pt]
        \ii $A$ said $B$ was insane.
        \ii $B$ said $A$ is a doctor.
        \ii $C$ said $B$ is a patient and $A$ is insane.
    \end{itemize}

    Did Inspector Melanie remove $A$, $B$, or both of them?
\end{example*}

\begin{soln} \nameref{example:pi-2022-4-p11}
    Suppose that $A$ was sane. Then, what $A$ said was true, so $B$ was insane, and therefore $B$'s belief that $A$ was a doctor
    was false, thus, $A$ is a sane patient and should be removed.
    $C$ said $A$ was insane, so $C$ was insane, thus $B$ was a doctor, thus $B$ should be removed as well.
    
    Now, if $A$ was insane, then $B$ was sane, and therefore $A$ was a insane doctor and should be removed.
    $C$ said $A$ was insane, so $C$ was sane, thus $B$ was patient, so $B$ should also be removed.
    
    Thus, \framebox{both $A$ and $B$ should be removed.}
\end{soln}

\section*{Exercises}

\begin{exercise*}[Which room has the tiger?]
    \label{exercise:pi-2022-4-p9}
    Karl was captured while sneaking into the Kingdom of the Hungry Tigers.
    He was lead in front of three rooms, each with a separate door marked with a sign,
    as shown below in \Cref{fig:pi-2022-4-p9}.
    \begin{figure}[h]
        \centering
        \begin{tabular}{|c|c|c|}
        \hline
        I & II & III \\
        Room III is empty & The tiger is in Room I & This room is empty \\ \hline
        \end{tabular}
        \caption{\nameref{exercise:pi-2022-4-p9}}
        \label{fig:pi-2022-4-p9}
    \end{figure}
    A large treasure chest was placed in one of the rooms and a hungry tiger in another.
    A beautiful girl told him that,
    \begin{enumerate}[topsep=0pt, partopsep=0pt, itemsep=0pt]
        \ii The sign on the door of the room containing the treasure was true,
        \ii The sign on the door of the room with the tiger was false, and
        \ii The sign on the door of the empty room could be either true or false.
        \ii If he opens the room with the tiger, he will be eaten.
        \ii If he opens the room with the chest, they will set him free and give him the chest.
    \end{enumerate}

    Which room has the tiger?
\end{exercise*}

\begin{soln} \nameref{exercise:pi-2022-4-p9}
    Instead of looking for the tiger, let's \textit{look for the treasure, because the sign on its room is true.}
    
    It cannot be in room II, because then according to the sign, room I has the tiger, room III is empty.
    Thus, the sign on room I is true, which contradicts that the sign on the room with the tiger is false.
    
    It cannot be in room III, because the according to the sign, room III is empty.
    Therefore the treasure is in room I, room III is empty, and the tiger is in room II.
\end{soln}

\begin{exercise*}[Was Sam awake?]
    \label{exercise:pi-2022-4-p12}
    Whenever Rob the bandit is asleep, everything he believes is wrong.
    In other words, everything Rob believes in his sleep is false.
    On the other hand, everything he believes while he is awake is true.
    Last night at $10$ o'clock sharp,
    while guarding the newly captured treasure chest,
    Rob believed that he and Sam the thug were asleep at that time.
    
    Was Sam asleep or awake at that time?
\end{exercise*}

\begin{soln} \nameref{exercise:pi-2022-4-p12}
    If Rob the bandit was awake at the time, he could not have had \textit{the false belief
    that both he and Sam the thug were asleep.}
    Therefore he was asleep. This means that his belief was false,
    so it is not true that both were asleep.
    Therefore Sam was awake.
\end{soln}

\end{document}